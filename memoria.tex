% -*-memoria.tex-*-
% Este fichero es parte de la plantilla LaTeX para
% la realización de Proyectos Final de Carrera, protejido
% bajo los términos de la licencia GFDL.
% Para más información, la licencia completa viene incluida en el
% fichero fdl-1.3.tex

% Copyright (C) 2009 Pablo Recio Quijano 

%-------------------------------------------------------
% ---- Plantilla para libros / memorias PFC -----

% Realizada por Pablo Recio Quijano y Noelia Sales Montes 
% Formato de portada y primera página tomado del PFC de
% Francisco Javier Vázquez Púa, en su proyecto 'libgann'
% -------------------------------------------------------

\documentclass[a4paper,11pt]{book}

\usepackage{./estilos/estiloBase} % Basicamente son todas las
                                  % bibliotecas usadas. En caso de que
                                  % falten bibliotecas se van añadiendo
                                  % al fichero.
\usepackage{./estilos/colores}  % Algunos colores ya generados, para
                                % los algunos estilos más avanzados.
\usepackage{./estilos/mis_macros} % Algunos macros personalizados

\usepackage{./estilos/mips} % Resaltado de código MIPS

\graphicspath{{./imagenes/}} % Indicamos la ruta donde se encuentran
                             % las imagenes, para ahorrarnos la ruta
                             % completa, y solo modificar aquí si en
                             % un momento dado lo movemos


\begin{document}

% Renombramos las figuras y las tablas
\renewcommand{\figurename}{Figura}
\renewcommand{\listfigurename}{Índice de figuras}
\renewcommand{\tablename}{Tabla}
\renewcommand{\listtablename}{Índice de tablas}

\pagestyle{empty}
% -*-portada.tex-*-
% Este fichero es parte de la plantilla LaTeX para
% la realización de Proyectos Final de Carrera, protejido
% bajo los términos de la licencia GFDL.
% Para más información, la licencia completa viene incluida en el
% fichero fdl-1.3.tex

% Fuente tomada del PFC 'libgann' de Javier Vázquez Púa

\begin{titlepage}

  \begin{center}

    \includegraphics[scale=0.35]{upv.png} \\
    
    \vspace{2.0cm}
    
    \LARGE{\textbf{E.T.S.I. TELECOMUNICACIÓN}} \\
    
    \vspace{1.0cm}
    
    \Large{\textbf{INGENIERÍA SUPERIOR EN TELECOMUNICACIÓN}} \\
    
    \vspace{3.0cm}
%mimateo: ¿seguro que quedamso en poner este título? 
%lluis: No, está esto puesto solamente de forma temporal. 
    \Large{Evaluación de Arduino sobre plataforma MIPS} \\
    
    \vspace{2.0cm}
    
    \Large{Lluís Satorre González} \\
    \vspace{0.5cm}
    
    \large{Director: Miguel Ángel Mateo Pla} \\
  
    \vspace{0.5cm}

    \large{\today}
    
  \end{center}
\end{titlepage}

\cleardoublepage

% -*-primerahoja.tex-*-
% Este fichero es parte de la plantilla LaTeX para
% la realización de Proyectos Final de Carrera, protejido
% bajo los términos de la licencia GFDL.
% Para más información, la licencia completa viene incluida en el
% fichero fdl-1.3.tex

% Fuente tomada del PFC 'libgann' de Javier Vázquez Púa

\begin{center}

  \includegraphics[scale=0.35]{upv.png} \\

  \vspace{2.0cm}

  \Large{ESCUELA SUPERIOR DE INGENIERÍA} \\

  \vspace{1.0cm}

  \large{INGENIERO SUPERIOR EN TELECOMUNICACIÓN} \\

  \vspace{2.0cm}

  \large{PLANTILLA PARA PROYECTO DE EJEMPLO} \\

  \vspace{1.0cm}

\end{center}

\begin{itemize}
\item \large{Departamento: Sistemas Informáticos y Computación}
\item \large{Director del proyecto: Miguel Ángel Mateo Pla}
\item \large{Autor del proyecto: Lluís Satorre González}
\end{itemize}

\vspace{1.0cm}

\begin{flushright}
  \large{Cádiz, \today} \\

  \vspace{2.5cm}

  \large{Fdo: Pablo Recio Quijano}
\end{flushright}

\cleardoublepage
\pagestyle{plain}

\frontmatter % Introducción, índices ...

% -*-previo.tex-*-
% Este fichero es parte de la plantilla LaTeX para
% la realización de Proyectos Final de Carrera, protejido
% bajo los términos de la licencia GFDL.
% Para más información, la licencia completa viene incluida en el
% fichero fdl-1.3.tex

% Copyright (C) 2009 Pablo Recio Quijano 

\section*{Agradecimientos}

Me gustaría agradecer a mi familia por todo su apoyo a lo largo de la carrera y a la amplia documentación de los programas GNU usados (GCC y Make), que ha facilitado enormemente la labor.

\cleardoublepage

%\section*{Licencia} % Por ejemplo GFDL, aunque puede ser cualquiera

%Este documento ha sido liberado bajo Licencia GFDL 1.3 (GNU Free
%Documentation License). Se incluyen los términos de la licencia en
%inglés al final del mismo.\\
%
%Copyright (c) 2009 Pablo Recio Quijano.\\
%
%Permission is granted to copy, distribute and/or modify this document under the
%terms of the GNU Free Documentation License, Version 1.3 or any later version
%published by the Free Software Foundation; with no Invariant Sections, no
%Front-Cover Texts, and no Back-Cover Texts. A copy of the license is included in
%the section entitled "GNU Free Documentation License".\\

\cleardoublepage

\section*{Notación y formato}

Cuando nos refiramos a un programa en concreto, utilizaremos la
notación: \\ \programa{vim}.\\

%mimateo: por favor no uses el término COMANDO en la memoria, es algo que a mi me molesta....
%mimateo: es más, comandos.sty debería ser mis_macros.sty o mis_ordenes.sty y la etiquetea \comandos debería ser \comand o \orden
Cuando nos refiramos a una orden, un elemento o una función de un lenguaje, usaremos
la notación: \\ \orden{quicksort}.\\

\cleardoublepage

\tableofcontents
\listoffigures
\listoftables

\mainmatter % Contenido en si ...

\chapter{Motivación y contexto del proyecto}
\label{cap1}
% -*-cap1.tex-*-
% Este fichero es parte de la plantilla LaTeX para
% la realización de Proyectos Final de Carrera, protejido
% bajo los términos de la licencia GFDL.
% Para más información, la licencia completa viene incluida en el
% fichero fdl-1.3.tex

% Copyright (C) 2009 Pablo Recio Quijano 

\section{Introducción}
La motivación para la realización de este proyecto viene a raíz de descubrir a principios del 2011 la plataforma Arduino. Hasta entonces todo lo que había programado antes tenía como objetivo mostrar un resultado de algún tipo en la pantalla de un ordenador, mientras que con Arduino tenía una manera sencilla de programar con resultados físicos más allá de la modificación del estado de unos píxeles en el monitor de mi PC. Gracias a su sencilla biblioteca es posible crear desde un pequeño programa que encienda y apague un LED con apenas nueve líneas y sabiendo solamente a que pin del procesador se encuentra conectado el LED hasta impresoras 3D\footnote{\url{http://reprapbook.appspot.com/}}.\\

Además, el hardware y software de Arduino es libre. Cualquiera puede descargarse los esquemáticos para ensamblar o modificar su propia placa y tiene acceso al código fuente de la biblioteca. También, el diseño de las placas Arduino (y ChipKit) facilita la creación de circuitos impresos que se pueden poner encima de la placa para añadir funcionalidades extra, como conexión Ethernet o control de motores.\\

Para la realización de este proyecto se ha optado por utilizar el entorno ChipKit, que es compatible con Arduino. La elección de esta plataforma tiene varios motivos, el primero es que posee la misma estructura de programa y mismas bibliotecas que cualquier placa Arduino. Otro motivo es el rendimiento que ChipKit ofrece y la memoria disponible con este entorno. Como se puede observar en la tabla~\ref{tab:arduino} la solución que nos ofrece Digilent (empresa que produce ChipKit) dispone de un procesador que funciona a cinco veces más frecuencia que su versión de Arduino y (comparando el Arduino Mega frente a Chipkit Max32) el doble de memoria flash, y 16 veces mas SRAM.

\begin{table}[H]
\begin{center}
\begin{tabular}{| c || c | c | c |}
    \hline
    Modelo & Frecuencia Reloj & Memoria Flash & SRAM\\
    \hline
    Arduino Uno   & 16MHz & 32KB  & 2KB   \\
    \hline
    Arduino MEGA  & 16MHz & 256KB & 8KB   \\
    \hline
    Chipkit Uno32 & 80MHz & 128KB & 16KB  \\
    \hline
    Chipkit Max32 & 80MHz & 512KB & 128KB \\
    \hline
  \end{tabular}
\end{center}
\caption{Comparación de modelos equivalentes Arduino y Chipkit}
\label{tab:arduino}
\end{table}

Chipkit, además, utiliza un chip PIC32 de Microchip (Max32 usa 32MX795F512L) con arquitectura MIPS de 32 bits, frente a la arquitectura AVR de 8 bits que podemos encontrar en las placas Arduino con chips ATMEL. Recientemente Arduino ha sacado al mercado una placa con chip basado en ARM Cortex-M (Arduino Due\footnote{\url{http://arduino.cc/en/Main/ArduinoBoardDue}}) para solucionar el problema de rendimiento, pero esto se sale fuera del objetivo de este proyecto. ChipKit también ofrece la posibilidad de utilizar herramientas de depuración, con las que podemos ejecutar el programa paso a paso, ver el estado de registros y memoria, etc. Para poder realizar el proceso de depuración se hará uso del ChipKit PGM, un depurador diseñado y construido expresamente para la plataforma ChipKit.

\section{Objetivos}
Podemos dividir los objetivos de la siguiente forma:
\begin{enumerate}
    \item Describir el proceso de compilación del IDE de ChipKit. Trataremos de describir como MPIDE lleva acabo la compilación del programa y su posterior carga en el microcontrolador.
    \item Proponer alternativa al IDE de ChipKit. Sabiendo como realiza la compilación el IDE de ChipKit intentaremos proponer una alternativa basada en el uso de archivos Makefile que nos facilitaran la labor de automatización de todo el proceso.
    \item Analizar las optimizaciones que lleva acabo el compilador según el nivel de optimización. Veremos que optimizaciones se llevan a cabo y cual es la finalidad de cada una de ellas.
    \item Escribir un programa con el que podamos llevar acabo los siguientes puntos. Escribiremos un programa que tendrá funciones compiladas con diferentes niveles de optimización.
    \item Comparar los efectos de las diferentes optimizaciones en el código ensamblador. Usando un desensamblador veremos los efectos que han tenido las optimizaciones en el lenguaje ensamblador de las funciones del programa.
    \item Medir las diferencias de las distintas optimizaciones en aplicaciones reales. Usaremos diferentes métodos de profiling para medir el efecto de las optimizaciones en la aplicación.
    \item Comparar la biblioteca Arduino con el uso de las bibliotecas de microchip de más bajo nivel. La biblioteca Arduino es una que está por encima de las bibliotecas que provee Microchip y que simplifica su uso. Veremos cual es la diferencia en rendimiento si usamos las funciones de la biblioteca de Microchip directamente frente a las proporcionadas por la biblioteca Arduino.
\end{enumerate}

\chapter{Hardware: Arduino y ChipKIT}
\label{cap2}
% -*-cap2.tex-*-
% Este fichero es parte de la plantilla LaTeX para
% la realización de Proyectos Final de Carrera, protejido
% bajo los términos de la licencia GFDL.
% Para más información, la licencia completa viene incluida en el
% fichero fdl-1.3.tex

% Copyright (C) 2009 Pablo Recio Quijano 

\section{Hardware}
En esta sección se describirán los dispositivos que van a ser utilizados para la realización del proyecto.

\subsection{ChipKIT Max32}
El microcontrolador elegido para el proyecto es un ChipKIT Max32. Es el modelo más avanzado de la familia Chipkit, que se compone además del Uno32, uC32 y DP32.

\figura{chipKIT-Max32.jpg}{scale=0.6}{Chipkit Max32}{max32}{H}

En el centro del Max32 encontramos como microcontrolador un Microchip PIC32MX795F12L que funciona a 80MHz y con el que disponemos de 512KB de memoria flash y 128KB de memoria RAM. El Max32 tiene 83 puertos de entrada/salida además de otros dispositivos como UART, SPI e I2C. Dieciséis de los 83 puertos puede ser usados como entradas analógicas o como entradas/salidas digitales. A diferencia de su equivalente Arduino (Mega), el ChipKIT Max32 también tiene integrado en el chip 10/100 Ethernet, un controlador USB 2.0 y dos controladores CAN, aunque para hacer uso de estos dispositivos necesitaremos el ChipKIT Network Shield.\\

Este dispositivo, al igual que el resto de microcontroladores de la familia ChipKIT, funciona con un voltaje lógico de 3.3V frente a los microcontroladores Arduino como el Uno o el Mega. Es importante tener este dato en cuenta a la hora de usar shields o dispositivos pensados para placas Arduino.\\

Podemos encontrar más información además de esquemáticos y manuales en la página de información del ChipKIT Max32\cite{website:max32}.\\

Características del ChipKIT Max32:
\begin{itemize}
	\item Procesador Microchip PIC32MX795F512L
		\begin{itemize}
			\item 80MHz 32-bit MIPS
			\item 512KB FLash, 128KB RAM
			\item Controlador USB 2.0 OTG
			\item 10/100 Ethernet MAC
			\item Dos controladores CAN
		\end{itemize}
	\item Proporciona memoria adicional (con respecto al Uno32) y periféricos de comunicación avanzados
	\item Compatible con gran parte del código existente de ejemplo para Arduino, así como con materiales de referencia y otros recursos.
	\item Puede ser programado usando Microchip MPLAB, mediante un programador gracias a su interfaz JTAG.
	\item Mismo tamaño y formato que el Arduino Mega
	\item 83 pines de entrada/salida
	\item LED de usuario (Pin 13)
	\item Se conecta a un PC usando un cable USB A -> mini B
\end{itemize}

\subsubsection{Chipkit PGM}
Para la depuración de los programas con MPLAB usaremos el programador chipKIT PGM\cite{website:pgm}, con el que podremos usar el puerto JTAG del Max32. Este programador está diseñado para ser usado exclusivamente con la familia de dispositivos chipKIT, a diferencia de picKIT3\cite{website:pickit} que puede ser usado con todos los microcontroladores de Microchip.

\figura{chipKIT-PGM.png}{scale=0.3}{ChipKIT PGM}{pgm}{H}

\newpage

\subsection{ChipKIT Network Shield}

El ChipKIT Network Shield es, como su nombre indica, un shield para el ChipKIT Max32. Cabe destacar que este shield no sirve para el ChipKIT Uno32 puesto que este microcontrolador no dispone de las características de las que hace uso este shield, como Ethernet y host USB.Se coloca encima del microcontrolador tal y como lo hace cualquier shield de Arduino.

\figura{chipKIT-NetworkShield.jpg}{scale=0.5}{Chipkit Network Shield}{networkshield}{H}

Este shield nos proporciona las conexiones necesarias para poder hacer uso del controlador Ethernet y USB que lleva el Max32 integrado y a los buses CAN. Además de esto también tiene integrado un oscilador de 32.768 KHz que nos permite hacer uso del RTC que también tiene integrado el Max32 y una memoria EEPROM de 256Kb a la que podemos acceder mediante I2C. En la página web del Network Shield\cite{website:network_shield} podemos encontrar las bibliotecas que nos sirven para hacer uso de sus características, en el entorno ChipKIT.\\

Características del ChipKIT Network Shield:
\begin{itemize}
	\item Puede ser usado con el Max32
	\item SMSC LAN8720 10/100 Ethernet PHY
	\item Conector RJ45
	\item Dispositivo USB y conectores Host
	\item Dos transceptores CAN MCP2551
	\item Dos conectores de 12 pines para los buses CAN
	\item Dos conectores para cadena margarita (daisy chain) I2C
	\item EEPROM de 256Kbits con interfaz I2C
	\item Oscilador a 32.768 KHz
\end{itemize}

\newpage


\subsection{ChipKIT Basic I/O Shield}
El otro shield que usaremos será el ChipKIT Basic I/O Shield que nos servirá para interactuar físicamente con el microcontrolador ya que este shield nos proporciona diferentes métodos de entrada y salida de información.

\figura{chipKIT-BasicIOShield.jpg}{scale=0.5}{ChipKIT Basic I/O Shield}{ioshield}{H}

El ChipKIT Basic I/O shield dispone de los siguientes dispositivos:
\begin{itemize}
	\item EEPROM de 256Kbits con interfaz I2C
	\item Pantalla OLED de 128x32 píxeles
	\item Sensor de temperatura con interfaz I2C
	\item Conector para cadena margarita (daisy chain) I2C
	\item Cuatro drivers FET de colector abierto
	\item Potenciometro analógico
	\item Cuatro pulsadores y cuatro interuptores
	\item Ocho LEDs
\end{itemize}

Este shield está inicialmente pensado para ser usado en un chipKIT Uno32, pero podemos conectarlo al Max32 sin ningún problema ya que comparten el mismo esquema de pines (aunque el Max32 tiene 83 pines frente a los 42 del Uno32). También es posible conectarlo al Network Shield y este a su vez al Max32, teniendo en cuenta la posibilidad de que algunas funciones en los dos shields puedan compartir pines.\\

Igual que el Network Shield, el Basic I/O Shield también dispone de una página web\cite{website:io_shield} en la que podemos encontrar su manual, esquemáticos y las librerías necesarias para usarlas en el entorno ChipKIT.

\newpage

\section{Software} A lo largo del proyecto utilizaremos dos entornos de
desarrollo integrado: \programa{MPIDE} y \programa{MPLAB X}.
\subsection{\programa{MPIDE}} \programa{MPIDE}\cite{website:mpide} es el IDE
creado por Digilent para el entorno ChipKIT. Su diseño es idéntico al IDE de
Arduino, variando solamente el color y el compilador usado.
\figura{mpide.jpg}{scale=0.6}{\programa{MPIDE}}{mpide}{H} \programa{MPIDE} es
un entorno de desarrollo sencillo y con unas características muy básicas y
limitadas. Con él podremos escribir nuestro programa, compilarlo y cargarlo en
nuestro microcontrolador sin necesidad de ninguna otra aplicación, siempre y
cuando hayamos seleccionado el modelo adecuado y el puerto serie en las
opciones del programa. \\ Al igual que el entorno de Arduino, no es posible
depurar programas, para eso usaremos \programa{MPLAB X}. También dispone de un monitor
para el puerto serie, con el que podemos leer lo que el microcontrolador envía
a través de este puerto y enviar información al microcontrolador.
\subsection{\programa{MPLAB X}} \programa{MPLAB X}\cite{website:mplab} es la última versión del
entorno de desarrollo integrado de Microchip. A diferencia de
\programa{MPIDE}, este programa es un IDE completo, basado en Netbeans. Está
pensado para ser utilizado con toda su gama de microcontroladores y, con la
ayuda de un depurador/programador como el chipKIT PGM podremos depurar
nuestros programas. Si queremos, también podemos usar los productos chipKIT
como cualquier otro microcontrolador de Microchip, escribiendo los programas
en este IDE y compilándolos con el compilador que tiene Microchip disponible
en su página web, C32\cite{website:c32}. \figura{mplab.jpg}{scale=0.35}{MPLAB
X Depurando}{mplab}{H}

Como se aprecia en la Figura \ref{mplab} \programa{MPLAB X} nos ofrece mucha más información y posibilidades que \programa{MPIDE}. En esta figura vemos a \programa{MPLAB X} en mitad de la depuración del programa Blink, que fue compilado usando \programa{MPIDE}.\\
Con este entorno de desarrollo veremos más adelante los diferentes efectos de que los distintos niveles de optimización tienen en nuestro programa y podremos usarla para hacer profiling.


\section{Estructura de un programa Arduino}
En cualquier programa para Chipkit o Arduino podemos encontrar tres grupos principales de código fuente.
\subsection{Core}
Para empezar, tenemos los archivos que formarán parte de la biblioteca ``Core'' a la hora de la compilación. Estos archivos vienen incluidos con \programa{MPIDE} y los podemos encontrar en el directorio en el que tengamos instalado \programa{MPIDE}\footnote{La carpeta de \programa{MPIDE} la podemos colocar donde queramos, pues no requiere de instalación}, en la subcarpeta:\\
\comando{hardware/pic32/cores/pic32}
\begin{table}[H]
\begin{center}
\begin{tabular}{| c || c |}
    \hline
    Sistema Operativo & Ruta\\
    \hline
    \hline
    Linux & /home/lluis/mpide/hardware/pic32/cores/pic32\\
    \hline
    Mac OSX & /Applications/Mpide.app/Contents/Resources/Java/hardware/pic32/cores/pic32\\
    \hline
    Windows & C:\textbackslash Archivos de Programa\textbackslash \programa{MPIDE}\textbackslash hardware\textbackslash pic32\textbackslash cores\textbackslash pic32\\
    \hline
  \end{tabular}
\end{center}
\caption{Ruta de la biblioteca Core}
\label{tab:core_path}
\end{table}
En esta biblioteca tenemos ya definidas las funciones que necesitamos para interactuar con los dispositivos que están integrados en la placa (en nuestro caso ChipKIT Max32). Para más información acerca de estas funciones y como utilizarlas podemos acudir a la página de referencia de Arduino\cite{website:arduino_funcs}.
Además, en esta biblioteca está ya definida la función main de nuestro programa y tiene el siguiente aspecto:
\lstset{language=C++}
\begin{lstlisting}[caption=main.cpp]
#define OPT_SYSTEM_INTERNAL
#include <System_Defs.h>

#if (ARDUINO >= 100)
	#include <Arduino.h>
#else
	#include <WProgram.h>
#endif

int main(void)
{
	init();
	setup();
	
	while(1)
	{
		_scheduleTask();
		loop();
	}
	return 0;
}
\end{lstlisting}

\subsection{Sketch}
Los programas que escribimos usando \programa{MPIDE} (o el IDE de Arduino) se llaman sketches. En realidad, no escribimos un programa completo porque, como acabamos de ver, el programa en sí ya está escrito en la biblioteca Core. Nuestro objetivo es implementar dos funciones: \comando{setup()} y \comando{loop()}, que ya están declaradas en el archivo de cabecera wiring.h.\\
\begin{lstlisting}[caption=Sección de wiring.h]
void setup(void);
void loop(void);
\end{lstlisting}

Como se puede apreciar en la declaración de ambas funciones, estas no esperan ni devuelven ningún valor.\\
\lstinputlisting[language=C++, caption=Sketch básico ("BareMinimum.pde")]{codigo_fuente/BareMinimum.pde}
En \comando{setup()} colocaremos el código para la inicialización de nuestro programa. Es el lugar donde podemos configurar el puerto serie, por ejemplo, o definir como se comportarán los pines (entrada o salida), entre otras cosas. En definitiva, es el lugar donde debemos colocar el código que queramos que se ejecuta una vez al inicio del programa y que no vuelva a ejecutarse más.\\
En \comando{loop()} pondremos el código que se ejecutara de forma cíclica indefinidamente ya que, como podemos observar en main.cpp, la función \comando{loop()} se encuentra dentro de un bucle infinito.\\

Veamos como ejemplo el sketch Blink:
\lstinputlisting[language=C++, caption=Blink.pde, label=code:blink]{codigo_fuente/Blink.pde}

En este sketch utilizamos las funciones de Arduino pinMode,digitalWrite y delay. En la función \comando{setup()} inicializamos el puerto al que está conectado el LED como una salida\footnote{PIN\_LED1 está definido en el archivo de cabecera Board\_Defs.h que depende del modelo de microcontrolador que estemos usando}, mientras que en \comando{loop()} encendemos y apagamos el LED con retardos de un segundo.

\subsection{Bibliotecas de terceros}
A parte del código de Arduino y del que creemos nosotros, también es posible añadir bibliotecas de terceros. Gracias a la amplia comunidad detrás de Arduino podemos encontrar cientos de ellas. A la hora de usar una biblioteca de Arduino en ChipKIT tendremos que asegurarnos de que no haga uso de funciones específicas de los procesadores Atmel que encontramos en las placas Arduino, ya que no funcionarán en PIC32.\\
Cuando trabajamos con \programa{MPIDE} podemos encontrar las bibliotecas en dos sitios diferentes. \programa{MPIDE} tiene incluidas varias bibliotecas que podemos encontrar en el directorio ``libraries'' dentro de la carpeta en la que esté instalado \programa{MPIDE}. A parte de estas bibliotecas podemos instalar bibliotecas que descarguemos de Internet o que creemos nosotros dentro de la carpeta ``libraries'' en el directorio ``sketchbook''\footnote{La carpeta ``libraries'' no existe. Es neciesario crearla dentro del directorio ``sketchbook''.} que \programa{MPIDE} habrá creado la primera vez que lo ejecutemos.\\
Con las bibliotecas instaladas en uno de los dos directorios podremos añadirlas a nuestro programa de dos formas diferentes. Podemos añadirlas usando el menú de MPIDE, pudiéndolas encontrar en \comando{Sketch > Import Library}. También es posible usar la directiva \comando{\#include <[Archivo De Cabezera].h>} como haríamos en cualquier programa escrito en C/C++. \programa{MPIDE} se encargará de compilar las bibliotecas y enlazarlas con nuestro programa.

\section{Proceso de compilación de \programa{MPIDE}}
En este apartado trataremos de describir el proceso que sigue \programa{MPIDE} para la compilación y posterior carga en el microcontrolador del programa creado en \programa{MPIDE}. Nuestro objetivo será saber que programas y opciones de compilación usa el IDE de ChipKIT.\\
También es interesante conocer que archivos genera \programa{MPIDE} a lo largo de la compilación, pero estos archivos son borrados una vez termina la compilación del programa por lo que tendremos que cambiar este comportamiento. Para ello acudiremos al archivo de configuración \comando{preferences.txt} cuya localización varía según el sistema operativo empleado.

\begin{table}[H]
\begin{center}
\begin{tabular}{| c || c |}
    \hline
    Sistema Operativo & Ruta\\
    \hline
    \hline
    Linux & /home/[nombre de usuario]/.mpide/preferences.txt\\
    \hline
    Mac OSX & /Users/[nombre de usuario]/Library/mpide/preferences.txt\\
    \hline
    Windows & C:\textbackslash Archivos de Programa\textbackslash \programa{MPIDE}\textbackslash hardware\textbackslash pic32\textbackslash cores\textbackslash pic32\\
    \hline
  \end{tabular}
\end{center}
\caption{Ruta del archivo \comando{preferences.txt}}
\label{tab:preferences_path}
\end{table}

El archivo \comando{preferences.txt} contiene toda la configuración de MPIDE a excepción de la configuración del compilador, pero nosotros estamos interesados en la línea 75:
\begin{lstlisting}
preproc.save_build_files=false
\end{lstlisting}

Cambiando el valor a \comando{true} \programa{MPIDE} guardará los archivos temporales de la compilación.\\
Para ver como funciona el proceso de compilación en \programa{MPIDE} usaremos el sketch Blink (\ref{code:blink}).\\
Para ver que se ejecuta cuando compilamos nuestro programa en \programa{MPIDE} podemos mantaner pulsado la tecla \comando{Mayus} mientras pulsamos el botón de compilar. De esta forma \programa{MPIDE} nos mostrará todos los comandos que ejecute, en el terminal.

\figura{mpide_compilado.jpg}{scale=0.6}{MPIDE después de compilar}{mpide_compilado}{H}

El proceso completo de compilación (lo que se muestra en el terminal) está en el Apéndice \ref{chap:apendice1}\\

Veamos que pasos sigue \programa{MPIDE} durante la compilación y ensamblado de nuestro programa:
\begin{enumerate}
	\item Compila, usando \programa{pic32-g++} (sin enlazar) Blink.cpp (este archivo es una copia de nuestro sketch, Blink.pde).
	\item Compila y comprime la biblioteca ``Core'' usando \programa{pic32-g++} (o \programa{pic32-gcc} según el caso) y \programa{pic32-ar}.
	\item Enlaza el programa usando un script de enlazado común a todos las diferentes placas chipKIT: ``chipKIT-application-COMMON.ld'', y uno específico para cada microcontrolador. En nuestro caso usa ``chipKIT-application-32MX795F12.ld''.
\end{enumerate}

\subsection{Parámetros de compilación}
\programa{MPIDE} utiliza diferentes parámetros a la hora de la compilación y posterior enlazado. Veamos cuales son:\\

\begin{itemize}
	\item \negrita{-ON}. Nivel de optimización. El valor ``N'' varía entre 0 y 3 para optimizaciones generales. Si su valor es ``s'' optimizamos para reducir el espacio que ocupa el programa.
	\item \negrita{-mno-smart-io}. Esta opción deshabilita el análisis estadístico de las cadenas con format que son pasadas a \programa{printf} o \programa{scanf} y a las variaciones `f' y `v' de estas funciones.
	\item \negrita{-w}. Esta opción deshabilitia todos los mensajes de advertencia.
	\item \negrita{-fno-exceptions}. Esta opción deshabilita el manejo de excepciones.
	\item \negrita{-ffunction-sections}. Con esta opción cada función se coloca en su propia sección en el archivo de salida.
	\item \negrita{-fdata-sections}.Con esta opción cada dato se coloca en su propia sección en el archivo de salida.
	\item \negrita{-g}. Produce información para la depuración
	\item \negrita{-mdebugger}. Esta opción define el símbolo ``\_DEBUGGER'', lo que reservara espacio en una sección del archivo de salida (.DBG\_EXCPT) para los vectores de excepción de depuración.
	\item \negrita{-Wcast-align}. Está opción hace que el compilador lance una advertencia cuando a un puntero se le ha hecho un cast. Por ejemplo, advierte si \programa{const char *} se convierte en \programa{int *}.
	\item \negrita{-fno-short-double}. Esta opción hace que el tipo de variable \programa{double} sea equivalente a \programa{long double} en vez de a \programa{float}, que es la opción por defecto.
	\item \negrita{-mprocessor=?}. Esta opción indica al compilador cual es el procesador objetivo. En nuestro caso particular se trata de ``32MX795F512L''.
	\item Definiciones de macros para el preprocesador.
	\begin{itemize}
		\item \negrita{-DF\_CPU=?}. Este macro define la frecuencia a la que funciona el procesador. En este caso ``80000000L''.
		\item \negrita{-DARDUINO}. Definimos el macro ``ARDUINO'' con valor 1.
		\item \negrita{-D\_BOARD\_MEGA\_}.Definimos el macro ``\_BOARD\_MEGA\_'' con valor 1. Indica que estamos usando una placa con el formato de un Arduino Mega.
		\item \negrita{-DMPIDEVER=?}. Definimos el valor de ``MPIDEVER'', que es ``0x01000305''.
		\item \negrita{-DMPIDE=?}. Definimos la versión de \programa{MPIDE} que hemos usado. En nuestro caso es la versión 23.
	\end{itemize}
	\item Parámetros de enlazado:
		\begin{itemize}
			\item \negrita{-Os}. Optimizamos para espacio.
			\item \negrita{-Wl,--gc-sections}. Usando ``-Wl'' pasamos opciones al enlazador, separadas por comas. En este caso pasamos la opción ``--gc-sections'' que sirve para eliminar todo el código y datos que vayan a ser usados por el programa final.
			\item \negrita{-mdebugger}. Misma opción que antes.
			\item \negrita{-mprocessor=?}. Misma opción que antes.
			\item \negrita{-lm}. Enlaza la librería ``math'' que nos proporciona funciones matemáticas\footnote{http://en.wikipedia.org/wiki/C\_mathematical\_functions}
			\item \negrita{-T}. Con esta opción podemos especificar un script de enlazado. Se usa dos veces: Una para el script ``chipKIT-application-COMMON.ld'' y otra para ``chipKIT-application-32MX795F512.ld''.
		\end{itemize}
\end{itemize}

Una vez el programa está compilado y enlazado\footnote{Suponemos que el nombre del archivo ejecutable es sketch.elf (como ejemplo)} es necesario generar el archivo ejecutable con extensión `.hex' que cargaremos en el microcontrolador. Esto lo consigue \programa{MPIDE} usando los programas \programa{pic32-objcopy} y \programa{pic32-bin2hex}.
\begin{lstlisting}[breaklines=true]
pic32-objcopy -O ihex -j .eeprom --set-section-flags=.eeprom=alloc,load --no-change-warnings --change-section-lma .eeprom=0 sketch.elf sketch.eep
pic32-bin2hex -a sketch.elf
\end{lstlisting}

El resultado de ejecutar estos dos comandos es el archivo ``sketch.hex'' que ya podrá ser cargado al microcontrolador.
Para este último paso será necesario usar el programa \programa{avrdude}, bien el que viene incluido con \programa{MPIDE} o el que tengamos instalado en nuestro sistema.
\begin{lstlisting}[breaklines=true]
avrdude -C[Directorio MPIDE]/hardware/tools/avrdude.conf -c stk500v2 -p pic32 -P /dev/ttyUSB0 -b 115200 -v -U
\end{lstlisting}\footnote{/dev/ttyUSB0 es el puerto serie en el que se encuentra el microcontrolador. Puede cambiar según el sistema o el número de dispositivos serie conectados.}

\newpage

\section{Alternativa a \programa{MPIDE}}

Una vez sabemos como funciona \programa{MPIDE} y el proceso que sigue para la compilación, enlazado y carga de nuestro programa, podemos proponer una alternativa más flexible.  
De esta forma es posible usar el editor de textos que queramos para generar nuestro código (por ejemplo \programa{VIM}), modificar las opciones de compilación, enlazado y carga de una forma rápida y sencilla\footnote{Podemos cambiar las opciones de compilación y enlazado en \programa{MPIDE}, pero requiere la modificación de un archivo de configuración del programa (platforms.txt) y reiniciar \programa{MPIDE} para que los cambios tengan efecto}.\\
Para esto vamos a crear una carpeta a modo de plantilla para los proyectos que hagamos. Dentro de esta carpeta tendremos diferentes subdirectorios que serviran para almacenar diferentes tipos de archivos que necesita nuestro programa.\\
La estructura de la plantilla es la siguiente:
\begin{lstlisting}
chipkit_template
	+- bin
	+- core
	    |
	    +- variants
	+- lib
	+- src
\end{lstlisting}

Función de cada carpeta:
\begin{itemize}
	\item \negrita{bin}. Aquí se guardará el programa compilado y enlazado. En esta carpeta estará el archivo final con extensión `.hex' que será el que carguemos en la placa.
	\item \negrita{core}. Aquí está la biblioteca Arduino (core) junto con los archivos de cabecera de las diferentes placas disponibles. Estos últimos archivos los encontramos en la carpeta ``variants'' y definen las constantes que dependen de la placa que estemos usando tales como, por ejemplo, la ubicación de los pines de entrada/salida.
	\item \negrita{lib}. Aquí se colocarán las bibliotecas de terceros de la misma forma que en MPIDE: copiando la carpeta de la biblioteca en este directorio. Usando el archivo Makefile compilaremos todas las bibliotecas y luego las enlazaremos con nuestro programa y la biblioteca core.
	\item \negrita{src}. Aquí estará el código fuente del programa que hayamos creado. Será el lugar en el que se encuentre el archivo de código fuente \programa{sketch.cpp} que contiene las definiciones de las funciones \programa{setup} y \programa{loop}. Además, si necesitamos crear nuestras propias bibliotecas (archivos .h + .c o .cpp) podemos crearlos tanto en en esta carpeta o en \programa{lib}.
\end{itemize}
Tanto en la carpeta principal como en \programa{core} tendremos un archivo Makefile para la automatización de los procesos de compilación, ensamblado y posterior carga de nuestro programa.\\
El Makefile principal se encarga de compilar, enlazar y comprimir la biblioteca core en el archivo \programa{core.a} usando el Makefile que se encuentra dentro de esa carpeta y luego compila todos los archivos en la carpeta \programa{lib} y la carpeta{src}. Una vez enlazado podemos usar la orden \programa{make load} para cargar el programa en la placa:
\newpage
\lstinputlisting[language=make, caption=Makefile Principal, label=code:makefile_principal, basicstyle=\tiny, breaklines=true]{codigo_fuente/Makefile_principal}
\newpage
Analicemos el funcionamiento de nuestro Makefile:\\
\lstinputlisting[language=make, firstline=0, lastline=9, breaklines=true]{codigo_fuente/Makefile_principal}
Empezamos declarando la localización de \programa{MPIDE} junto con la `toolchain' que viene incluida con este. La localización de \programa{MPIDE} en este caso es la carpeta \$HOME de mi PC (Es necesario modificar este valor dependiendo de la localización de \programa{MPIDE}). Las utilidades de la `toolchain' que viene incluida con \programa{MPIDE} se encuentran siempre en la misma subcarpeta, que en el Makefile es la variable `TOOLCHAIN\_PREFIX', independientemente del sistema operativo empleado.\\

\lstinputlisting[language=make, firstline=11, lastline=16, breaklines=true]{codigo_fuente/Makefile_principal}
Definimos el puerto serie en el que está conectado el microcontrolador. En Linux suele estar en `/dev/ttyUSBx' mientras que en Windows es `COMx' (sustituyendo `x' por el valor del puerto). También definimos la localización de \programa{avrdude} junto con su archivo de configuración y sus opciones, que usaremos para cargar el programa en la placa y que viene incluido con \programa{MPIDE}.\\

\lstinputlisting[language=make, firstline=17, lastline=20, breaklines=true]{codigo_fuente/Makefile_principal}
Definimos la CPU que lleva nuestra placa, que en este caso corresponde con el Modelo Microchip 32MX975F12L. También declaramos que placa es, para luego poder usar el archivo de cabecera correspondiente a dicha placa. Estas dos variables las exportamos para que sean accesibles también desde el Makefile que hay en la carpeta `core' y no sea necesario volverlas a definir ahí (hacemos lo mismo con la variable \$MPIDE).\\

\lstinputlisting[language=make, firstline=22, lastline=23, breaklines=true]{codigo_fuente/Makefile_principal}
El enlazador utiliza dos scripts para el correcto enlazado del programa: uno común a todas las placas chipKIT (chipKIT\-application\-COMMON.ld) y uno específico según el modelo que usemos.

\lstinputlisting[language=make, firstline=25, lastline=30, breaklines=true]{codigo_fuente/Makefile_principal}
Guardamos las opciones de compilación y enlazado en las variables \$CFLAGS y \$LDFLAGS respectivamente.

\lstinputlisting[language=make, firstline=34, lastline=34, breaklines=true]{codigo_fuente/Makefile_principal}
Añadimos al final de las opciones de compilación los directorios donde habrá que buscar los archivos de cabecera.

\lstinputlisting[language=make, firstline=37, lastline=39, breaklines=true]{codigo_fuente/Makefile_principal}
Buscamos en la carpeta `src' los ficheros con código fuente en C, C++ o ensamblador.

\lstinputlisting[language=make, firstline=41, lastline=43, breaklines=true]{codigo_fuente/Makefile_principal}
Generamos los nombres de los archivos compilados y ensamblados sustituyendo la extensión de los archivos de código fuente de la carpeta `src'.

\lstinputlisting[language=make, firstline=45, lastline=51, breaklines=true]{codigo_fuente/Makefile_principal}
Igual que antes, pero esta vez buscamos los archivos con código fuente dentro de la carpeta `lib' que contiene todas las librerías de terceros.

\lstinputlisting[language=make, firstline=53, lastline=53, breaklines=true]{codigo_fuente/Makefile_principal}
Este es el primer objetivo que hay en nuestro Makefile y es el que se ejecutará por defecto cuando ejecutamos \programa{make} sin ningún argumento. Lo utilizamos para definir el objetivo por defecto sin tener que modificar nada más que esta línea del Makefile. En este caso indicamos que este objetivo depende del objetivo `load' que veremos más adelante.

\lstinputlisting[language=make, firstline=55, lastline=56, breaklines=true]{codigo_fuente/Makefile_principal}
Llamamos al Makefile que se encuentra en la carpeta `core' para obtener la biblioteca estática `core/core.a' que luego enlazaremos con nuestro programa y las librerías de tercero que usemos.

\lstinputlisting[language=make, firstline=58, lastline=66, breaklines=true]{codigo_fuente/Makefile_principal}
Estas son las reglas que utilizamos para generar los ficheros con extensión `.o' a partir de su correspondiente archivo de código fuente.

\lstinputlisting[language=make, firstline=69, lastline=82, breaklines=true]{codigo_fuente/Makefile_principal}
Estas reglas las utilizaremos en el caso de querer generar archivos que hayan sido solo preprocesados o preprocesados y compilados sin ser ensamblados. Para usarlas habrá que indicar explícitamente que archivo queremos generar, por ejemplo \programa{make src/sketch.s} para obtener el archivo en ensamblador a partir de nuestro código C++ de `sketch.cpp'.

\lstinputlisting[language=make, firstline=84, lastline=85, breaklines=true]{codigo_fuente/Makefile_principal}
Con este objetivo enlazamos el programa usando los archivos compilados. En las dependencias se encuentran todos los archivos y reglas necesarias para llevar acabo el enlazado.

\lstinputlisting[language=make, firstline=86, lastline=89, breaklines=true]{codigo_fuente/Makefile_principal}
De la misma forma que lo hace \programa{MPIDE} creamos el archivo con extensión `.hex', que cargaremos mas tarde en el microcontrolador. Este objetivo depende del objetivo `link', ya que necesitamos el programa enlazado.

\lstinputlisting[language=make, firstline=91, lastline=92, breaklines=true]{codigo_fuente/Makefile_principal}
Este es el objetivo que hemos definido antes como el objetivo por defecto. Es el que se encarga de invocar a \programa{avrdude} que cargará nuestro programa en la placa.

\lstinputlisting[language=make, firstline=94, lastline=99, breaklines=true]{codigo_fuente/Makefile_principal}
Por último, el objetivo `clean' nos permitirá eliminar todos los archivos creados durante el proceso de compilación, enlazado y carga.

\section{Niveles de optimización}
Durante la compilación es posible especificar, mediante la opción `-O' diferentes niveles de optimización. Estos niveles agrupan una serie de optimizaciones.

\begin{table}[H]
\begin{center}
\begin{tabularx}{\textwidth}{|l|X|}
    \hline
    Nivel de optimización & Descripción\\
    \hline
    \hline
   	-O0 & Opción por defecto. Reduce el tiempo de compilación y facilita la depuración.\\
    \hline
    -O1 & Optimiza, pero aumenta el tiempo de compilación y la memoria empleada.\\
    \hline
    -O2 & Optimiza más todavía. GCC realiza casi todas la optimizaciones que no impliquen aumento de la velocidad a costa de un mayor tamaño del programa.\\
    \hline
    -O3 & Optimiza todavía más que -O2.\\
    \hline
    -Os & Optimiza el tamaño del programa. Habilita todas las optimizaciones de -O2 que normalmente no aumenten el tamaño del código y realiza más optimizaciones destinadas a reducir el tamaño del programa.\\
    \hline
  \end{tabularx}
\end{center}
\caption{Niveles de optimización}
\label{tab:optimizacionx}
\end{table}

Cada uno de estos niveles (a excepción de `Os') añade una serie de optimizaciones sobre el nivel inicial `O0'.\\
Veamos que optimizaciones lleva a cabo cada nivel. Esto podemos verlo usando el comando ``\programa{pic32-gcc -Ox -Q --help=optimizers | grep enabled}''.\\
\newline
Optimizaciones del nivel 0:
\begin{multicols}{2}
\begin{itemize}[noitemsep, topsep=0pt, parsep=0pt, partopsep=0pt]
	\item -falign-loops
	\item -fargumnt-alias
	\item -fbranch-count-reg
	\item -fcommon
	\item -fdata-sections
	\item -fdce
	\item -fdelete-null-pointer-checks
	\item -fdse
	\item -fearly-inlining
	\item -fgcse-lm
	\item -finline-functions-called-once
	\item -fivopts
	\item -fjump-tables
	\item -fmath-errno
	\item -fmove-loop-invariants
	\item -fpeephole
	\item -frename-registers
	\item -fsched-critical-path-heuristic
	\item -fsched-dep-count-heuristic
	\item -fsched-group-heurisitc
	\item -fsched-interblock
	\item -fsched-last-insn-heuristic
	\item -fsched-spec
	\item -fsched-spec-insn-heuristic
	\item -fsched-stalled-insns-dep
	\item -fsigned-zeros
	\item -fsplit-ivs-in-unroller
	\item -ftoplevel-reorder
	\item -ftrapping-math
	\item -ftree-cselim
	\item -ftree-forwprop
	\item -ftree-loop-im
	\item -ftree-loop-ivcanon
	\item -ftree-loop-optimize
	\item -ftree-phiprop
	\item -ftree-pta
	\item -ftree-reassoc
	\item -ftree-scev-cprop
	\item -ftree-slp-vectorize
	\item -ftree-vect-loop-version
	\item -funit-at-a-time
	\item -fvar-tracking
	\item -fvar-tracking-assignments
	\item -fweb
\end{itemize}
\end{multicols}

Optimizaciones que añade el nivel `O1' al nivel `O0':

\begin{multicols}{2}
\begin{itemize}[noitemsep, topsep=0pt, parsep=0pt, partopsep=0pt]
	\item -fcprop-registers
	\item -fdefer-pop
	\item -fdelayed-branch
	\item -fforward-propagate
	\item -fguess-branch-probability
	\item -fif-conversion
	\item -fif-conversion2
	\item -fipa-pure-const
	\item -fipa-reference
	\item -fmerge-constants
	\item -fomit-frame-pointer
	\item -fsplit-wide-types
	\item -ftree-ccp
	\item -ftree-ch
	\item -ftree-copy-prop
	\item -ftree-copyrename
	\item -ftree-dce
	\item -ftree-dominator-opts
	\item -ftree-dse
	\item -ftree-fre
	\item -ftree-sink
	\item -ftree-sra
	\item -ftree-ter
\end{itemize}
\end{multicols}

Optimizaciones que añade el nivel `O2' al nivel `O1':

\begin{multicols}{2}
\begin{itemize}[noitemsep, topsep=0pt, parsep=0pt, partopsep=0pt]
	\item -falign-functions
	\item -falign-jumps
	\item -falign-labels
	\item -fcaller-saves
	\item -fcrossjumping
	\item -fcse-follow-jumps
	\item -fexpensive-optimizations
	\item -fgcse
	\item -finline-small-functions
	\item -fipa-cp
	\item -fipa-sra
	\item -foptimize-register-move
	\item -foptimize-sibling-calls
	\item -fpeephole2
	\item -fregmove
	\item -fremove-local-statics
	\item -freorder-blocks
	\item -freorder-functions
	\item -frerun-cse-after-loop
	\item -fschedule-insns
	\item -fschedule-insns2
	\item -fstrict-aliasing
	\item -fthread-jumps
	\item -ftree-builtin-call-dce
	\item -ftree-pre
	\item -ftree-switch-conversion
	\item -ftree-vrp
\end{itemize}
\end{multicols}

Optimizaciones que añade el nivel `O3' al nivel `O2':

\begin{multicols}{2}
\begin{itemize}[noitemsep, topsep=0pt, parsep=0pt, partopsep=0pt]
	\item -fgcse-after-reload
	\item -finline-functions
	\item -fipa-cp-clone
	\item -fpredictive-commoning
	\item -ftree-pre-partial-partial
	\item -ftree-vectorize
	\item -funswitch-loops
\end{itemize}
\end{multicols}

El nivel `Os' realiza las mismas optimizaciones que `O2' menos las siguientes:
\begin{multicols}{2}
\begin{itemize}[noitemsep, topsep=0pt, parsep=0pt, partopsep=0pt]
	\item -falign-functions
	\item -falign-jumps
	\item -falign-loops
	\item -falign-labels
	\item -freorder-blocks
	\item -freorder-blocks-and-partition
	\item -fprefetch-loop-arrays
	\item -ftree-vect-loop-version
\end{itemize}
\end{multicols}

Las optimizaciones que comienzan con `fsched' se usan para reordenar las instrucciones. También hay varias optimizaciones que afectan a objetos `tree' que, de acuerdo a la documentación de GCC\footnote{http://gcc.gnu.org/onlinedocs/gccint/Tree-overview.html} son estructuras centrales de datos usadas para la representación interna. Un `tree' es un tipo de puntero que puede apuntar a dataos de diferente tipo.\\
Las optimizaciones `finline-functions-called-once', `finline-functions' y `finline-small-functions' sustituyen la llamada a la función por el código de la misma. La primera de ellas está habilitada en todos los niveles de optimización y sustituye, como su nombre indica, todas las funciones que sean llamadas una sola vez. `finline-small-functions' realiza la sustitución solo si su tamaño es menor quelas instrucciones necesarias para llamar a la función, en ensamblador. Estas dos optimizaciones hacen posible escribir pequeñas funciones auxiliares que puedan facilitar tanto la escritura como la lectura de nuestro programa sin suponer una disminución del rendimiento. La última de estas optimizaciones, `finline-functions', considera todas las funciones para sustituir su llamada por el código.\\
En el caso de las optimizaciones que comienzan por `falign' \programa{gcc} se encarga de alinear los elementos especificados en cada una de estas optimizaciones para reducir los accesos a memoria a cambio de ocupar más espacio. Es por esto que en el nivel `Os' se desactivan estas, entre otras.

\newpage
\section{Aplicación para ChipKIT}

En esta sección vamos a ver el programa que emplearemos para observar y analizar los diferentes niveles de optimización que lleva acabo \programa{gcc}. Se trata de un terminal remoto al que nos conectamos a través del puerto 23 con el protocolo TCP. Una vez conectados disponemos de varios comandos para ordenar un array que pasaremos como parámetro. Los comandos disponibles son los siguientes:

\begin{itemize}
	\item bubble0, bubble1, bubble2, bubble3. Ordenan los números usando el algoritmo `Bubble Sort'.
	\item quick0, quick1, quick2, quick3. Ordenan los números usando el algoritmo `Quick Sort'.
	\item sel0, sel1, sel2, sel3. Ordenan los números usando el algoritmo `Selection Sort'.
	\item help. Muestra los comandos disponibles.
	\item exit. Cierra la conexión.
\end{itemize}

El número que acompaña a cada uno de los comandos es el nivel de optimización con el que han sido compilados. El resultado que devuelve cada una de las órdenes es la lista ordenada y el tiempo de ejecución.\\
Para la realización del programa se ha usado el ejemplo del servidor Telnet que viene incluido con la biblioteca `chipKITEthernet'. Veamos la estructura y funcionamiento de nuestro `sketch'\footnote{Archivo donde definimos las funciones `setup' y `loop'}.\\
\lstinputlisting[firstline=1, lastline=2, breaklines=true]{codigo_fuente/sketch.cpp}
Comenzamos incluyendo la biblioteca del shield Ethernet y el archivo de cabecera `funciones.h' que contiene los diferentes algoritmos de ordenación que usaremos en los comandos expuestos anteriormente.\\

\lstinputlisting[firstline=4, lastline=5]{codigo_fuente/sketch.cpp}
Definimos el tamaño máximo del mensaje, esto es, la máxima longitud de la línea que podemos envíar a nuestro programa, esto incluye el comando seguido por el array. También definimos el tamaño máximo del array que acepta el programa, siendo este la mitad que el mensaje. Esta longitud debería ser suficiente ya que, con números de un solo dígito (o caracter en este caso), no es posible superar este tamaño ya que por cada uno de estos numeros deberemos añadir un separador, lo que implica un mínimo de dos caracteres por dígito, sin contar el comando.

\lstinputlisting[firstline=6, lastline=9]{codigo_fuente/sketch.cpp}
Esta parte está inalterada con respecto al ejemplo que viene con la biblioteca Ethernet. Aquí declaramos la dirección física que queramos que tenga nuestro dispositivo. Si la inicializamos a 0 la dirección física que se usará será la del propio dispositivo. En el caso de elegir un valor diferente será este el que se use\footnote{En algunas aplicaciones puede ser interesante modificar este valor para enmascarar la identidad del dispositivo o para suplantar la de otro, pero esto se sale del objetivo de este proyecto.}.


\chapter{Entorno de desarrollo}
\label{cap3}
Cuando hablamos de entorno de desarrollo normalmente nos referimos al conjunto de procedimientos y herramientas para desarrollar, probar y depurar aplicaciones o programas.

En un entorno de desarrollo es común encontrar ciertas herramientas básicas: un editor de texto más o menos avanzado, una suite de compilación, un entorno de depuración y, para el caso de los microcontroladores, un programador.

Podemos diferenciar dos tipos de entorno de desarrollo: los entornos clásicos y los entornos de desarrollo integrados. En la forma clásica de entorno de desarrollo se utilizan de forma explicita herramientas independientes. Por ejemplo, si queremos desarrollar un programa escrito en C, podemos escribirlo usando \programa{Vim} o \programa{Emacs}, compilarlo usando \programa{gcc} y por último, depurarlo con herramientas como \programa{gdb} o \programa{valgrind}.

Por otro lado, en los entornos de desarrollo integrados o IDE (Integrated Development Environment)  aúnan todas las herramientas en un mismo interfaz de usuario, del que no es necesario salir en todo el proceso de desarrollo del programa. En algunos casos las herramientas son las mismas que en los entornos clásicos pero el IDE las oculta. Hay multitud de ejemplos de IDEs, como \programa{Eclipse}~\footnote{\url{http://www.eclipse.org/}} o \programa{Netbeans}~\footnote{\url{https://netbeans.org/}}, por citar algunos.

\section{MPLAB X}
\programa{MPLAB X}~\cite{website:mplab} es la última versión del entorno de desarrollo integrado que proporciona Microchip para programar sus microcontroladores. Este IDE, basado en el entorno de código abierto Netbeans, nos permite escribir nuestro programa, compilarlo, programar el microcontrolador y, con la ayuda de un depurador como los mencionados en el capítulo anterior, depurar el programa.

Aunque un IDE es mucho más que un editor de texto, el entorno de edición suele ser la herramienta más utilizada. El editor que incorpora \programa{MPLAB X} realiza el sangrado de las líneas de código automáticamente, identifica las parejas de corchetes, comillas, etc.\@ y enfatiza el texto con colores de forma tanto sintáctica como semántica. También dispone de plantillas de código y consejos.

\programa{MPLAB X} ofrece también una fácil gestión de proyectos y de archivos dentro de un mismo proyecto. En la parte izquierda del IDE podemos encontrar una sección en la que se encuentran los proyectos y sus archivos, ordenados de forma jerárquica, lo que ayuda en la gestión de diferentes proyectos en un mismo entorno.

\figura{mplab_descr.png}{scale=0.30}{MPLABX Depurando el programa Blink}{mplab}{H}

Como se aprecia en la Figura~\ref{mplab}, \programa{MPLAB X} nos ofrece mucha información. En esta figura vemos a \programa{MPLAB X} en mitad de la depuración del programa Blink~\footnote{Este programa es el equivalente en microcontroladores al `Hola Mundo', su única función es hacer parpadear un LED.}. En la zona superior del programa se encuentra el menú principal, que contiene los botones para crear un nuevo proyecto, abrirlo o guardarlo. También incluye todos los botones dedicados a la depuración: empezar o parar la depuración, ir a la siguiente instrucción, etc. En la parte izquierda se puede apreciar dos secciones diferentes. La superior muestra la organización del proyecto y la lista de proyectos en caso de que hubiese más de uno. Debajo de esto se encuentra toda la información acerca del estado actual del dispositivo seleccionado durante la creación del proyecto, en este caso un PIC32MX795F512L, como por ejemplo la cantidad de memoria RAM y flash que requiere la aplicación para funcionar comparado con la cantidad total de la que dispone el microcontrolador. En la parte central, y ocupando la mayor parte de la ventana, observamos el editor con el que se puede modificar los archivos de código fuente del proyecto. Por último, en la zona inferior, se encuentra el área de mensajes, consola e información del estado del proyecto. En este caso en particular está mostrando el estado del depurador. En esta zona también se puede mostrar el valor de variables del proyecto en tiempo real y el estado de los registros del procesador, entre otros.

El IDE dispone de muchas más herramientas para la depuración de programas. Durante la ejecución del mismo en modo depuración se pueden establecer puntos en el el código fuente donde el programa debe detener su ejecución, conocidos por su nombre en inglés: breakpoints. Se utilizan para detener la ejecución y así poder ver o modificar el estado de la memoria y registros del procesador en ese momento, ejecutar el programa línea a línea del código fuente o instrucción a instrucción en el código ensamblador, comprobar la traducción a ensamblador hecha por el compilador, etc.

Con este entorno de desarrollo veremos más adelante los diferentes efectos que los distintos niveles de optimización tienen en nuestro programa y podremos usarla para hacer profiling.

\section{Arduino IDE}
% Wiring y processing
% gcc
% bibliotecas incluidas
% Otros programas
% Multiplataforma
Para realizar proyectos en la plataforma Arduino es necesario disponer del entorno de desarrollo `Arduino IDE' que es posible encontrar en la página web de Arduino. Este programa está basado en Wiring\footnote{\url{http://wiring.org.co/}}, que a su vez nació a partir de Processing\footnote{\url{http://processing.org/}} por lo que el aspecto del mismo es prácticamente idéntico en todos ellos.

Processing se creó como una plataforma para facilitar la creación de programas sencillos, que muestren resultados de forma visual en un monitor. Usando esta plataforma como base, nació Wiring, que tenía un objetivo similar pero centrado en los microcontroladores.~\footnote{En el blog de Marlon J. Manrique podemos encontrar un buen artículo sobre Wiring: \url{http://www.marlonj.com/blog/2011/06/ques-es-wiring/}.}

Las diferencias entre \programa{Wiring} y el IDE de Arduino radican en el funcionamiento interno. Arduino IDE tiene incluido todo lo necesario para poder escribir, compilar, ensamblar, enlazar y cargar los programas a la placa Arduino, por lo que el usuario solo debe abrir el IDE, escribir su programa y darle al botón de ejecución; el IDE se encargará de compilar, enlazar y cargar el programa en la placa microcontroladora. Este proceso lo he descrito con más detalle en su propio apartado.

Si accedemos a la carpeta de instalación de Arduino veremos diferentes subcarpetas, donde se guardan diferentes componentes necesarios para su funcionamiento.
\begin{figure}[H]
\begin{center}
\tikzstyle{every node}=[draw=black,thick,anchor=west]
\tikzstyle{selected}=[draw=red,fill=red!30]
\tikzstyle{optional}=[dashed,fill=gray!50]
\begin{tikzpicture}[%
  grow via three points={one child at (0.5,-0.7) and
  two children at (0.5,-0.7) and (0.5,-1.4)},
  edge from parent path={(\tikzparentnode.south) |- (\tikzchildnode.west)}]
	\node{Arduino}
		child { node {examples}}
		child { node {hardware}
			child { node {arduino}
				child { node {bootloaders}}
				child { node {cores}}
				child { node {firmwares}}
				child { node {variants}}
			}
			child [missing] {}
			child [missing] {}
			child [missing] {}
			child [missing] {}
			child { node {tools}
				child { node {avr}}
			}
			child [missing] {}
		}
		child [missing] {}
		child [missing] {}
		child [missing] {}
		child [missing] {}
		child [missing] {}
		child [missing] {}
		child [missing] {}
	child { node {lib}}
	child { node {libraries}}
	child { node { reference}};
\end{tikzpicture}
\end{center}
\caption{Estructura de directorios del entorno de programación de Arduino}
\label{arduino_carpetas}
\end{figure}

\begin{description}
	\item[\orden{examples}] El programa incluye algunos ejemplos que se pueden compilar y cargar sin modificarlos.
	\item[\orden{hardware}] En esta carpeta se guarda todo lo relacionado con el hardware, como su nombre indica. Aquí encontraremos la `toolchain' y las bibliotecas necesarias para la compilación y ensamblado de los programas Arduino.
	\item[\orden{lib}] Contiene bibliotecas Java para el funcionamiento del entorno de desarrollo.
		\item[\orden{libraries}] El IDE incluye algunas bibliotecas de terceros que se pueden incluir en nuestros proyectos.
		\item[\orden{reference}] Contiene toda la documentación.
\end{description}

En la carpeta \orden{hardware} encontramos otras dos carpetas: \orden{arduino} y \orden{tools}.  La primera contiene bootloaders, firmwares, la biblioteca `core' y en el directorio \orden{variants} están los archivos de cabecera de cada placa microcontroladora en los que se especifica la posición de cada periférico y pin según el microcontrolador.  En \orden{tools} está avrdude (utilidad para cargar los programas en la placa) y su archivo de configuración junto con la `toolchain' \programa{GCC} para realizar la compilación cruzada de los programas.

\section{\programa{MPIDE}}
% Igual que arduino
% Gcc pic32
\programa{MPIDE}\cite{website:mpide} es el IDE creado por Digilent para el entorno ChipKIT. Su diseño es idéntico al IDE de Arduino, variando solamente el color y el compilador usado.

\figura{mpide.jpg}{scale=0.6}{\programa{MPIDE}}{mpide}{H}

Si accedemos a la carpeta \orden{hardware} de este programa, encontramos otra carpeta además de \orden{arduino} y \orden{tools}: \orden{pic32}. La estructura es similar  a la de la carpeta de Arduino. Aquí están todas las bibliotecas para la plataforma ChipKIT. También podemos encontrar aquí la `toolchain' \programa{GCC} para pic32 en el directorio \orden{compiler}.

\begin{figure}
\begin{center}
\tikzstyle{every node}=[draw=black,thick,anchor=west]
\tikzstyle{selected}=[draw=red,fill=red!30]
\tikzstyle{optional}=[dashed,fill=gray!50]
\begin{tikzpicture}[%
grow via three points={one child at (0.5,-0.7) and
two children at (0.5,-0.7) and (0.5,-1.4)},
edge from parent path={(\tikzparentnode.south) |- (\tikzchildnode.west)}]
\node{hardware}
	child { node {arduino}
		child { node {bootloaders}}
		child { node {cores}}
		child { node {firmwares}}
		child { node {variants}}
	}
	child [missing] {}
	child [missing] {}
	child [missing] {}
	child [missing] {}
	child { node {pic32}
		child { node {compiler}}
		child { node {cores}}
		child { node {libraries}}
		child { node {variants}}
	}
	child [missing] {}
	child [missing] {}
	child [missing] {}
	child [missing] {}
	child { node {tools}
		child { node {avr}}
	}
	child [missing] {};
\end{tikzpicture}
\end{center}
\caption{Estructura de directorios de hardware en la carpeta de instalación de MPIDE}
\label{mpide_carpetas}
\end{figure}

Estos entornos de desarrollo, tanto el de Arduino como el de ChipKIT, son fáciles de utilizar por su interfaz simple, pero carecen de muchas características de otros entornos como podrían ser Eclipse o incluso Vim con plugins. El editor no dispone de autocompletado ni de atajos u órdenes especiales para moverse por el archivo. Para configurar alguna opción, como la fuente empleada o su tamaño hay que recurrir a editar el archivo de configuración directamente.

Al igual que el entorno de Arduino, no es posible depurar programas.  No se puede hacer una ejecución paso a paso, ver el código ensamblador producido o ver el contenido de la memoria y su estado.  Si queremos depurar el programa de forma rudimentaria es necesario, por ejemplo, enviar por el puerto serie los valores de las variables que queramos controlar o mensajes cuando algo no funciona como debe. Tampoco muestra por defecto el proceso que sigue para compilar el programa.

En definitiva, ambos programas están pensados para proporcinar un entorno sencillo y accesible para todo tipo de usuarios.

\section{Estructura de un programa Arduino}
En cualquier programa para ChipKIT o Arduino podemos encontrar tres grupos principales de código fuente.
\subsection{Core}
Para empezar, tenemos los archivos que formarán parte de la biblioteca ``Core'' a la hora de la compilación. Estos archivos vienen incluidos con \programa{MPIDE} y los podemos encontrar en el directorio en el que tengamos instalado \programa{MPIDE}~\protect\footnote{La carpeta de \programa{MPIDE} la podemos colocar donde queramos, pues no requiere instalación}: en la subcarpeta\\
\orden{hardware/pic32/cores/pic32}

\begin{table}[H]
\begin{center}
\begin{tabular}{ll}
  \textbf{Sistema Operativo} & \textbf{Ruta}\\
	\midrule
  \textit{Linux} & /home/lluis/mpide/hardware/pic32/cores/pic32\\
  \textit{Mac OSX} & /Applications/Mpide.app/Contents/Resources/Java/hardware/pic32/cores/pic32\\
  \textit{Windows} & C:\textbackslash{}Archivos de Programa\textbackslash{}\programa{MPIDE}\textbackslash{}hardware\textbackslash{}pic32\textbackslash{}cores\textbackslash{}pic32\\
\end{tabular}
\end{center}
\caption{Ruta de la biblioteca Core}
\label{tab:core_path}
\end{table}

En esta biblioteca tenemos ya definidas las funciones que necesitamos para interactuar con los dispositivos que están integrados en la placa (en nuestro caso, ChipKIT Max32). Para más información acerca de estas funciones y cómo utilizarlas podemos acudir a la página de referencia de Arduino~\cite{website:arduino_funcs}. Además, en esta biblioteca está ya definida la función main de nuestro programa y tiene el siguiente aspecto:

\lstset{language=C++}
\begin{lstlisting}[caption=main.cpp]
#define OPT_SYSTEM_INTERNAL
#include <System_Defs.h>

#if (ARDUINO >= 100)
#include <Arduino.h>
#else
#include <WProgram.h>
#endif

int main(void) {
init();
setup();

while(1) {
	_scheduleTask();
	loop();
}
return 0;
}
\end{lstlisting}

La función \orden{\_scheduleTask()} forma parte del módulo de administración de tareas de la biblioteca de Arduino y se encarga de ejecutar las tareas periódicas del procesador. Podemos encontrar su definición y una breve descripción de su funcionamiento en el archivo `task\_manager.c' junto con el resto de funciones de este módulo.

\subsection{Sketch}
Los programas que escribimos usando \programa{MPIDE} (o el IDE de Arduino) se llaman sketches. En realidad, no escribimos un programa completo porque, como acabamos de ver, el programa en sí ya está escrito en la biblioteca Core. Nuestro objetivo es implementar dos funciones: \orden{setup()} y \orden{loop()}, que ya están declaradas en el archivo de cabecera wiring.h.

\begin{lstlisting}[caption=Sección de wiring.h]
void setup(void);
void loop(void);
\end{lstlisting}

Como se puede apreciar en la declaración de ambas funciones, estas no esperan ni devuelven ningún valor.

\lstinputlisting[language=C++, caption=Sketch básico ("BareMinimum.pde")]{codigo_fuente/BareMinimum.pde}
En \orden{setup()} colocaremos el código para la inicialización de nuestro programa. Es el lugar donde podemos configurar el puerto serie, por ejemplo, o definir cómo se comportarán los pines (entrada o salida), entre otras cosas. En definitiva, es el lugar donde debemos colocar el código que queramos que se ejecute una vez al inicio del programa y que no vuelva a ejecutarse más.

En \orden{loop()} pondremos el código que se ejecutará de forma cíclica indefinidamente ya que, como podemos observar en \orden{main.cpp}, la función \orden{loop()} se encuentra dentro de un bucle infinito.

Veamos como ejemplo el sketch Blink:
\lstinputlisting[language=C++, caption=Blink.pde, label=code:blink]{codigo_fuente/Blink.pde}

En este sketch utilizamos las funciones de Arduino \orden{pinMode}, \orden{digitalWrite} y \orden{delay}. En la función \orden{setup()} inicializamos el puerto al que está conectado el LED como una salida~\protect\footnote{PIN\_LED1 está definido en el archivo de cabecera Board\_Defs.h que depende del modelo de microcontrolador que estemos usando. Generalmente, este es el LED que se encuentra en la propia placa, conectado al pin 13.}, mientras que en \orden{loop()} encendemos y apagamos el LED con retardos de 1 segundo.

\subsection{Bibliotecas de terceros}
Aparte del código de Arduino y del que creemos nosotros, también es posible añadir bibliotecas de terceros. Gracias a la amplia comunidad que hay detrás de Arduino podemos encontrar cientos de ellas. A la hora de usar una biblioteca de Arduino en ChipKIT tendremos que asegurarnos de que no haga uso de funciones específicas de los procesadores Atmel que encontramos en las placas Arduino, ya que no funcionarán en PIC32.

Cuando trabajamos con \programa{MPIDE} podemos encontrar las bibliotecas en dos sitios diferentes. \programa{MPIDE} tiene incluidas varias bibliotecas que podemos encontrar en el directorio ``libraries'' dentro de la carpeta en la que esté instalado \programa{MPIDE}. Aparte de estas bibliotecas, podemos instalar bibliotecas que descarguemos de Internet o que creemos nosotros dentro de la carpeta ``libraries'' en el directorio \orden{sketchbook}~\protect\footnote{La carpeta ``libraries'' no existe. Es necesario crearla dentro del directorio \orden{sketchbook}.} que \programa{MPIDE} habrá creado la primera vez que lo ejecutemos.

Con las bibliotecas instaladas en uno de los dos directorios podremos añadirlas a nuestro programa de dos formas diferentes. Podemos añadirlas usando el menú de MPIDE, pudiéndolas encontrar en \orden{Sketch > Import Library}. También es posible usar la directiva \orden{\#include <[Archivo De Cabezera].h>} como haríamos en cualquier programa escrito en C/C++. \programa{MPIDE} se encargará de compilar las bibliotecas y enlazarlas con nuestro programa.

\section{Proceso de compilación de \programa{MPIDE}}
\label{sec:compilacion}
En este apartado he tratado de describir el proceso que sigue \programa{MPIDE} para la compilación y posterior carga en el microcontrolador del programa creado en \programa{MPIDE}\@. Nuestro objetivo será saber que programas y opciones de compilación usa el IDE de ChipKIT\@.

\tikzstyle{format} = [draw, thin, fill=blue!20]
\tikzstyle{medium} = [ellipse, draw, thin, fill=green!20, minimum height=2.5em]
\begin{figure}[htb]
  \begin{tikzpicture}[node distance=3cm, auto,>=latex', thick]
    % We need to set at bounding box first. Otherwise the diagram
    % will change position for each frame.
    %\path[use as bounding box] (-1,0) rectangle (10,-2);
    \path[->]node[](start){};
    \path[->]node[format,below of=start] (preproc) {Preprocesador}
    (start) edge node {.c .cpp} (preproc);
    \path[->]node[format, right of=preproc] (comp) {Compilador}
    (preproc) edge node {.i} (comp);
    \path[->]node[format, right of=comp] (ensam) {Ensamblador}
    (comp) edge node {.S} (ensam);
    \path[->]node[format, right of=ensam] (enlaz) {Enlazador}
    (ensam) edge node {.o} (enlaz);
    \path[->]node[format, right of=enlaz] (carga) {Carga en Placa}
    (enlaz) edge node {.elf} (carga);
    %\path[->, draw](preproc) -- +(0,1) -| node[near start] {ensam\TeX} (ensam);
  \end{tikzpicture}
  \caption{Proceso de compilación}
\end{figure}
También es interesante conocer que archivos genera \programa{MPIDE} a lo largo de la compilación, pero estos archivos son borrados una vez termina la compilación del programa por lo que tendremos que cambiar este comportamiento. Para ello acudiremos al archivo de configuración \orden{preferences.txt} cuya localización varía según el sistema operativo empleado.

\begin{table}[H]
\begin{center}
\begin{tabular}{ll}
  \textbf{Sistema Operativo} & \textbf{Ruta}\\
    \midrule
    \textit{Linux} & /home/[nombre de usuario]/.mpide/preferences.txt\\
    \textit{Mac OSX} & /Users/[nombre de usuario]/Library/mpide/preferences.txt\\
    \textit{Windows} & C:\textbackslash Archivos de Programa\textbackslash \programa{MPIDE}\textbackslash hardware\textbackslash pic32\textbackslash cores\textbackslash pic32\\
  \end{tabular}
\end{center}
\caption{Ruta del archivo \orden{preferences.txt}}
\label{tab:preferences_path}
\end{table}

El archivo \orden{preferences.txt} contiene toda la configuración de MPIDE a excepción de la configuración del compilador, pero nosotros estamos interesados en la línea 75:
\begin{lstlisting}
preproc.save_build_files=false
\end{lstlisting}

Cambiando el valor a \orden{true}, \programa{MPIDE} guardará los archivos temporales de la compilación.

Para ver cómo funciona el proceso de compilación en \programa{MPIDE} usaremos el sketch Blink (\ref{code:blink}).

Para poder observar las acciones que lleva a cabo \programa{MPIDE} durante la compilación del programa podemos mantener apretada la tecla \orden{Mayus} mientras pulsamos el botón de compilar. De esta forma \programa{MPIDE} nos mostrará en el terminal todas las órdenes que ejecute.

\figura{mpide_compilado.jpg}{scale=0.6}{MPIDE después de compilar}{mpide_compilado}{H}

El proceso completo de compilación (lo que se muestra en el terminal) está en el Apéndice \ref{chap:apendice1}.

\programa{MPIDE} comienza copiando nuestro sketch y las bibliotecas que hayamos incluido a una carpeta temporal en la que se realizará el proceso de compilación y ensamblado. Antes de comenzar la compilación, se cambia la extensión del sketch de `.pde'~\footnote{En las últimas versiones de Arduino sería `.ino'.} a `.cpp'. Este archivo se escanea en busca de funciones y las que se encuentren (en \programa{MPIDE} se pueden definir funciones sin declararlas) son declaradas al inicio del archivo de forma automática, incluyendo \orden{setup()} y \orden{loop()} a pesar de estar ya declaradas en el archivo \orden{main.cpp}, junto con \verb|#include<WProgram.h>|. Esta declaración automática de funciones puede ocasionar errores durante la compilación si usamos una función que reciba un parámetro cuyo tipo sea uno creado con una declaración \verb|typedef| al comienzo del archivo, ya que esta función se declarará antes que el parámetro y ocasionará un error de compilación.

Utilizando \orden{pic32-g++} se compila el sketch (con extensión `.cpp') a un archivo objeto `.o'. A continuación compila y, con \orden{pic32-ar}, empaqueta en el archivo \orden{core.a} la biblioteca `Core', es decir, los archivos que se encuentran en la subcarpeta \verb|hardware/pic32/cores| en el direcotorio de instalación de \programa{MPIDE}; de esta forma se facilita su manejo más adelante al ser muchos archivos. Si se han utilizado bibliotecas de terceros se compilan. A continuación se enlazan todos los programas objeto produciendo un archivo ejecutable con extensión `.elf'. Para realizar el enlazado se pasa a la orden, como parámetro, dos scripts de enlazado: uno, común a todas las placas de la misma familia (\orden{ChipKIT-application-common}); y otro, que depende de la placa microcontroladora a la que vaya destinado; en el caso de ChipKIT Max32 es \orden{ChipKIT-application-32MX795F512.ld}.


\programa{MPIDE} utiliza diferentes parámetros a la hora de la compilación y posterior enlazado, comunes a todos los archivos de código fuente, ya sean de la biblioteca `Core' o el sketch. Veamos cuáles son:

\begin{itemize}
    \item \negrita{-ON}. Nivel de optimización. El valor ``N'' varía entre 0 y 3 para optimizaciones generales. Si su valor es ``s'' optimizamos para reducir el espacio que ocupa el programa. Por defecto, el nivel de optimización que utiliza \programa{MPIDE} es 2.
    \item \negrita{-mno-smart-io}. Esta opción deshabilita el análisis estadístico de las cadenas con formato, que son pasadas a \orden{printf} o \orden{scanf} y a las variaciones `f' y `v' de estas funciones.
    \item \negrita{-w}. Esta opción deshabilita todos los mensajes de advertencia.
    \item \negrita{-fno-exceptions}. Esta opción deshabilita el manejo de excepciones.
    \item \negrita{-ffunction-sections}. Con esta opción cada función se coloca en su propia sección en el archivo de salida.
    \item \negrita{-fdata-sections}.Con esta opción cada dato se coloca en su propia sección en el archivo de salida.
    \item \negrita{-g}. Produce información para la depuración.
    \item \negrita{-mdebugger}. Esta opción define el símbolo `\_DEBUGGER', lo que reservará espacio en una sección del archivo de salida (.DBG\_EXCPT) para los vectores de excepción de depuración.
    \item \negrita{-Wcast-align}. Está opción hace que el compilador lance una advertencia cuando a un puntero se le ha hecho un cast, es decir, si cambia el tipo de dato al que estaba apuntando a uno que requiera un tamaño de alineamiento mayor. Por ejemplo, advierte si \orden{char *} se convierte en \orden{int *} en máquinas donde solo se puede acceder a datos de tipo \orden{int} cada dos o cuatro palabras.
    \item \negrita{-fno-short-double}. Esta opción hace que el tipo de variable \orden{double} sea equivalente a \orden{long double} en vez de a \orden{float}, que es la opción por defecto.
    \item \negrita{-mprocessor=?}. Esta opción indica al compilador cuál es el procesador objetivo. En nuestro caso particular se trata de ``32MX795F512L''.
    \item Definiciones de macros para el preprocesador.
    \begin{itemize}
        \item \negrita{-DF\_CPU=?}. Este macro define la frecuencia a la que funciona el procesador. En este caso ``80000000L''.
        \item \negrita{-DARDUINO}. Definimos el macro ``ARDUINO'' con valor 1.
        \item \negrita{-D\_BOARD\_MEGA\_}.Definimos el macro ``\_BOARD\_MEGA\_'' con valor 1. Indica que estamos usando una placa con el formato de un Arduino Mega.
        \item \negrita{-DMPIDEVER=?}. Definimos el valor de ``MPIDEVER'', que es ``0x01000305''.
        \item \negrita{-DMPIDE=?}. Definimos la versión de \programa{MPIDE} que hemos usado. En nuestro caso es la versión 23.
    \end{itemize}
    \item Parámetros de enlazado:
        \begin{itemize}
            \item \negrita{-Os}. Optimizamos para espacio.
            \item \negrita{-Wl,--gc-sections}. Usando ``-Wl'' pasamos opciones al enlazador, separadas por comas. En este caso pasamos la opción ``--gc-sections'' que sirve para eliminar todo el código y datos que no vayan a ser usados por el programa final y que, por lo tanto, no es necesario que se encuentren en él. De esta forma se reduce el tamaño del programa final.
            \item \negrita{-mdebugger}. Misma opción que antes.
            \item \negrita{-mprocessor=?}. Misma opción que antes.
            \item \negrita{-lm}. Enlaza la biblioteca estándar \programa{math} que nos proporciona funciones matemáticas\protect\footnote{http://en.wikipedia.org/wiki/C\_mathematical\_functions}
            \item \negrita{-T}. Con esta opción podemos especificar un script de enlazado. Se usa dos veces: una, para el script \orden{ChipKIT-application-COMMON.ld};\\  y otra para \orden{ChipKIT-application-32MX795F512.ld}.
        \end{itemize}
\end{itemize}

Una vez el programa está compilado y enlazado~\protect\footnote{Suponemos que el nombre del archivo ejecutable es sketch.elf (como ejemplo).} es necesario generar el archivo ejecutable con extensión `.hex' que cargaremos en el microcontrolador. Esto lo consigue \programa{MPIDE} usando los programas \programa{pic32-objcopy} y \programa{pic32-bin2hex}.
\begin{lstlisting}[breaklines=true]
pic32-objcopy -O ihex -j .eeprom --set-section-flags=.eeprom=alloc,load --no-change-warnings --change-section-lma .eeprom=0 sketch.elf sketch.eep
pic32-bin2hex -a sketch.elf
\end{lstlisting}

\begin{itemize}
  \item \negrita{-O ihex} Crea un archivo con formato hexadecimal.
  \item \negrita{-j .eeprom} Solo copia la sección especificada en la salida, en este caso, la sección .eeprom.
  \item \negrita{--set-section-flags=.eeprom=alloc,load} Establece las propiedades de una sección.
  \item \negrita{--no-change-warnings} Avisa si la sección no existe.
  \item \negrita{--change-section-lma .eeprom=0} Cambia el LMA (Load Memory Address) de la sección.
\end{itemize}

El resultado de ejecutar estas dos órdenes es el archivo \orden{sketch.hex}, que ya podrá ser cargado al microcontrolador.
Para este último paso será necesario usar el programa \programa{avrdude}, bien el que viene incluido con \programa{MPIDE} o el que tengamos instalado en nuestro sistema.
\begin{lstlisting}[breaklines=true]
avrdude -C[Directorio MPIDE]/hardware/tools/avrdude.conf -c stk500v2 -p pic32 -P /dev/ttyUSB0 -b 115200 -v
\end{lstlisting}

\begin{itemize}
  \item \negrita{-C[Directorio MPIDE]/hardware/tools/avrdude.conf} Carga la configuración personalizada de \programa{avrdude}.
  \item \negrita{-c stk500v2} Indica el modelo de progamador.
  \item \negrita{-p pic32} Especifica el modelo de microcontrolador.
  \item \negrita{-P /dev/ttyUSB0} El puerto serie en el que se encuentra conectado el microcontrolador.
  \item \negrita{-b 115200} La velocidad de transmisión en bps del puerto serie.
  \item \negrita{-v} Muestra más información de lo que está sucediendo.
\end{itemize}

\section{Alternativa a \programa{MPIDE}}

Ya hemos visto cómo el IDE genera y carga el sketch. El problema de este método es que no es posible modificar el proceso de una manera sencilla para adaptarlo a un programa en particular. Es posible ejecutar cada uno de las órdenes en un terminal y obtener el mismo resultado que con \programa{MPIDE} pero, aunque esto nos permitiría modificar cada parte del proceso, también lo haría largo y tedioso. Por suerte, hay un programa que nos permite automatizar todo el proceso, manteniendo la posibilidad de modificarlo dependiendo de las necesidades de cada proyecto. Se trata de \programa{GNU make}, un programa muy polivalente que nos permite automatizar multitud de tareas~\footnote{De hecho, con la ayuda de \programa{make} se genera esta memoria en formato PDF a partir de archivos escritos en \LaTeX.}.

De esta forma es posible usar el editor de textos que queramos para generar nuestro código (por ejemplo \programa{VIM} o \programa{emacs}), modificar las opciones de compilación, enlazado y carga de una forma rápida y sencilla. Podemos cambiar las opciones de compilación y enlazado en \programa{MPIDE} mediante la modificación de un archivo de configuración del programa (\orden{platforms.txt}); pero estos cambios afectarían a todos los programas compilados con \programa{MPIDE} y es necesario reiniciar el programa para que tengan efecto.

Para esto he creado una estructura de carpetas a modo de plantilla, para las distintas aplicaciones. Dentro de esta carpeta tenemos diferentes subdirectorios que sirven para almacenar los diferentes tipos de archivos que necesita nuestro programa. En la figura~\ref{plantilla_carpeta} se puede ver la estructura de esta plantilla.

\begin{figure}[H]
\begin{center}
\tikzstyle{every node}=[draw=black,thick,anchor=west]
\tikzstyle{selected}=[draw=red,fill=red!30]
\tikzstyle{optional}=[dashed,fill=gray!50]
\begin{tikzpicture}[%
  grow via three points={one child at (0.5,-0.7) and
  two children at (0.5,-0.7) and (0.5,-1.4)},
  edge from parent path={(\tikzparentnode.south) |- (\tikzchildnode.west)}]
	\node{ChipKIT\_template}
		child { node {bin}}
		child { node {lib}}
		child { node {src}};
\end{tikzpicture}
\end{center}
\caption{Estructura de directorios de la plantilla}
\label{plantilla_carpeta}
\end{figure}

Función de cada carpeta:
\begin{description}
	\item[src] Aquí estrá el código fuente del programa que hayamos creado. Es el lugar en el que se encuentra el archivo de código fuente \orden{sketch.cpp} que contiene las definiciones de las funciones \orden{setup()} y \orden{loop()}. Además, si necesitamos crear nuestras propias bibliotecas (archivos .h junto con su correspondiente archivo .c o .cpp) podemos crearlos tanto en esta carpeta como en \orden{lib}.
	\item[lib] Las bibliotecas de terceros se deben copiar a esta carpeta. Usando el archivo Makefile compilaremos todas las bibliotecas y luego las enlazaremos con nuestro programa y la biblioteca core.
	\item[bin] Una vez el programa es enlazado, el arhivo con extensión `.elf' se guardará en esta carpeta. También se generará ahí el archivo final `.hex' que será el que se cargue en la placa microcontroladora.
\end{description}

Para poder realizar todo el proceso con \programa{make} se necesita un archivo \orden{Makefile} en el que se describe cómo llevarlo acabo. Este archivo se escribe de la siguiente forma:
\begin{listing}[language=make]
objetivo: dependencia_1 dependencia_2 ...
	ordenes
\end{listing}
Esta estructura la podemos repetir para todos los objetivos que sean necesarios. El objetivo es lo que se quiere conseguir a través de las órdenes. Pero para poder ejecutarse es necesario que se cumplan las dependencias (que pueden ser archivos u otros objetivos). En el apéndice~\ref{chap:apendice2} se encuentra el Makefile completo y su funcionamiento.

\programa{make}, siguiendo las instrucciones que se encuentran en el \orden{Makefile}, se encarga de compilar, enlazar y empaquetar la biblioteca core en el archivo \orden{core.a} y luego compila todos los archivos en la carpeta \orden{lib} y la carpeta \orden{src}. Es decir, realiza las mismas operaciones que \programa{MPIDE}, a excepción de la primera de ellas (Buscar funciones en el sketch).

En la plantilla también he incluido un script escrito en Perl que puede ser ejecutado de forma explícita por el usuario y que permite replicar este comportamiento. El script se puede ejecutar sobre el archivo \orden{src/sketch.pde} si existe y genera el fichero en código fuente \orden{src/sketch.cpp} siempre y cuando este último archivo, en caso de existir, esté vacío.

Con el \orden{Makefile} incluido podemos ejecutar diversas acciones dependiendo del objetivo que elijamos. Si se ejecuta \orden{make src/sketch.S} (se puede hacer con cualquier archivo escrito en C o C++ que se encuentre en la carpeta) se compila pero no ensambla ni enlaza el archivo \orden{sketch.cpp}. Del mismo modo, si llamamos a la orden \orden{make src/sketch.o} se compila y ensambla \orden{sketch.cpp}. Por otro lado, \orden{make link} compilará, ensamblará y enlazará el programa guardando el resultado en \orden{bin/main.elf}. Este último es el objetivo por defecto, por lo que si ejecutamos \orden{make} sin ningún argumento, el resultado será el mismo.

Una vez enlazado podemos usar la orden \orden{make load} para cargar el programa en la placa conectada al puerto USB\@. Con esta orden se ejecuta \orden{avrdude} con los mismos parámetros que si fuese llamado a través de \programa{MPIDE}.

En el \orden{Makefile} ya hay dos puertos serie predefinidos según la plataforma: uno para GNU Linux y otro para Mac OS\@. Son los dos puertos en los que podremos encontrar el microntrolador por defecto. Solo hace falta modificar una variable para cambiar el puerto al que está conectada la placa en caso de que sea necesario.


\chapter{Ejemplos de utilización}
\label{cap4}
En este capítulo hablaré sobre los diferentes niveles de optimización que se pueden seleccionar durante la compilación y sus consecuencias en el programa final. Para ver dichas consecuencias se propondrá un programa que se ejecute en la placa microcontroladora y con el que se podrán medir los efectos de las optimizaciones. Por último se creará un programa que muestra las limitaciones de la biblioteca Arduino y que, además, está relacionado con las telecomunicaciones.
\section{Niveles de optimización}
Durante la compilación es posible especificar, mediante la opción \comando{-O} diferentes niveles de optimización. Estos niveles agrupan una serie de optimizaciones.

\begin{table}[H]
\begin{center}
\begin{tabularx}{\textwidth}{|l|X|}
    \hline
    Nivel de optimización & Descripción\\
    \hline
    \hline
   	\comando{-O0} & Opción por defecto. Reduce el tiempo de compilación y facilita la depuración.\\
    \hline
    \comando{-O1} & Optimiza, pero aumenta el tiempo de compilación y la memoria empleada.\\
    \hline
    \comando{-O2} & Optimiza más todavía. \programa{GCC} realiza casi todas la optimizaciones que no impliquen aumento de la velocidad a costa de un mayor tamaño del programa.\\
    \hline
    \comando{-O3} & Optimiza todavía más que \comando{-O2}, incluyendo las optimizaciones que aumenten el tamaño del ejecutable.\\
    \hline
    \comando{-Os} & Optimiza el tamaño del programa. Habilita todas las optimizaciones de \comando{-O2} que normalmente no aumenten el tamaño del código y realiza más optimizaciones destinadas a reducir el tamaño del programa.\\
    \hline
  \end{tabularx}
\end{center}
\caption{Niveles de optimización}
\label{tab:optimizacionx}
\end{table}

Cada uno de estos niveles (a excepción de \comando{-Os}) añade una serie de optimizaciones sobre el nivel inicial \comando{-O0}.

Para poder averiguar que optimizaciones lleva acabo el compilador dependiendo del nivel podemos emplear el comando \comando{pic32-gcc -Ox -Q --help=optimizers | grep enabled} sustituyendo \comando{x} por el nivel que queramos. Esta orden muestra todos las posibles optimizaciones, indicando si están habilitadas o no y para ver solo las habilitadas pasamos el resultado a \comando{grep} para filtrarlas.

\begin{table}[H]
\begin{center}
	\begin{tabular}{|ll|}
		\hline
		-falign-loops & -fargument-alias\\
		-fbranch-count-reg & -fcommon\\
		-fdata-sections & -fdce\\
		-fdelete-null-pointer-checks & -fdse\\
		-fearly-inlining & -fgcse-lm\\
		-finline-functions-called-once & -fivopts\\
		-fjump-tables & -fmath-errno\\
		-fmove-loop-invariants & -fpeephole\\
		-frename-registers & -fsched-critical-path-heuristic\\
		-fsched-dep-count-heuristic & -fsched-group-heurisitc\\
		-fsched-interblock & -fsched-last-insn-heuristic\\
		-fsched-spec & -fsched-spec-insn-heuristic\\
		-fsched-stalled-insns-dep & -fsigned-zeros\\
		-fsplit-ivs-in-unroller & -ftoplevel-reorder\\
		-ftrapping-math & -ftree-cselim\\
		-ftree-forwprop & -ftree-loop-im\\
		-ftree-loop-ivcanon & -ftree-loop-optimize\\
		-ftree-phiprop & -ftree-pta\\
		-ftree-reassoc & -ftree-scev-cprop\\
		-ftree-slp-vectorize & -ftree-vect-loop-version\\
		-funit-at-a-time & -fvar-tracking\\
		-fvar-tracking-assignments & -fweb\\
		\hline
	\end{tabular}
\end{center}
\caption{Optimizaciones del nivel 0.}
\label{opt0}
\end{table}

En el nivel más bajo, \programa{gcc} ya realiza optimizaciones. Todas las optimizaciones que comienzan con \comando{-ftree} afectan al árbol sintáctico que utiliza \programa{gcc} como representación interna del programa. También se puede observar como una de las optimizaciones que incluye este nivel es \comando{-fdata-sections} y a pesar de esto \programa{MPIDE} la incluye explícitamente cuando ejecuta la compilación. Otro grupo de optimizaciones que ya aparecen en este nivel son las que comienzan \comando{-fsched} que afectan a la reordenación de instrucciones (en ensamblador) para aprovechar lo mejor posible la segmentación de instrucciones del procesador evitando ciclos de espera. En este nivel, si el programa tiene una función que solo es llamada una vez, su llamada será sustituida por el contenido de la función, directamente, gracias a la opción \comando{-finline-functions-called-once}.

\begin{table}[H]
\begin{center}
	\begin{tabular}{|ll|}
		\hline
		-fcprop-registers & -fdefer-pop\\
		-fdelayed-branch & -fforward-propagate\\
		-fguess-branch-probability &-fif-conversion\\
		-fif-conversion2 & -fipa-pure-const\\
		-fipa-reference & -fmerge-constants\\
		-fomit-frame-pointer & -fsplit-wide-types\\
		-ftree-ccp & -ftree-ch\\
		-ftree-copy-prop & -ftree-copyrename\\
		-ftree-dce & -ftree-dominator-opts\\
		-ftree-dse & -ftree-fre\\
		-ftree-sink & -ftree-sra\\
		-ftree-ter & \\
		\hline
	\end{tabular}
\end{center}
\caption{Optimizaciones que añade el nivel `O1' al nivel `O0'.}
\label{optimizaciones1}
\end{table}

En el nivel 1 se añaden más optimizaciones que afectan al arbol sintáctico, suponiendo estas casi la mitad de las nuevas optimizaciones. \comando{-fif-conversion} y \comando{-fif-conversion2} tienen como obejtivo tratar de transformar secciones de código donde se usan saltos condicionales para que no los requieran, reduciendo de esta forma el número de instrucciones. También encontramos aquí \comando{-fdelayed-branch} que intentará reordenar las instrucciones para aprovechar el tiempo que se pierde durante saltos condicionales. Si se produce una instrucción de salto condicional, no es posible saber cual es la instrucción que se debe ejecutar después hasta que acabe, por lo que normalmente la siguiente instrucción suele ser un \comando{nop}. Si el procesador lo soporta, se mueve una de las instrucciones anteriores a la instrucción de salto que no tenga ninguna dependencia con esta, para aprovechar el hueco libre. En lugar de colocar después del salto, una instrucción anterior, \programa{gcc} también trata de predecir, con la opción \comando{-fguess-branch-probability}, que se ejecutará despues por lo que este lugar lo ocupará la primera instrucción de uno de los destinos. Si la predicción resulta erronea, se descartará esta instrucción.

\begin{table}[H]
\begin{center}
	\begin{tabular}{|ll|}
		\hline
		-falign-functions & -falign-jumps\\
		-falign-labels & -fcaller-saves\\
		-fcrossjumping & -fcse-follow-jumps\\
		-fexpensive-optimizations & -fgcse\\
		-finline-small-functions & -fipa-cp\\
		-fipa-sra & -foptimize-register-move\\
		-foptimize-sibling-calls & -fpeephole2\\
		-fregmove & -fremove-local-statics\\
		-freorder-blocks & -freorder-functions\\
		-frerun-cse-after-loop & -fschedule-insns\\
		-fschedule-insns2 & -fstrict-aliasing\\
		-fthread-jumps & -ftree-builtin-call-dce\\
		-ftree-pre & -ftree-switch-conversion\\
		-ftree-vrp & \\
		\hline
	\end{tabular}
\end{center}
\caption{Optimizaciones que añade el nivel `O2' al nivel `O1'.}
\label{optimizaciones2}
\end{table}

El nivel 2 incluye muchas más optimizaciones. En este nivel encontramos las optimizaciones \comando{-falign} que se encargan de alinear el comienzo de funciones, saltos y etiquetas (\comando{-falign-functions}, \comando{-falign-jumps} y \comando{-falign-labels}) en memoria. Su alineamiento en memoria supone una carga más rápida en detrimento del tamaño de la aplicación final, ya que se necesita añadir \comando{nop}s para que las funciones comiencen en bloques de N bytes, donde N depende del procesador.Con la opción \comando{-finline-small-functions} el contenido de las funciones pequeñas se copia en el lugar del código donde son llamadas. El requisito que deben cumplir estas funciones es que el espacio que ocupen al copiarse sea inferior al que sería ocupado si fuesen llamadas. De este modo, si usamos este nivel de optimización podemos escribir funciones triviales que nos ayuden a que el programa sea mas legible sin perder rendimiento ni espacio.

\begin{table}[H]
\begin{center}
	\begin{tabular}{|ll|}
		\hline
		-fgcse-after-reload & -finline-functions\\
		-fipa-cp-clone & -fpredictive-commoning\\
		-ftree-pre-partial-partial & -ftree-vectorize\\
		-funswitch-loops & \\
		\hline
	\end{tabular}
\end{center}
\caption{Optimizaciones que añade el nivel `O3' al nivel `O2'.}
\label{optimizaciones3}
\end{table}

En el nivel más alto \programa{gcc} considera todas las funciones para que su contenido sustituya a las llamadas a las mismas, gracias a la opción \comando{-finline-functions}. Es decir, ya no se sustituyen solo las funciones cuyo contenido sea menor que la sobrecarga que supone la llamada a la función. La consecuencia directa de está optimización es un aumento considerable del programa final y aumento del rendimiento al eliminar el código de llamada y retorno de una función. 

\begin{table}[H]
\begin{center}
	\begin{tabular}{|ll|}
		\hline
		-falign-functions & -falign-jumps\\
		-falign-loops & -falign-labels\\
		-freorder-blocks & -freorder-blocks-and-partition\\
		-fprefetch-loop-arrays & -ftree-vect-loop-version\\
		\hline
\end{tabular}
\end{center}
\caption{Optimizaciones del nivel `O2' que no se realizan en el nivel `Os'.}
\label{optimizaciones_s}
\end{table}

Las optimizaciones que lleva acabo el compilador cuando se activa la opción \comando{-Os} (Optimizar para el tamaño) son las mismas que cuando se usa \comando{-O2} a excepción de las que se encuentran en la tabla \ref{optimizaciones_s}. Estas optimizaciones son las que, aumentando el rendimiento, mas aumentan el tamaño de la aplicación final.

Cada nivel ofrece diferentes optimizaciones, aumentando el rendimiento con forme se aumenta el nivel. Es necesario valorar que nivel es el más adecuado para la aplicación. Por ejemplo, en un entorno empotrado (embedded) como sería el de un microcontrolador, debemos tener en cuenta la cantidad de memoria que tiene disponible al elegir la optimización del programa. Es posible elegir un nivel alto de optimización si el programa es pequeño y, por lo tanto, la memoria es suficiente. Si por el contrario la memoria es escasa se tendrá que elegir una optimización que no aumente el tamaño del programa. En el caso de depuración del código, el nivel más adecuado suele ser el más bajo ya que en este nivel, el compilador no modifica la estructura del programa, ni reordena instrucciones, haciendo más fácil la lectura del mismo en ensamblador.

Por último, es importante mencionar que se puede elegir un nivel de los cuatro disponibles y añadir más optimizaciones o quitarlas en la línea de comandos. Por ejemplo, sería posible elegir el nivel 3 de optimización, sin usar \comando{-finline-functions} añadiendo como argumento \comando{-fno-inline-functions}. Todas las opciones mencionadas antes se pueden negar añadiendo `no' a la opción. También podemos añadir optimizaciones que no se encuentran en ninguno de estos niveles, como \comando{-funroll-loops}\footnote{Esta opción `desenrolla' bucles, repitiendo el bloque de ejecución y reduciendo las veces que se ejecuta.} en la línea de comandos.

\section{Aplicación para ChipKIT}

Para ver los efectos de las optimizaciones que \programa{gcc} pueda efectuar, he creado una aplicación para ChipKIT. Se trata de un terminal al que podemos acceder de forma remota. Conectándonos a la placa a través de la red de área local tendremos acceso a un terminal básico en el que se pueden ejecutar varios comandos. La conexión se realiza utilizando el protocolo TCP en el puerto 23. He elegido este puerto al ser el estándar de Telnet aunque hubiese sido posible elegir cualquier otro. Para conectarse a la placa solo es necesario un programa que pueda establecer esta conexión, por lo que se puede usar el propio \programa{telnet} o \programa{netcat}. Con este último trabajamos directamente en la capa de transporte del modelo OSI, obviando todas las convenciones, procedimientos y comandos de la capa de aplicación.

Una vez establecida la conexión con la aplicación hay disponibles varios comandos que se pueden ejecutar. Se trata de comandos de ordenación en su mayoría, que permiten ordenar un array de números que se pasa como argumento del comando. Los elementos del array pueden estar separados por espacios, tabulaciones o comas. Dependiendo del nombre del comando, el array se ordena con un algoritmo determinado.
\begin{description}
	\item[\comando{bubble0}, \comando{bubble1}, \comando{bubble2}, \comando{bubble3}] Ordenan los números usando el algoritmo `Bubble Sort'.
	\item[\comando{quick0}, \comando{quick1}, \comando{quick2}, \comando{quick3}] Ordenan los números usando el algoritmo `Quick Sort'.
	\item[\comando{sel0}, \comando{sel1}, \comando{sel2}, \comando{sel3}] Ordenan los números usando el algoritmo `Selection Sort'.
	\item[\comando{comparar}] Reordena un array de números aleatorios usando todos los algoritmos anteriores.
	\item[\comando{help}] Muestra los comandos disponibles.
	\item[\comando{exit}] Cierra la conexión.
\end{description}

Por ejemplo, si quisieramos ordenar el vector [9 8 7 6 5 4 3 2 1] usando el algoritmo `Bubble Sort' escribiríamos lo siguiente:\footnote{Este vector representa el peor caso para este algoritmo (para esta longitud de vector), al estar los números ordenados de mayor a menor.}
\begin{lstlisting}
bubble0 9 8 7 6 5 4 3 2 1
\end{lstlisting}

El número que acompaña a cada uno de los comandos indica el nivel de optimización con el que han sido compilados, de este modo es posible probar el efecto de los distintos niveles sin tener que recompilar y volver a cargar el programa. En el programa se han implementado dos modos para conocer el tiempo que tardan en completarse los comandos. El primero de ellos implica llamar a las funciones \comando{millis()} y \comando{micros()} antes de que se ejecute el comando y al finalizar este. Estos dos comandos devuelven la cuenta actual de dos contadores:uno que se incrementa cada milisegundo y otro que lo hace cada microsegundo, por lo que si calculamos la diferencia entre ambos instantes (antes y después de la ejecución del comando) obtendremos el tiempo que ha tardado. Debido a que la longitud del array que es posible ordenar puede variar entre dos y 4096 elementos, el contador que utiliza el \comando{micros()} puede desbordar debido a que el tiempo de ejecución es mayor que el que puede alcanzar este contador y dar un valor incorrecto, por lo que es necesario utilizar \comando{millis()}. Tampoco sería posible usar solo \comando{millis()} al carecer de la precisión suficiente cuando se ordenan arrays cortos. El tiempo obtenido se envía al cliente y se muestra en la pantalla OLED del chipKIT I/O Shield. Además, durante la ejecución de un comando, el pin 70 de la placa microcontroladora (que corresponde al LED LD1 en el I/O Shield) se pone a nivel alto, lo que permite, no solo mostrar que se está ejecutando un comando, sino también medir el tiempo que permanece a nivel alto con un osciloscopio cuya sonda este conectada a este pin. 

Para automatizar el proceso de comparación entre los diferentes algoritmos y niveles de optimización, he incluido en el programa el comando \comando{comparar}. Este comando genera un array aleatorio de máxima longitud y lo ordena con cada uno de los comandos de ordenación disponibles, mostrando los resultados en pantalla.

En el apéndice \ref{chap:apendice3} se encuentra una descripción detallada del funcionamiento interno del programa.

\subsection{Funciones de ordenación}
Las funciones de ordenación que hemos comentado anteriormente, las que usa el programa par ordenar el array que le envíamos, se encuentran declaradas en el archivo \programa{funciones.h} y definidas en \programa{funciones.c}.

\subsubsection{Bubble Sort}
Este es el algoritmo más sencillo de los tres implementados. Se recorre el array comparando un elemento con el siguiente e intercambiándolos en el caso de que el primero sea mayor que el segundo. Para completar el ordenamiento es necesario recorrer el array varias veces.
\begin{lstlisting}[caption=Implementación de Bubble Sort]
void bubble0(int * array, int len) {
  int c, d, swap;

  for (c = 0; c < len; c++) {
    for (d = 0; d < (len - c - 1); d++) {
      if(array[d] > array[d+1]) {
        swap = array[d];
        array[d] = array[d+1];
        array[d+1] = swap;
      }
    }
  }
}  
\end{lstlisting}\label{Implementación de Bubble Sort en nuestro programa}

\subsubsection{Quick Sort}
En el algoritmo de ordenamiento rápido, se elige un elemento del array que denominaremos pivote. A la izquierda del pivote se colocan los elementos menores a él, mientras que a la derecha colocaremos los elementos mayores. De esta forma el elemnto `pivote' ocupará su posición final de la misma. Se repite el procedimiento de forma recursiva con los sub-arrays que quedan a la izquierda y a la derecha del pivote, hasta que todo el array esté ordenado. En el programa se ha optado por elegir como elemento `pivote' el primero del array.
\begin{lstlisting}[caption=Implementación de Quick Sort]
void quicksort0(int *v, int b, int t) {
  if(b < t) {
    int pivote = colocar0(v, b, t);
    quicksort0(v, b, pivote - 1);
    quicksort0(v, pivote + 1, t);
  }
}

int colocar0(int *v, int b, int t) {
  int i;
  int pivote, valor_pivote;
  int temp;

  pivote = b;
  valor_pivote = v[pivote];
  for (i = b + 1; i <= t; i++) {
    if (v[i] < valor_pivote) {
      pivote++;
      temp = v[i];
      v[i] = v[pivote];
      v[pivote] = temp;
    }
  }
  temp = v[b];
  v[b] = v[pivote];
  v[pivote] = temp;

  return pivote;
}
\end{lstlisting}

\subsubsection{Selection Sort}
El tercer algoritmo implementado en el programa es Selection Sort. Su funcionamiento consiste en recorrer el array, almacenándose el elemento de menor valor. Este elemento se extrae del array y se coloca al principio para, a continuación, volver a hacer lo mismo con el resto del array, hasta que esté ordenado.

\begin{lstlisting}[caption=Implementación de Selection Sort]
void selection0(int *array, int len) {
  for (int i = 0; i < len; ++i) {
    int index_of_min = i;
    for (int j = i; j < len; ++j) {
      if (array[index_of_min] > array[j]) {
        index_of_min = j;
      }
    }
    int temp = array[i];
    array[i] = array[index_of_min];
    array[index_of_min] = temp;
  }
}
\end{lstlisting}

\subsection{Análisis estático de la optimización en los algoritmos de ordenación.}
Una vez visto como están implementados los algoritmos de ordenación en C, podemos ver su aspecto una vez compilados y comparar entre los diferentes niveles de optimización y como afectan a estas funciones.

Comenzando con el algoritmo `Bubble Sort', tenemos en la figura \ref{bubble} el código de sus cuatro versiones para compararlas entre ellas.

\begin{figure}
\begin{center}
\begin{tabular}{c c}
	\begin{minipage}[t]{.5\textwidth}
		\begin{lstlisting}[caption=bubble0,basicstyle=\ttfamily\scriptsize, tabsize=2]
00000000 <_Z7bubble0Pii>:
   0:	18a00015 	blez	a1,58 <_Z7bubble0Pii+0x58>
   4:	24a5ffff 	addiu	a1,a1,-1
   8:	08000011 	j	44 <_Z7bubble0Pii+0x44>
   c:	240affff 	li	t2,-1
  10:	8c460000 	lw	a2,0(v0)
  14:	8c470004 	lw	a3,4(v0)
  18:	00e6402a 	slt	t0,a3,a2
  1c:	11000003 	beqz	t0,2c <_Z7bubble0Pii+0x2c>
  20:	24630001 	addiu	v1,v1,1
  24:	ac470000 	sw	a3,0(v0)
  28:	ac460004 	sw	a2,4(v0)
  2c:	0069302a 	slt	a2,v1,t1
  30:	14c0fff7 	bnez	a2,10 <_Z7bubble0Pii+0x10>
  34:	24420004 	addiu	v0,v0,4
  38:	24a5ffff 	addiu	a1,a1,-1
  3c:	10aa0006 	beq	a1,t2,58 <_Z7bubble0Pii+0x58>
  40:	00000000 	nop
  44:	18a0fffc 	blez	a1,38 <_Z7bubble0Pii+0x38>
  48:	00a04821 	move	t1,a1
  4c:	00801021 	move	v0,a0
  50:	08000004 	j	10 <_Z7bubble0Pii+0x10>
  54:	00001821 	move	v1,zero
  58:	03e00008 	jr	ra
  5c:	00000000 	nop
		\end{lstlisting}
	\end{minipage} 
	&
	\begin{minipage}[t]{.5\textwidth}
		\begin{lstlisting}[caption=bubble1, basicstyle=\ttfamily\scriptsize, tabsize=2]
00000000 <_Z7bubble1Pii>:
   0:	18a00015 	blez	a1,58 <_Z7bubble1Pii+0x58>
   4:	24a5ffff 	addiu	a1,a1,-1
   8:	08000011 	j	44 <_Z7bubble1Pii+0x44>
   c:	240affff 	li	t2,-1
  10:	8c460000 	lw	a2,0(v0)
  14:	8c470004 	lw	a3,4(v0)
  18:	00e6402a 	slt	t0,a3,a2
  1c:	11000003 	beqz	t0,2c <_Z7bubble1Pii+0x2c>
  20:	24630001 	addiu	v1,v1,1
  24:	ac470000 	sw	a3,0(v0)
  28:	ac460004 	sw	a2,4(v0)
  2c:	0069302a 	slt	a2,v1,t1
  30:	14c0fff7 	bnez	a2,10 <_Z7bubble1Pii+0x10>
  34:	24420004 	addiu	v0,v0,4
  38:	24a5ffff 	addiu	a1,a1,-1
  3c:	10aa0006 	beq	a1,t2,58 <_Z7bubble1Pii+0x58>
  40:	00000000 	nop
  44:	18a0fffc 	blez	a1,38 <_Z7bubble1Pii+0x38>
  48:	00a04821 	move	t1,a1
  4c:	00801021 	move	v0,a0
  50:	08000004 	j	10 <_Z7bubble1Pii+0x10>
  54:	00001821 	move	v1,zero
  58:	03e00008 	jr	ra
  5c:	00000000 	nop
		\end{lstlisting}
	\end{minipage} \\

	\begin{minipage}[t]{.5\textwidth}
		\begin{lstlisting}[caption=bubble2, basicstyle=\ttfamily\scriptsize, tabsize=2]
00000000 <_Z7bubble2Pii>:
   0:	18a00013 	blez	a1,50 <_Z7bubble2Pii+0x50>
   4:	24a5ffff 	addiu	a1,a1,-1
   8:	240affff 	li	t2,-1
   c:	58a0000e 	blezl	a1,48 <_Z7bubble2Pii+0x48>
  10:	24a5ffff 	addiu	a1,a1,-1
  14:	00801021 	move	v0,a0
  18:	00001821 	move	v1,zero
  1c:	8c460000 	lw	a2,0(v0)
  20:	8c470004 	lw	a3,4(v0)
  24:	24630001 	addiu	v1,v1,1
  28:	00e6482a 	slt	t1,a3,a2
  2c:	11200003 	beqz	t1,3c <_Z7bubble2Pii+0x3c>
  30:	0065402a 	slt	t0,v1,a1
  34:	ac470000 	sw	a3,0(v0)
  38:	ac460004 	sw	a2,4(v0)
  3c:	1500fff7 	bnez	t0,1c <_Z7bubble2Pii+0x1c>
  40:	24420004 	addiu	v0,v0,4
  44:	24a5ffff 	addiu	a1,a1,-1
  48:	14aafff0 	bne	a1,t2,c <_Z7bubble2Pii+0xc>
  4c:	00000000 	nop
  50:	03e00008 	jr	ra
  54:	00000000 	nop
		\end{lstlisting}
	\end{minipage}
	&
	\begin{minipage}[t]{.5\textwidth}
		\begin{lstlisting}[caption=bubble3, basicstyle=\ttfamily\scriptsize, tabsize=2]
00000000 <_Z7bubble3Pii>:
   0:	18a00013 	blez	a1,50 <_Z7bubble3Pii+0x50>
   4:	24a5ffff 	addiu	a1,a1,-1
   8:	240affff 	li	t2,-1
   c:	58a0000e 	blezl	a1,48 <_Z7bubble3Pii+0x48>
  10:	24a5ffff 	addiu	a1,a1,-1
  14:	00801021 	move	v0,a0
  18:	00001821 	move	v1,zero
  1c:	8c460000 	lw	a2,0(v0)
  20:	8c470004 	lw	a3,4(v0)
  24:	24630001 	addiu	v1,v1,1
  28:	00e6482a 	slt	t1,a3,a2
  2c:	11200003 	beqz	t1,3c <_Z7bubble3Pii+0x3c>
  30:	0065402a 	slt	t0,v1,a1
  34:	ac470000 	sw	a3,0(v0)
  38:	ac460004 	sw	a2,4(v0)
  3c:	1500fff7 	bnez	t0,1c <_Z7bubble3Pii+0x1c>
  40:	24420004 	addiu	v0,v0,4
  44:	24a5ffff 	addiu	a1,a1,-1
  48:	14aafff0 	bne	a1,t2,c <_Z7bubble3Pii+0xc>
  4c:	00000000 	nop
  50:	03e00008 	jr	ra
  54:	00000000 	nop
		\end{lstlisting}
	\end{minipage} \\
\end{tabular}
\label{bubble}
\end{center}
\caption{Implementaciones de `Bubble Sort' en ensamblador}
\end{figure}



\chapter{Conclusiones}
\label{cap5}
En este capítulo expondré las conclusiones obtenidas durante la elaboración de este trabajo. Se subdivide en dos apartados: conclusiones técnicas, donde comentaré los resultados de las mediciones y el proceso usado para obtenerlas; conclusiones personales, donde expondré mis pensamientos acerca de este trabajo junto con las cosas que he podido aprender durante la realización del mismo.

\subsection{Conclusiones técnicas}
Este trabajo tenía diferentes objetivos. El primero de ellos era analizar el proceso de compilación, enlazado y carga en la placa que lleva a cabo el IDE de Microchip (\programa{MPIDE}) de forma transparente al usuario. Una vez conocido este proceso había que desarrollar un método alternativo que me permitiera más transparencia y flexibilidad. Usando la opción de mostrar las órdenes ejecutadas en \programa{MPIDE} pude ver los programas a los que llama y en que orden. A partir de esta información ideé un archivo \orden{Makefile} que realizara las mismas acciones y que diese como resultado el programa cargado en la placa. Más tarde expandí la idea de un archivo Makefile a la creación de una plantilla de programa, es decir, una organización de carpetas en las que organizar las bibliotecas de terceros que pueda necesitar el programa, el código fuente y el ejecutable final. Fue interesante comprobar como había algunas acciones que el IDE llevaba a cabo antes de comenzar la compilación. A la hora de escribir un sketch en \programa{MPIDE} no es necesario declarar las funciones auxuiliares que pueda requerir la aplicación, a diferencia de como se haría en un programa normal escrito en C o C++. El IDE hace esta labor por el usuario, buscando las funciones que se hayan escrito en el sketch y añadiendo su definición al principio del mismo antes de comenzar la compilación. A parte de esto también añade un `include' con el archivo de cabecera que contiene todas las definiciones de las funciones de la biblioteca `core'.

El segundo objetivo planteado era comparar el efecto de los distintos niveles de optimización tenían sobre el rendimiento final del programa. Para esto se proponía la creación de un programa en el que se pudiese realizar esta comparación. El programa tiene como principal objetivo la ordenación de números usando diferentes algoritmos de ordenación, cada uno de ellos compilado con los cuatro principales niveles de optimización. El programa es accesible a través de la red, usando una conexión TCP en el puerto 21, es posible conectarse usando \programa{telnet} o \programa{netcat}. El programa presenta una línea de comandos en la que podemos llamar las distintas funciones pasando como argumento la lista de números a ordenar. El programa está creado usando como base la plantilla creada como parte del primer objetivo y, por lo tanto, fue posible desarrollarlo sin necesidad de abrir el IDE de ChipKIT en ningún momento, tan solo haciendo uso de las bibliotecas que incluye y de la \orden{toolchain} con la que se instala. Debido a la naturaleza del programa y usando \programa{netcat} es posible ordenar conjuntos de hasta 8192 números leyéndo la orden y los números desde un fichero y enviándoselos usando \programa{netcat}, sin necesidad de tener que escribirlos uno a uno. El problema principal a la hora de llevar a cabo esto es que los bytes que recibe el programa a través de la red se almacenan en un buffer de tan solo 1000 bytes por lo que era necesario ir leyendo el contenido de este y recibiendo más datos hasta encontrar un retorno de carro o salto de línea, punto en el que el programa asume que finaliza la órden.

Los resultados después de realizar las mediciones fueron los esperados. Cuando se aplicaba el primer nivel de optimzación el rendimiento era considerablemente mejor a la misma función sin optimizar. Aumentando el nivel de optimización mas allá solo suponía una pequeña mejora en el segundo nivel siendo práctimante despreciable en el tercero e incluso empeorando en algún caso. 

Una vez hecho esto, el siguiente objetivo era poner de manifiesto las limitaciones de la biblioteca Arduino frente a funciones de más bajo nivel que el fabricante provea en una biblioteca o directamente usando los registros del procesador y de periféricos. Para esto se utilizó el programa del LED parpadeante. Una versión de este se hizo usando las funciones que nos proporciona la biblioteca de Arduino para escribir en las salidas digitales, mientras que la otra versión se hizo modificando el valor de los registros de periféricos directamente reduciendo considerablemente la sobrecarga que conllevan todas las operaciones que se realizan en la biblioteca de Arduino. El resultado es una diferencia en frecuencia de apagado y encendido del LED de más de un orden de magnitud. Hay que tener en cuenta que al hacer uso de los registros del microcontrolador directamente se pierde la portabilidad que proporciona el uso de la biblioteca, de forma que el segundo programa solo es usable en el modelo concreto de microcontrolador para el que fue programado, mientras que la versión de Arduino puede funcionar en una amplía variedad de dispositivos sin necesidad de realizar ningún cambio.

Por último, se proponía la creación de un programa que pusiese en práctica conocimientos adquiridos durante la carrera. El programa tendría como objetivo detectar la presencia de un tono en una entrada analógica cuya señal se obtendría con un micrófono. Como ya se comentó en su correspondiente apartado, se eligió este programa porque permite poner en práctica conocimientos de procesamiento de señales adquiridos durante la carrera. El resultado ha sido un programa con un funcionamiento análogo al de un DSP, en el que se toma una muestra y esta se procesa antes de tomar la muestra siguiente para, al terminar el muestreo, realizar el cálculo final y obtener el valor. Bajo unas condiciones similares al DSP utilizado durante la carrera, el microcontrolador era capaz de obtener el mismo rendimiento que el dispositivo de Analog Devices, procesando muestras a 8kHz, con la ventaja de ser más barato, fácil de programar y multipropósito.

\subsection{Conclusiones personales}

En general, ha sido un trabajo muy interesante en el que he tenido que poner en práctica conocimientos de diferentes áreas adquiridos durante la carrera, además de necesitar otros nuevos.
%C
%Electronica
%Latex
%Git
%Procesado de señales
%Makefiles
%Ensamblador




\nocite{*}

\backmatter % Apéndices, bibliografia ...


\clearpage
\addcontentsline{toc}{chapter}{Bibliografia y referencias}
\bibliographystyle{plain}
\bibliography{bibliografia}


\addcontentsline{toc}{chapter}{Software usado}
\chapter*{Software utilizado}
% -*-programas.tex-*-
% Este fichero es parte de la plantilla LaTeX para
% la realización de Proyectos Final de Carrera, protejido
% bajo los términos de la licencia GFDL.
% Para más información, la licencia completa viene incluida en el
% fichero fdl-1.3.tex

% Copyright (C) 2009 Pablo Recio Quijano 
Para la realización de esta memoria he utilizado diferentes programas.

\section*{Vim}
\programa{Vim} es uno de los editores de texto más utilizado en entornos Unix, enfocado en programación. La ventaja de usar este editor es que está preinstalado en cualquier sistema Unix (puede estar instalado Vim o Vi) y es altamente personalizable mediante su archivo de configuración y con plugins.

\section*{Sublime Text + Latexing}
Otro editor de textos que he utilizado para la creación de esta memoria es \programa{Sublime Text} que, a diferencia de \programa{Vim}, es de pago. Al igual que \programa{Vim} es posible ampliar su funcionalidad con plugins.

\section*{Git}
Para mantener un control de versiones he optado por utilizar \programa{Git}. Con \programa{Git} puedo, no solo guardar las versiones, sino también crear diferentes ramas o utilizar un servicio como Github.com, que me permite almacenar tanto esta memoria como los programas que he creado para este trabajo.

\section*{GNU Make}

\programa{GNU Make} permite la automatización de procesos como pueden ser, por ejemplo, compilar varios archivos que contengan código fuente y enlazarlos para crear un ejecutable. También puede ser usado para compilar archivos latex y crear un fichero en formato pdf.

Para más información \cite{pdf:make}


%\addcontentsline{toc}{chapter}{Apéndice 1: Resultado de la compilación en \programa{MPIDE}}
\appendix
\chapter{Apéndice 1: Resultado de la compilación en \programa{MPIDE}}\label{chap:apendice1}
\begin{lstlisting}[language=bash, label=bash:mpidecompilado, basicstyle=\scriptsize, breaklines=true]

/home/lluis/mpide/hardware/pic32/compiler/pic32-tools/bin/pic32-g++  -O2  -c  -mno-smart-io  -w  -fno-exceptions  -ffunction-sections  -fdata-sections  -g  -mdebugger  -Wcast-align  -fno-short-double  -mprocessor=32MX795F512L  -DF_CPU=80000000L  -DARDUINO=23  -D_BOARD_MEGA_    -DMPIDEVER=0x01000305  -DMPIDE=23      -I/home/lluis/mpide/examples/1.Basics/Blink   -I/home/lluis/mpide/hardware/pic32/cores/pic32   -I/home/lluis/mpide/hardware/pic32/variants/Max32    /tmp/build7676535287638374045.tmp/Blink.cpp  -o  /tmp/build7676535287638374045.tmp/Blink.cpp.o
/home/lluis/mpide/hardware/pic32/compiler/pic32-tools/bin/pic32-g++  -O2  -g1  -c  -mprocessor=32MX795F512L  -DF_CPU=80000000L  -DARDUINO=23  -D_BOARD_MEGA_    -DMPIDEVER=0x01000305  -DMPIDE=23      -I/home/lluis/mpide/hardware/pic32/cores/pic32   -I/home/lluis/mpide/hardware/pic32/variants/Max32    /home/lluis/mpide/hardware/pic32/cores/pic32/crtn.S  -o  /tmp/build7676535287638374045.tmp/crtn.S.o
/home/lluis/mpide/hardware/pic32/compiler/pic32-tools/bin/pic32-g++  -O2  -g1  -c  -mprocessor=32MX795F512L  -DF_CPU=80000000L  -DARDUINO=23  -D_BOARD_MEGA_    -DMPIDEVER=0x01000305  -DMPIDE=23      -I/home/lluis/mpide/hardware/pic32/cores/pic32   -I/home/lluis/mpide/hardware/pic32/variants/Max32    /home/lluis/mpide/hardware/pic32/cores/pic32/crti.S  -o  /tmp/build7676535287638374045.tmp/crti.S.o
/home/lluis/mpide/hardware/pic32/compiler/pic32-tools/bin/pic32-g++  -O2  -g1  -c  -mprocessor=32MX795F512L  -DF_CPU=80000000L  -DARDUINO=23  -D_BOARD_MEGA_    -DMPIDEVER=0x01000305  -DMPIDE=23      -I/home/lluis/mpide/hardware/pic32/cores/pic32   -I/home/lluis/mpide/hardware/pic32/variants/Max32    /home/lluis/mpide/hardware/pic32/cores/pic32/cpp-startup.S  -o  /tmp/build7676535287638374045.tmp/cpp-startup.S.o
/home/lluis/mpide/hardware/pic32/compiler/pic32-tools/bin/pic32-gcc  -O2  -c  -mno-smart-io  -w  -ffunction-sections  -fdata-sections  -g  -mdebugger  -Wcast-align  -fno-short-double  -mprocessor=32MX795F512L  -DF_CPU=80000000L  -DARDUINO=23  -D_BOARD_MEGA_    -DMPIDEVER=0x01000305  -DMPIDE=23      -I/home/lluis/mpide/hardware/pic32/cores/pic32   -I/home/lluis/mpide/hardware/pic32/variants/Max32    /home/lluis/mpide/hardware/pic32/cores/pic32/wiring_digital.c  -o  /tmp/build7676535287638374045.tmp/wiring_digital.c.o
/home/lluis/mpide/hardware/pic32/compiler/pic32-tools/bin/pic32-gcc  -O2  -c  -mno-smart-io  -w  -ffunction-sections  -fdata-sections  -g  -mdebugger  -Wcast-align  -fno-short-double  -mprocessor=32MX795F512L  -DF_CPU=80000000L  -DARDUINO=23  -D_BOARD_MEGA_    -DMPIDEVER=0x01000305  -DMPIDE=23      -I/home/lluis/mpide/hardware/pic32/cores/pic32   -I/home/lluis/mpide/hardware/pic32/variants/Max32    /home/lluis/mpide/hardware/pic32/cores/pic32/WInterrupts.c  -o  /tmp/build7676535287638374045.tmp/WInterrupts.c.o
/home/lluis/mpide/hardware/pic32/compiler/pic32-tools/bin/pic32-gcc  -O2  -c  -mno-smart-io  -w  -ffunction-sections  -fdata-sections  -g  -mdebugger  -Wcast-align  -fno-short-double  -mprocessor=32MX795F512L  -DF_CPU=80000000L  -DARDUINO=23  -D_BOARD_MEGA_    -DMPIDEVER=0x01000305  -DMPIDE=23      -I/home/lluis/mpide/hardware/pic32/cores/pic32   -I/home/lluis/mpide/hardware/pic32/variants/Max32    /home/lluis/mpide/hardware/pic32/cores/pic32/pins_arduino.c  -o  /tmp/build7676535287638374045.tmp/pins_arduino.c.o
/home/lluis/mpide/hardware/pic32/compiler/pic32-tools/bin/pic32-gcc  -O2  -c  -mno-smart-io  -w  -ffunction-sections  -fdata-sections  -g  -mdebugger  -Wcast-align  -fno-short-double  -mprocessor=32MX795F512L  -DF_CPU=80000000L  -DARDUINO=23  -D_BOARD_MEGA_    -DMPIDEVER=0x01000305  -DMPIDE=23      -I/home/lluis/mpide/hardware/pic32/cores/pic32   -I/home/lluis/mpide/hardware/pic32/variants/Max32    /home/lluis/mpide/hardware/pic32/cores/pic32/WSystem.c  -o  /tmp/build7676535287638374045.tmp/WSystem.c.o
/home/lluis/mpide/hardware/pic32/compiler/pic32-tools/bin/pic32-gcc  -O2  -c  -mno-smart-io  -w  -ffunction-sections  -fdata-sections  -g  -mdebugger  -Wcast-align  -fno-short-double  -mprocessor=32MX795F512L  -DF_CPU=80000000L  -DARDUINO=23  -D_BOARD_MEGA_    -DMPIDEVER=0x01000305  -DMPIDE=23      -I/home/lluis/mpide/hardware/pic32/cores/pic32   -I/home/lluis/mpide/hardware/pic32/variants/Max32    /home/lluis/mpide/hardware/pic32/cores/pic32/HardwareSerial_cdcacm.c  -o  /tmp/build7676535287638374045.tmp/HardwareSerial_cdcacm.c.o
/home/lluis/mpide/hardware/pic32/compiler/pic32-tools/bin/pic32-gcc  -O2  -c  -mno-smart-io  -w  -ffunction-sections  -fdata-sections  -g  -mdebugger  -Wcast-align  -fno-short-double  -mprocessor=32MX795F512L  -DF_CPU=80000000L  -DARDUINO=23  -D_BOARD_MEGA_    -DMPIDEVER=0x01000305  -DMPIDE=23      -I/home/lluis/mpide/hardware/pic32/cores/pic32   -I/home/lluis/mpide/hardware/pic32/variants/Max32    /home/lluis/mpide/hardware/pic32/cores/pic32/wiring_shift.c  -o  /tmp/build7676535287638374045.tmp/wiring_shift.c.o
/home/lluis/mpide/hardware/pic32/compiler/pic32-tools/bin/pic32-gcc  -O2  -c  -mno-smart-io  -w  -ffunction-sections  -fdata-sections  -g  -mdebugger  -Wcast-align  -fno-short-double  -mprocessor=32MX795F512L  -DF_CPU=80000000L  -DARDUINO=23  -D_BOARD_MEGA_    -DMPIDEVER=0x01000305  -DMPIDE=23      -I/home/lluis/mpide/hardware/pic32/cores/pic32   -I/home/lluis/mpide/hardware/pic32/variants/Max32    /home/lluis/mpide/hardware/pic32/cores/pic32/wiring_analog.c  -o  /tmp/build7676535287638374045.tmp/wiring_analog.c.o
/home/lluis/mpide/hardware/pic32/compiler/pic32-tools/bin/pic32-gcc  -O2  -c  -mno-smart-io  -w  -ffunction-sections  -fdata-sections  -g  -mdebugger  -Wcast-align  -fno-short-double  -mprocessor=32MX795F512L  -DF_CPU=80000000L  -DARDUINO=23  -D_BOARD_MEGA_    -DMPIDEVER=0x01000305  -DMPIDE=23      -I/home/lluis/mpide/hardware/pic32/cores/pic32   -I/home/lluis/mpide/hardware/pic32/variants/Max32    /home/lluis/mpide/hardware/pic32/cores/pic32/task_manager.c  -o  /tmp/build7676535287638374045.tmp/task_manager.c.o
/home/lluis/mpide/hardware/pic32/compiler/pic32-tools/bin/pic32-gcc  -O2  -c  -mno-smart-io  -w  -ffunction-sections  -fdata-sections  -g  -mdebugger  -Wcast-align  -fno-short-double  -mprocessor=32MX795F512L  -DF_CPU=80000000L  -DARDUINO=23  -D_BOARD_MEGA_    -DMPIDEVER=0x01000305  -DMPIDE=23      -I/home/lluis/mpide/hardware/pic32/cores/pic32   -I/home/lluis/mpide/hardware/pic32/variants/Max32    /home/lluis/mpide/hardware/pic32/cores/pic32/wiring.c  -o  /tmp/build7676535287638374045.tmp/wiring.c.o
/home/lluis/mpide/hardware/pic32/compiler/pic32-tools/bin/pic32-gcc  -O2  -c  -mno-smart-io  -w  -ffunction-sections  -fdata-sections  -g  -mdebugger  -Wcast-align  -fno-short-double  -mprocessor=32MX795F512L  -DF_CPU=80000000L  -DARDUINO=23  -D_BOARD_MEGA_    -DMPIDEVER=0x01000305  -DMPIDE=23      -I/home/lluis/mpide/hardware/pic32/cores/pic32   -I/home/lluis/mpide/hardware/pic32/variants/Max32    /home/lluis/mpide/hardware/pic32/cores/pic32/wiring_pulse.c  -o  /tmp/build7676535287638374045.tmp/wiring_pulse.c.o
/home/lluis/mpide/hardware/pic32/compiler/pic32-tools/bin/pic32-gcc  -O2  -c  -mno-smart-io  -w  -ffunction-sections  -fdata-sections  -g  -mdebugger  -Wcast-align  -fno-short-double  -mprocessor=32MX795F512L  -DF_CPU=80000000L  -DARDUINO=23  -D_BOARD_MEGA_    -DMPIDEVER=0x01000305  -DMPIDE=23      -I/home/lluis/mpide/hardware/pic32/cores/pic32   -I/home/lluis/mpide/hardware/pic32/variants/Max32    /home/lluis/mpide/hardware/pic32/cores/pic32/HardwareSerial_usb.c  -o  /tmp/build7676535287638374045.tmp/HardwareSerial_usb.c.o
/home/lluis/mpide/hardware/pic32/compiler/pic32-tools/bin/pic32-gcc  -O2  -c  -mno-smart-io  -w  -ffunction-sections  -fdata-sections  -g  -mdebugger  -Wcast-align  -fno-short-double  -mprocessor=32MX795F512L  -DF_CPU=80000000L  -DARDUINO=23  -D_BOARD_MEGA_    -DMPIDEVER=0x01000305  -DMPIDE=23      -I/home/lluis/mpide/hardware/pic32/cores/pic32   -I/home/lluis/mpide/hardware/pic32/variants/Max32    /home/lluis/mpide/hardware/pic32/cores/pic32/exceptions.c  -o  /tmp/build7676535287638374045.tmp/exceptions.c.o
/home/lluis/mpide/hardware/pic32/compiler/pic32-tools/bin/pic32-g++  -O2  -c  -mno-smart-io  -w  -fno-exceptions  -ffunction-sections  -fdata-sections  -g  -mdebugger  -Wcast-align  -fno-short-double  -mprocessor=32MX795F512L  -DF_CPU=80000000L  -DARDUINO=23  -D_BOARD_MEGA_    -DMPIDEVER=0x01000305  -DMPIDE=23      -I/home/lluis/mpide/hardware/pic32/cores/pic32   -I/home/lluis/mpide/hardware/pic32/variants/Max32    /home/lluis/mpide/hardware/pic32/cores/pic32/WString.cpp  -o  /tmp/build7676535287638374045.tmp/WString.cpp.o
/home/lluis/mpide/hardware/pic32/compiler/pic32-tools/bin/pic32-g++  -O2  -c  -mno-smart-io  -w  -fno-exceptions  -ffunction-sections  -fdata-sections  -g  -mdebugger  -Wcast-align  -fno-short-double  -mprocessor=32MX795F512L  -DF_CPU=80000000L  -DARDUINO=23  -D_BOARD_MEGA_    -DMPIDEVER=0x01000305  -DMPIDE=23      -I/home/lluis/mpide/hardware/pic32/cores/pic32   -I/home/lluis/mpide/hardware/pic32/variants/Max32    /home/lluis/mpide/hardware/pic32/cores/pic32/Print.cpp  -o  /tmp/build7676535287638374045.tmp/Print.cpp.o
/home/lluis/mpide/hardware/pic32/compiler/pic32-tools/bin/pic32-g++  -O2  -c  -mno-smart-io  -w  -fno-exceptions  -ffunction-sections  -fdata-sections  -g  -mdebugger  -Wcast-align  -fno-short-double  -mprocessor=32MX795F512L  -DF_CPU=80000000L  -DARDUINO=23  -D_BOARD_MEGA_    -DMPIDEVER=0x01000305  -DMPIDE=23      -I/home/lluis/mpide/hardware/pic32/cores/pic32   -I/home/lluis/mpide/hardware/pic32/variants/Max32    /home/lluis/mpide/hardware/pic32/cores/pic32/HardwareSerial.cpp  -o  /tmp/build7676535287638374045.tmp/HardwareSerial.cpp.o
/home/lluis/mpide/hardware/pic32/compiler/pic32-tools/bin/pic32-g++  -O2  -c  -mno-smart-io  -w  -fno-exceptions  -ffunction-sections  -fdata-sections  -g  -mdebugger  -Wcast-align  -fno-short-double  -mprocessor=32MX795F512L  -DF_CPU=80000000L  -DARDUINO=23  -D_BOARD_MEGA_    -DMPIDEVER=0x01000305  -DMPIDE=23      -I/home/lluis/mpide/hardware/pic32/cores/pic32   -I/home/lluis/mpide/hardware/pic32/variants/Max32    /home/lluis/mpide/hardware/pic32/cores/pic32/WMath.cpp  -o  /tmp/build7676535287638374045.tmp/WMath.cpp.o
/home/lluis/mpide/hardware/pic32/compiler/pic32-tools/bin/pic32-g++  -O2  -c  -mno-smart-io  -w  -fno-exceptions  -ffunction-sections  -fdata-sections  -g  -mdebugger  -Wcast-align  -fno-short-double  -mprocessor=32MX795F512L  -DF_CPU=80000000L  -DARDUINO=23  -D_BOARD_MEGA_    -DMPIDEVER=0x01000305  -DMPIDE=23      -I/home/lluis/mpide/hardware/pic32/cores/pic32   -I/home/lluis/mpide/hardware/pic32/variants/Max32    /home/lluis/mpide/hardware/pic32/cores/pic32/main.cpp  -o  /tmp/build7676535287638374045.tmp/main.cpp.o
/home/lluis/mpide/hardware/pic32/compiler/pic32-tools/bin/pic32-g++  -O2  -c  -mno-smart-io  -w  -fno-exceptions  -ffunction-sections  -fdata-sections  -g  -mdebugger  -Wcast-align  -fno-short-double  -mprocessor=32MX795F512L  -DF_CPU=80000000L  -DARDUINO=23  -D_BOARD_MEGA_    -DMPIDEVER=0x01000305  -DMPIDE=23      -I/home/lluis/mpide/hardware/pic32/cores/pic32   -I/home/lluis/mpide/hardware/pic32/variants/Max32    /home/lluis/mpide/hardware/pic32/cores/pic32/Tone.cpp  -o  /tmp/build7676535287638374045.tmp/Tone.cpp.o
/home/lluis/mpide/hardware/pic32/compiler/pic32-tools/bin/pic32-ar  rcs  /tmp/build7676535287638374045.tmp/core.a  /tmp/build7676535287638374045.tmp/crtn.S.o
/home/lluis/mpide/hardware/pic32/compiler/pic32-tools/bin/pic32-ar  rcs  /tmp/build7676535287638374045.tmp/core.a  /tmp/build7676535287638374045.tmp/crti.S.o
/home/lluis/mpide/hardware/pic32/compiler/pic32-tools/bin/pic32-ar  rcs  /tmp/build7676535287638374045.tmp/core.a  /tmp/build7676535287638374045.tmp/cpp-startup.S.o
/home/lluis/mpide/hardware/pic32/compiler/pic32-tools/bin/pic32-ar  rcs  /tmp/build7676535287638374045.tmp/core.a  /tmp/build7676535287638374045.tmp/wiring_digital.c.o
/home/lluis/mpide/hardware/pic32/compiler/pic32-tools/bin/pic32-ar  rcs  /tmp/build7676535287638374045.tmp/core.a  /tmp/build7676535287638374045.tmp/WInterrupts.c.o
/home/lluis/mpide/hardware/pic32/compiler/pic32-tools/bin/pic32-ar  rcs  /tmp/build7676535287638374045.tmp/core.a  /tmp/build7676535287638374045.tmp/pins_arduino.c.o
/home/lluis/mpide/hardware/pic32/compiler/pic32-tools/bin/pic32-ar  rcs  /tmp/build7676535287638374045.tmp/core.a  /tmp/build7676535287638374045.tmp/WSystem.c.o
/home/lluis/mpide/hardware/pic32/compiler/pic32-tools/bin/pic32-ar  rcs  /tmp/build7676535287638374045.tmp/core.a  /tmp/build7676535287638374045.tmp/HardwareSerial_cdcacm.c.o
/home/lluis/mpide/hardware/pic32/compiler/pic32-tools/bin/pic32-ar  rcs  /tmp/build7676535287638374045.tmp/core.a  /tmp/build7676535287638374045.tmp/wiring_shift.c.o
/home/lluis/mpide/hardware/pic32/compiler/pic32-tools/bin/pic32-ar  rcs  /tmp/build7676535287638374045.tmp/core.a  /tmp/build7676535287638374045.tmp/wiring_analog.c.o
/home/lluis/mpide/hardware/pic32/compiler/pic32-tools/bin/pic32-ar  rcs  /tmp/build7676535287638374045.tmp/core.a  /tmp/build7676535287638374045.tmp/task_manager.c.o
/home/lluis/mpide/hardware/pic32/compiler/pic32-tools/bin/pic32-ar  rcs  /tmp/build7676535287638374045.tmp/core.a  /tmp/build7676535287638374045.tmp/wiring.c.o
/home/lluis/mpide/hardware/pic32/compiler/pic32-tools/bin/pic32-ar  rcs  /tmp/build7676535287638374045.tmp/core.a  /tmp/build7676535287638374045.tmp/wiring_pulse.c.o
/home/lluis/mpide/hardware/pic32/compiler/pic32-tools/bin/pic32-ar  rcs  /tmp/build7676535287638374045.tmp/core.a  /tmp/build7676535287638374045.tmp/HardwareSerial_usb.c.o
/home/lluis/mpide/hardware/pic32/compiler/pic32-tools/bin/pic32-ar  rcs  /tmp/build7676535287638374045.tmp/core.a  /tmp/build7676535287638374045.tmp/exceptions.c.o
/home/lluis/mpide/hardware/pic32/compiler/pic32-tools/bin/pic32-ar  rcs  /tmp/build7676535287638374045.tmp/core.a  /tmp/build7676535287638374045.tmp/WString.cpp.o
/home/lluis/mpide/hardware/pic32/compiler/pic32-tools/bin/pic32-ar  rcs  /tmp/build7676535287638374045.tmp/core.a  /tmp/build7676535287638374045.tmp/Print.cpp.o
/home/lluis/mpide/hardware/pic32/compiler/pic32-tools/bin/pic32-ar  rcs  /tmp/build7676535287638374045.tmp/core.a  /tmp/build7676535287638374045.tmp/HardwareSerial.cpp.o
/home/lluis/mpide/hardware/pic32/compiler/pic32-tools/bin/pic32-ar  rcs  /tmp/build7676535287638374045.tmp/core.a  /tmp/build7676535287638374045.tmp/WMath.cpp.o
/home/lluis/mpide/hardware/pic32/compiler/pic32-tools/bin/pic32-ar  rcs  /tmp/build7676535287638374045.tmp/core.a  /tmp/build7676535287638374045.tmp/main.cpp.o
/home/lluis/mpide/hardware/pic32/compiler/pic32-tools/bin/pic32-ar  rcs  /tmp/build7676535287638374045.tmp/core.a  /tmp/build7676535287638374045.tmp/Tone.cpp.o
/home/lluis/mpide/hardware/pic32/compiler/pic32-tools/bin/pic32-g++  -Os  -Wl,--gc-sections  -mdebugger  -mprocessor=32MX795F512L  -o  /tmp/build7676535287638374045.tmp/Blink.cpp.elf  /tmp/build7676535287638374045.tmp/Blink.cpp.o    /tmp/build7676535287638374045.tmp/core.a  -L/tmp/build7676535287638374045.tmp  -lm  -T  /home/lluis/mpide/hardware/pic32/cores/pic32/ChipKIT-application-32MX795F512.ld  -T/home/lluis/mpide/hardware/pic32/cores/pic32/ChipKIT-application-COMMON.ld
/home/lluis/mpide/hardware/pic32/compiler/pic32-tools/bin/pic32-objcopy  -O  ihex  -j  .eeprom  --set-section-flags=.eeprom=alloc,load  --no-change-warnings  --change-section-lma  .eeprom=0  /tmp/build7676535287638374045.tmp/Blink.cpp.elf  /tmp/build7676535287638374045.tmp/Blink.cpp.eep
/home/lluis/mpide/hardware/pic32/compiler/pic32-tools/bin/pic32-bin2hex  -a  /tmp/build7676535287638374045.tmp/Blink.cpp.elf
Binary sketch size: 4088 bytes (of a 520192 byte maximum)

\end{lstlisting}

%\addcontentsline{toc}{chapter}{Apéndice 2: Makefile}
\appendix
\chapter{Apéndice 2: Makefile}\label{chap:apendice2}
%mimateo: lo de Makefile. es para poder compilar en windows...
\lstinputlisting[language=make, caption=Makefile, label=code:Makefile, basicstyle=\scriptsize]{codigo_fuente/Makefile.}

 Analicemos el funcionamiento de nuestro Makefile:\\
\lstinputlisting[language=make, firstline=1, lastline=12, breaklines=true]{codigo_fuente/Makefile.}
 Comenzamos definiendo la localización de \programa{MPIDE}, \programa{avrdude} y el puerto serie. Esto lo hacemos de forma diferente dependiendo de si nos encontremos en Mac OS on en GNU/Linux\protect\footnote{En el caso de Windows el puerto serie suele tener un nombre de la forma COMx, cambiando x por el número de puerto.}. También definimos la localización en la que normalmente podremos encontrar el puerto serie del microcontrolador.
 
\lstinputlisting[language=make, firstline=13, lastline=21, breaklines=true]{codigo_fuente/Makefile.}
 Continuamos declarando la localización de la `toolchain' que viene incluida con \programa{MPIDE}. La localización de \programa{MPIDE} en este caso es la carpeta \$HOME de mi PC (Es necesario modificar este valor dependiendo de la localización de \programa{MPIDE}). Las utilidades de la `toolchain' que viene incluida con \programa{MPIDE} se encuentran siempre en la misma subcarpeta, que en el Makefile es la variable `TOOLCHAIN\_PREFIX', independientemente del sistema operativo empleado.\\ 
 
 \lstinputlisting[language=make, firstline=23, lastline=23, breaklines=true]{codigo_fuente/Makefile.}
 Definimos las opciones para \programa{avrdude}, que usaremos para cargar el programa en la placa y que viene incluido con \programa{MPIDE}.\\
 
 \lstinputlisting[language=make, firstline=25, lastline=27, breaklines=true]{codigo_fuente/Makefile.}
 Definimos la CPU que lleva nuestra placa, que en este caso corresponde con el modelo Microchip 32MX795F12L. También declaramos que placa es\protect\footnote{La placa podría ser cualquiera de los modelos compatibles con Arduino que vende Digilent}, para luego poder usar el archivo de cabecera correspondiente a dicha placa.\\
 
 \lstinputlisting[language=make, firstline=29, lastline=30, breaklines=true]{codigo_fuente/Makefile.}
 El enlazador utiliza dos scripts para el correcto enlazado del programa: uno común a todas las placas ChipKIT (\programa{ChipKIT\-application\-COMMON.ld}) y uno específico según el modelo que usemos (\programa{ChipKIT\-application-32MX795F512.ld}).
 
 \lstinputlisting[language=make, firstline=32, lastline=37, breaklines=true]{codigo_fuente/Makefile.}
 Guardamos las opciones de compilación y enlazado en las variables \$CFLAGS y \$LDFLAGS respectivamente.
 
 \lstinputlisting[language=make, firstline=39, lastline=41, breaklines=true]{codigo_fuente/Makefile.}
 Añadimos al final de las opciones de compilación los directorios donde habrá que buscar los archivos de cabecera de las bibliotecas de terceros.

\lstinputlisting[language=make, firstline=47, lastline=50, breaklines=true]{codigo_fuente/Makefile.}
Buscamos todos los archivos de código fuente que forman parte de la biblioteca Core.
 
\lstinputlisting[language=make, firstline=47, lastline=50, breaklines=true]{codigo_fuente/Makefile.}
 Definimos el directorio temporal en el que se copiaran los archivos compilados de la biblioteca Core y cuales son a partir de los archivos de código fuente.
 
 \lstinputlisting[language=make, firstline=52, lastline=54, breaklines=true]{codigo_fuente/Makefile.}
 Buscamos en la carpeta `src' los ficheros con código fuente en C, C++ o ensamblador.

 \lstinputlisting[language=make, firstline=56, lastline=58, breaklines=true]{codigo_fuente/Makefile.}
 Generamos los nombres de los archivos compilados y ensamblados sustituyendo la extensión de los archivos de código fuente de la carpeta `src'.
 
 \lstinputlisting[language=make, firstline=60, lastline=66, breaklines=true]{codigo_fuente/Makefile.}
 Igual que antes, pero esta vez buscamos los archivos con código fuente dentro de la carpeta `lib' que contiene todas las bibliotecas de terceros.
 
 \lstinputlisting[language=make, firstline=68, lastline=68, breaklines=true]{codigo_fuente/Makefile.}
 Este es el primer objetivo que hay en nuestro Makefile y es el que se ejecutará por defecto cuando ejecutamos \programa{make} sin ningún argumento. Lo utilizamos para definir el objetivo por defecto sin tener que modificar nada más que esta línea del Makefile. En este caso indicamos que este objetivo depende del objetivo `link' que veremos más adelante.
 
 \lstinputlisting[language=make, firstline=70, lastline=81, breaklines=true]{codigo_fuente/Makefile.}
 Compilamos, enlazamos y empaquetamos la biblioteca Core, obteniendo como resultado el archivo \orden{core.a}.
 
 \lstinputlisting[language=make, firstline=83, lastline=91, breaklines=true]{codigo_fuente/Makefile.}
 Estas son las reglas que utilizamos para generar los ficheros con extensión `.o' a partir de su correspondiente archivo de código fuente.
 
 \lstinputlisting[language=make, firstline=95, lastline=107, breaklines=true]{codigo_fuente/Makefile.}
 Estas reglas las utilizaremos en el caso de querer generar archivos que hayan sido solo preprocesados o preprocesados y compilados sin ser ensamblados. Para usarlas habrá que indicar explícitamente que archivo queremos generar, por ejemplo \orden{make src/sketch.s} para obtener el archivo en ensamblador a partir de nuestro código C++ de `sketch.cpp'.
 
 \lstinputlisting[language=make, firstline=109, lastline=111, breaklines=true]{codigo_fuente/Makefile.}
 Con este objetivo realizamos el enlazado de los archivos compilados para generear el ejecutable. En las dependencias se encuentran todos los archivos y reglas necesarias para llevar acabo el enlazado.
 
 \lstinputlisting[language=make, firstline=113, lastline=116, breaklines=true]{codigo_fuente/Makefile.}
 Este objetivo tiene la finalidad de enlazar el programa usando un script de enlazado diferente, que no inserta un bootloader. De esta forma podremos crear un ejecutable que puede ser usado en \programa{MPLAB X} para depuración.
 
 \lstinputlisting[language=make, firstline=118, lastline=121, breaklines=true]{codigo_fuente/Makefile.}
 De la misma forma que lo hace \programa{MPIDE} creamos el archivo con extensión `.hex', que cargaremos mas tarde en el microcontrolador. Este objetivo depende del objetivo \orden{link}, ya que necesitamos el programa enlazado.
 
 \lstinputlisting[language=make, firstline=123, lastline=124, breaklines=true]{codigo_fuente/Makefile.}
 Este es el objetivo que hemos definido antes como el objetivo por defecto. Es el que se encarga de invocar a \programa{avrdude} que cargará nuestro programa en la placa.
 
 \lstinputlisting[language=make, firstline=127, breaklines=true]{codigo_fuente/Makefile.}
 Por último, el objetivo \orden{clean} nos permitirá eliminar todos los archivos creados durante el proceso de compilación, enlazado y carga.


\appendix
\chapter{Apéndice 3: Funcionamiento de Sorted}\label{chap:apendice3}
Veamos la estructura y funcionamiento de nuestro `sketch'\protect\footnote{Archivo donde definimos las funciones \comando{setup()} y \comando{loop()}}.\\
\lstinputlisting[firstline=1, lastline=3, breaklines=true]{codigo_fuente/sketch.cpp}
Comenzamos incluyendo la biblioteca del Shield Ethernet, IO Shield y el archivo de cabecera \programa{funciones.h} que contiene la declaración de los diferentes algoritmos de ordenación que usaremos en los comandos expuestos anteriormente.\\

\lstinputlisting[firstline=5, lastline=6]{codigo_fuente/sketch.cpp}
Definimos el tamaño máximo del mensaje, esto es, la máxima longitud de la línea que el cliente puede envíar a nuestro programa, esto incluye el comando seguido por el array. También definimos el tamaño máximo del array que acepta el programa, siendo este la mitad que el mensaje. Esta longitud debería ser suficiente ya que, con números de un solo dígito (o carácter en este caso), no es posible superar este tamaño ya que por cada uno de estos numeros deberemos añadir un separador\protect\footnote{El caracter que separe los números puede ser tanto un espacio una coma o tabulador.}, lo que implica un mínimo de dos caracteres por dígito, sin contar el comando.

\lstinputlisting[firstline=8, lastline=11]{codigo_fuente/sketch.cpp}
Estos son los arrays que usaremos para guardar el mensaje recibido del cliente, el comando extraído de ese mensaje, el array de números que pueda haber en el mensaje y si es o no el primer mensaje que recibimos del cliente.\protect\footnote{El protocolo telnet suele enviar ciertos caracteres al iniciar una conexión que descartamos.}

\lstinputlisting[firstline=15, lastline=33]{codigo_fuente/sketch.cpp}
Creamos la estructura algoritmo con la que podremos asociar cómodamente el nombre en forma de cadena de caracteres de la función con su nombre. Justo después declaramos un array de estructuras algoritmo que contiene esta información. Se usará a la hora de mostrar los resultados de los algoritmos de ordenación.

\lstinputlisting[firstline=36, lastline=37]{codigo_fuente/sketch.cpp}
Aquí declaramos la dirección física que queramos que tenga nuestro dispositivo. Si la inicializamos a 0 la dirección física que se usará será la del propio dispositivo. En el caso de elegir un valor diferente será este el que se use\protect\footnote{En algunas aplicaciones puede ser interesante modificar este valor para enmascarar la identidad del dispositivo o para suplantar la de otro, pero esto se sale del objetivo de este proyecto.}.

\lstinputlisting[firstline=40, lastline=41]{codigo_fuente/sketch.cpp}
Ahora declaramos la dirección de red (IP) de nuestro dispositivo\protect\footnote{Se ha elegido esta en concreto (10.0.0.8) porque durante el desarrollo del programa el dispositivo se encontraba conectado a la red 10.0.0.0/24}. Es posible no declarar la dirección IP del dispositivo para usar DHCP.

\lstinputlisting[firstline=52, lastline=54]{codigo_fuente/sketch.cpp}
Creamos una instancia de la clase Server, con nombre \comando{server} en el puerto 23. Es posible elegir otros puertos, pero se ha elegido este por ser el que usa el servicio de terminal remoto Telnet.

\lstinputlisting[firstline=56, lastline=60]{codigo_fuente/sketch.cpp}
Declaramos funciones auxiliares que se utilizarán a lo largo del programa:
\begin{description}
	\item[\comando{print\_array}] es una función que imprime un array tanto por el puerto serie como al cliente.
	\item[\comando{selecciona\_comando}] se encarga de llamar a la función \comando{ejecuta\_comando} pasándole como parámetro el númor de comando adecuado, el array a ordenar y la longitud de dicho array.
	\item[\comando{array\_aleatorio}] genera un array de números aleatorio de longitud \comando{MAX\_LONG\_ARRAY}.
	\item[\comando{ejecuta\_comando}] ejecuta el comando cuyo índice le pasemos como parámetro.
	\item[\comando{compara}] llama a \comando{ejecuta\_comando} para cada función de ordenación que tengamos, pasándole como parámetro el mismo array de números aleatorios a todas.
\end{description}

\lstinputlisting[firstline=62, lastline=75]{codigo_fuente/sketch.cpp}
Función \comando{setup()} en la que inicializamos el shield ethernet con la función \comando{Ethernet.begin(mac, ip)}\protect\footnote{Si quisiéramos usar DHCP en lugar de una IP fija podemos llamar a la función \comando{Ethernet.begin(mac)}.}, inicializamos el servidor para que comience a escuchar a la espera de nuevos clientes e inicializamos el puerto serie. Este último se usa principalmente para facilitar la depuración del programa, puesto que el acceso será via Telnet no por el puerto serie. Además inicializaremos la pantalla oled del IO Shield y mostraremos en ella el nombre del programa.

\lstinputlisting[firstline=77, lastline=82]{codigo_fuente/sketch.cpp}
Comienza la función \comando{loop()}. Lo primero que hace es inicializar la variable cliente, tomando el resultado de \comando{server.available()}. Declaramos e inicializamos las variables que necesitaremos a lo largo de \comando{loop()}.

\lstinputlisting[firstline=84, lastline=90, breaklines=true]{codigo_fuente/sketch.cpp}
Si \comando{cliente} no es nulo significa que un cliente se ha conectado. Si este es el primer mensaje del cliente lo descartamos, ya que Telnet envía caracteres al iniciar la conexión. Inicializamos a \comando{0} todos los arrays que usaremos para guardar el mensaje, comando y números una vez hayan sido convertidos.

\lstinputlisting[firstline=92, lastline=111, breaklines=true]{codigo_fuente/sketch.cpp}
El método \comando{client.available()} devuelve la cantidad de bytes que hay disponible en el buffer de recepción mientras que \comando{client.read()} nos permite leerlos uno a uno (cada vez que llamamos a este método sin argumentos lee el siguiente byte hasta terminar el buffer). Cuando ya hemos leído todo el buffer, que debido a su implementación está limitado a 1000 bytes, \comando{client.read()} devolverá \comando{-1}. Cuando esto ocurre nos aseguramos de no guardar este valor y mientras siga devolviendo esto, lo ignoramos. Cada vez que leemos un byte aumentamos la variable \comando{long\_mensaje} asegurándonos de no superar el tamaño máximo de mensaje. Seguimos leyendo hasta que recibamos el carácter \comando{\textbackslash n} (retorno de carro) que guardaremos en el mensaje y en este momento dejaremos de leer del buffer de recepción.

\lstinputlisting[firstline=116, lastline=149, breaklines=true, resetmargins=true]{codigo_fuente/sketch.cpp}
Una vez tenemos guardado el mensaje en el array \comando{message}, procedemos dividirlo en el comando y el array de números que ordenaremos más tarde. Para realizar la división buscamos en el array los separadores, esto lo hacemos en el primer \comando{if} que hay nada más comenzar una iteración. En este caso los separadores pueden ser espacios, comas, tabuladores y retorno de carro\protect\footnote{Están incluídos tanto \textbackslash r como \textbackslash n para que funcione con cualquier tipo de retorno de carro, ya sea CR o LF. De está forma podemos conectarnos mediante telnet, que finaliza las líneas con \textbackslash r\textbackslash n o directamente mediante TCP usando, por ejemplo \programa{netcat} o una función dentro de algún programa.}. Utilizamos la variable \comando{cantidad\_num} para identificar el comando y para conocer la cantidad de números que hay en el mensaje. Cada vez que encontremos uno de los separadores esta variable se incrementa. La primera vez que encontramos un separador y, por lo tanto, la variable \comando{cantidad\_num} es nula, sabemos que se trata del comando, ya que este debe ser el primer elemento del mensaje. En este momento y copiamos el contenido del mensaje hasta el índice actual a la variable comando (comprobando antes la longitud del array a copiar para evitar desbordamientos).\\
A partir de aquí, todo lo que debe preceder al comando deben ser números. Antes de convertir los números a enteros, los extraemos, guardándolos en un array temporal llamado \comando{numero}. Para hacer esto usamos un nuevo índice (\comando{j}) que indicará la posición dentro del array \comando{numero}. Este índice se reinicia cada vez que encontramos un separador, mientras que aumenta en cada iteración cuando no lo encotremos. De está forma podemos copiar los números contenidos entre dos separadores al array \comando{numero}. Cada vez que encontremos un separador intaremos convertir \comando{numero} a un entero que guardaremos en el array de enteros \comando{array\_numeros}. En el caso de no ser válido, indicamos por el puerto serie que ha habido un error al convertir.\\
En el caso de que el mensaje fuera correcto, al terminar este bucle tendremos el array \comando{array\_numeros} que contiene todos los números yla variable \comando{cantidad\_num} que indica la cantidad de números en dicho array\protect\footnote{Esta variable es necesaria puesto que generalmente no se llenará el array al completo.}. 

\lstinputlisting[firstline=151, lastline=158, breaklines=true]{codigo_fuente/sketch.cpp}
Si \comando{cantidad\_num} es mayor que 0, es decir, el mensaje no contenía solo el comando, guardamos la longitud de \comando{array\_numeros} en \comando{longitud} y envíamos por puerto serie tanto la longitud de este array como el comando que se ha extraído del mensaje.

\lstinputlisting[firstline=160, lastline=168, breaklines=true]{codigo_fuente/sketch.cpp}
Por si acaso, vacíamos el buffer de recepción con \comando{cliente.flush\(\)}. Llamamos a la función \comando{selecciona\_comando} que se encargará de obtener el número de comando a partir del array \comando{comando} y mostramos al cliente y por el puerto serie el contenido de \comando{array\_numeros} que debería estar ya ordenado.

Veamos el resto de funciones que podemos encontrar en el archivo `sketch.cpp' y que hemos utilizado en la función \comando{loop()}.

\lstinputlisting[firstline=173, lastline=182, breaklines=true]{codigo_fuente/sketch.cpp}
Está función tiene como único objetivo envíar al cliente y por puerto serie el array que pasemos como parámetro, para poder visualizarlo.

\lstinputlisting[firstline=188, lastline=215, breaklines=true]{codigo_fuente/sketch.cpp}
\centerline{\raisebox{-1pt}[0pt][0pt]{$\vdots$}}
\lstinputlisting[firstline=231, lastline=247, breaklines=true]{codigo_fuente/sketch.cpp}
\comando{selecciona\_comando} recibe como parámetros el array \comando{comando} y la longitud de \comando{array\_numeros}. En base al contenido de \comando{comando} llamará a \comando{ejecuta\_comando} con el número de comando adecuado o realizará las operaciones necesarias. Para el caso del comando \comando{help} enviará al cliente una lista con los camndos disponibles.\footnote{Se ha omitido el contenido de \comando{help} para reducir el tamaño.} Si el comando es \comando{compara} llamará a la función del mismo nombre. Si el comando es \comando{exit} mostrará al cliente que se está desconectando, se desconectará indicándolo por el puerto serie y cambiará el valor de la variable \comando{primer\_mensaje} además de reiniciar la pantalla OLED del IO Shield. Si el contenido de \comando{comando} no coincide con ninguno de los comandos disponibles se le indicará al cliente.

\lstinputlisting[firstline=249, lastline=283, breaklines=true]{codigo_fuente/sketch.cpp}
\comando{ejecuta\_comando} se encarga, como su nombre indica, de ejecutar el comando cuyo índice se haya pasado como parámetro. A parte de ejecutar el algoritmo de ordenación adecuado, \comando{ejecuta\_comando} mide también el tiempo que tarda en ejecutarse dicho algoritmo, usando para ello las funciones \comando{millis\(\)} y \comando{micros\(\)} que nos permiten medir el tiempo en milisegundos y microsegundos respectivamente\footnote{Usamos ambas ya que si la duración de la ejecución del algoritmo es alta el contador que usa la funcion \comando{micros\(\)} desbordará, sin embargo, si la duración es muy baja, con \comando{millis\(\)} no tendremos suficiente precisión.}. Como alternativa esta función también se encarga de poner a nivel alto el pin 70 del ChipKIT MAX32, que corresponde al LED LD1 de IO Shield, justo antes de ejecutar el comando y luego cambia su nivel a bajo otra vez, lo que nos permite, por ejemplo, medir la duración con un osciloscopio y tener un cierto `feedback' cuando se está ejecutando un algoritmo.\footnote{El encendido y apagado del LED lo hacemos usando directamente los registros del microcontrolador, evitando la función que nos ofrece la biblioteca Core: \comando{digitalWrite\(\)}, de forma que es lo más rápido posible.} Una vez ha terminado la ejecución mostramos al cliente a través de la conexión Telnet y en la pantalla OLED del IO Shield el tiempo que ha tardado en ejecutarse.

\lstinputlisting[firstline=288, lastline=292, breaklines=true]{codigo_fuente/sketch.cpp}
\comando{array\_aleatorio} simplemente rellena \comando{array\_numeros} con números aleatorios entre 0 y \(2^{31}\). Genera un array de tamaño máximo, es decir, \comando{MAX\_LONG\_ARRAY}.

\lstinputlisting[firstline=298, breaklines=true]{codigo_fuente/sketch.cpp}
En \comando{compara} generamos un array con números aleatorios con la función \comando{array\_aleatorio} y ejecutamos todos los algoritmos de ordenación sobre el mismo array. Debido a que los comandos de ordenación modifican el array cuya dirección pasamos como parámetro, es necesario realizar una copia usando la función \comando{memcpy} a un array auxiliar para que todos ordenen el mismo array de números. De esta forma podemos compararlos con un solo comando.




\appendix
\chapter{Apéndice 4: Funciones de ordenación desensambladas}\label{chap:apendice4}

\begin{figure}
\begin{center}
\begin{tabular}{c c}
	\begin{minipage}[t]{.4\textwidth}
		\lstinputlisting[language={[mips]Assembler},caption=bubble0,basicstyle=\ttfamily\tiny, tabsize=2, firstline=6, lastline=81]{codigo_fuente/funciones_objdump}
	\end{minipage} 
	&
	\begin{minipage}[t]{.4\textwidth}
		\lstinputlisting[language={[mips]Assembler},caption=bubble1,basicstyle=\ttfamily\tiny, tabsize=2, firstline=85, lastline=110]{codigo_fuente/funciones_objdump}
		\lstinputlisting[language={[mips]Assembler},caption=bubble2,basicstyle=\ttfamily\tiny, tabsize=2, firstline=114, lastline=136]{codigo_fuente/funciones_objdump}
		\lstinputlisting[language={[mips]Assembler},caption=bubble3,basicstyle=\ttfamily\tiny, tabsize=2, firstline=140, lastline=162]{codigo_fuente/funciones_objdump}
	\end{minipage} \\
\end{tabular}
\end{center}
\label{bubble_ensamblador}
\end{figure}

\begin{figure}
\begin{center}
\begin{tabular}{c c}
	\begin{minipage}[t]{.4\textwidth}
		\lstinputlisting[language={[mips]Assembler},caption=colocar0,basicstyle=\ttfamily\tiny, tabsize=2, firstline=166, lastline=257]{codigo_fuente/funciones_objdump}
	\end{minipage} 
	&
	\begin{minipage}[t]{.4\textwidth}
		\lstinputlisting[language={[mips]Assembler},caption=quicksort0,basicstyle=\ttfamily\tiny, tabsize=2, firstline=261, lastline=299]{codigo_fuente/funciones_objdump}
	\end{minipage} \\
\end{tabular}
\end{center}
\label{quick0}
\end{figure}

\begin{figure}
\begin{center}
\begin{tabular}{c c}
	\begin{minipage}[t]{.4\textwidth}
		\lstinputlisting[language={[mips]Assembler},caption=colocar1,basicstyle=\ttfamily\tiny, tabsize=2, firstline=303, lastline=331]{codigo_fuente/funciones_objdump}
	\end{minipage} 
	&
	\begin{minipage}[t]{.4\textwidth}
		\lstinputlisting[language={[mips]Assembler},caption=quicksort1,basicstyle=\ttfamily\tiny, tabsize=2, firstline=335, lastline=364]{codigo_fuente/funciones_objdump}
	\end{minipage} \\
\end{tabular}
\end{center}
\label{quick1}
\end{figure}

\begin{figure}
\begin{center}
\begin{tabular}{c c}
	\begin{minipage}[t]{.4\textwidth}
		\lstinputlisting[language={[mips]Assembler},caption=colocar2,basicstyle=\ttfamily\tiny, tabsize=2, firstline=368, lastline=400]{codigo_fuente/funciones_objdump}
	\end{minipage} 
	&
	\begin{minipage}[t]{.4\textwidth}
		\lstinputlisting[language={[mips]Assembler},caption=quicksort2,basicstyle=\ttfamily\tiny, tabsize=2, firstline=404, lastline=432]{codigo_fuente/funciones_objdump}
	\end{minipage} \\
\end{tabular}
\end{center}
\label{quick2}
\end{figure}

\begin{figure}
\begin{center}
\begin{tabular}{c c}
	\begin{minipage}[t]{.4\textwidth}
		\lstinputlisting[language={[mips]Assembler},caption=colocar3,basicstyle=\ttfamily\tiny, tabsize=2, firstline=436, lastline=468]{codigo_fuente/funciones_objdump}
	\end{minipage} 
	&
	\begin{minipage}[t]{.4\textwidth}
		\lstinputlisting[language={[mips]Assembler},caption=quicksort3,basicstyle=\ttfamily\tiny, tabsize=2, firstline=472, lastline=500]{codigo_fuente/funciones_objdump}
	\end{minipage} \\
\end{tabular}
\end{center}
\label{quick3}
\end{figure}


\begin{figure}
\begin{center}
\begin{tabular}{c c}
	\begin{minipage}[t]{.4\textwidth}
		\lstinputlisting[language={[mips]Assembler},caption=selection0,basicstyle=\ttfamily\tiny, tabsize=2, firstline=532, lastline=606]{codigo_fuente/funciones_objdump}
	\end{minipage} 
	&
	\begin{minipage}[t]{.4\textwidth}
		\lstinputlisting[language={[mips]Assembler},caption=selection1,basicstyle=\ttfamily\tiny, tabsize=2, firstline=610, lastline=642]{codigo_fuente/funciones_objdump}
	\end{minipage} \\
\end{tabular}
\end{center}
\label{selection0_1}
\end{figure}

\begin{figure}
\begin{center}
\begin{tabular}{c c}
	\begin{minipage}[t]{.4\textwidth}
		\lstinputlisting[language={[mips]Assembler},caption=selection2,basicstyle=\ttfamily\tiny, tabsize=2, firstline=646, lastline=675]{codigo_fuente/funciones_objdump}
	\end{minipage} 
	&
	\begin{minipage}[t]{.4\textwidth}
		\lstinputlisting[language={[mips]Assembler},caption=selection3,basicstyle=\ttfamily\tiny, tabsize=2, firstline=679]{codigo_fuente/funciones_objdump}
	\end{minipage} \\
\end{tabular}
\end{center}
\label{selection2_3}
\end{figure}


\end{document}
