% -*-cap2.tex-*-
% Este fichero es parte de la plantilla LaTeX para
% la realización de Proyectos Final de Carrera, protejido
% bajo los términos de la licencia GFDL.
% Para más información, la licencia completa viene incluida en el
% fichero fdl-1.3.tex

% Copyright (C) 2009 Pablo Recio Quijano 

En este capitulo analizaré las características del Arduino Mega 2560\footnote{En la introducción menciono el Arduino Mega ADK, pero la única diferencia entre ambos es que este último puede hacer de Host USB para poder conectar un teléfono móvil Android.} y del ChipKIT MAX32, que es el equivalente en chipKIT tanto en formato y dimensiones, como en precio.

\section{Arduino Mega 2560}
La familia Arduino se compone de multitud de placas microcontroladoras con características diferentes pero hay dos que se pueden destacar sobre las demas debido a que son las más utilizadas.  Estas dos placas son la Arduino Uno y Arduino Mega 2560. La placa Arduino Uno es la más básica y económica, con un formato reducido, mientras que la segunda tiene un precio superior y mejores características.

\figura{ArduinoMega2560_R3_Front.jpg}{scale=0.1}{Arduino Mega 2560}{mega2560}{H}

Arduino Mega 2560 es una placa microcontroladora basada en el chip ATmega2560 que fue introducida al mercado durante el último trimestre de 2010.  Es una actualización del Arduino Mega, basada en el chip ATmega1280.  Dispone de 54 pines de entrada/salida (14 de estos pines pueden ser usado como salidas PWM), 16 entradas analógicas, 4 UARTs (puertos serie que se pueden usar para conectar la placa al ordenador, a periféricos o a otra placa microcontroladora), un oscilador de cristal de 16 MHz, un conector de alimentación, un botón de reinicio y un conector ICSP.

\begin{table}
\begin{center}
\begin{tabular}{| l | c |}
\hline
Microcontrolador 						& ATmega2560 \\
Voltaje									& 5V \\
Voltaje de alimentación (recomendado) 	& 7-12V \\
Voltaje de alimentación (límites) 		& 6-20V \\
Pines Entrada/Salida digitales			& 54 (14 salidas PWM) \\
Pines de entrada analógicos				& 16 \\
Corriente por cada pin entrada/salida 	& 40 mA \\
Corriente en el pin de 3.3V				& 50 mA \\
Memoria Flash 							& 256 kB (8 kB usados por el bootloader) \\
SRAM									& 8 kB \\
EEPROM									& 4 kB \\
Frecuencia del reloj					& 16 MHz \\
\hline
\end{tabular}
\end{center}
\caption{Especificaciones de Arduino Mega 2560.}
\label{tab:mega2560_specs}
\end{table}
% Cosas que mencionar
% Cantidad de memoria
% Voltaje (5V)
% JTAG Pero en las lineas de los ADC
% ADC (pag 275 datasheet) 'solo' 15ksps
Una de las carácterísticas de las placas Arduino es su voltaje de funcionamiento de 5V. Este voltaje se puede obtener desde la conexión USB al PC ya que la línea de alimentación de este es también de 5V o se puede usar una fuente de alimentación externa con un voltaje de entre 7V y 12V.

El Arduino Mega 2560 dispone de 256kB de memoria flash.\footnote{En los microcontroladores de Atmel esto viene indicado en el propio nombre, en este caso 2560 hace referencia a sus 256kB de memoria Flash.} Esta cantidad es suficiente para la mayoría de proyectos a los que estaría destinada esta placa. Por otro lado su memoria SRAM es de tan solo 8kB, lo que puede ser el factor limitante en algunos proyectos.

Esta placa tiene también la posibilidad de depurar sus programas a través del puerto JTAG del Atmega 2560. En lugar de tener una conexión dedicada, los pines del puerto JTAG están situados en los mismos dedicados a las entradas analógicas, lo que dificulta la depuración de código que haga uso de estas entradas.

Otra característica que se puede destacar de este Arduino son los convertidores analógico-digital que tiene incorporados el microcontrolador. Hay dos ADCs en los que podemos encontrar 8 pines de entrada multiplexados en cada uno. Su resolución es de 10 bits y su máxima frecuencia de muestreo (con toda la resolución) es de 15ksps. Como veremos después, este valor es bajo y puede resultar determinante dependiendo del proyecto en que se vaya a usar. Es posible obtener una mayor frecuencia de muestreo, pero a cambio de reducir la resolución.

\subsection*{Atmega 2560}
La placa Mega 2560 recibe su nombre del microcontrolador que lleva instalado, un Atmel Atmega 2560. Todos los periféricos que están incluidos en la placa los podemos encontrar integrados en este chip, de forma que el resto de componentes electrónicos son los dedicados a las entradas y salidas o a la alimentación de la propia placa, con la excepción del chip Atmega 16u2 que se encarga de gestionar la conexión a través del puerto USB con el ordenador.

El núcleo del Atmega 2560 está basado en una arquitectura AVR de 8 bits. AVR utiliza una arquitectura Harvard, de forma que la CPU dispone de buses independientes para datos y programa. El ciclo de instrucción está segmentado en dos partes: mientras una instrucción es ejecutada, la siguiente está siendo obtenida desde la memoria Flash. De esta forma se ejecuta una instrucción por ciclo siempre y cuando no haya dependencias entre las instrucciones.
% Cosas para mencionar
% Frecuencia de reloj
% 8 Bits
% Arquitectura AVR
% Pipeline de 2 fases
% NO MDU
% NO FPU
% 32 Registros
% No MMU
\figura{atmel2560_diagrama.jpg}{scale=0.7}{Diagrama de la arquitectura AVR}{atmel2560}{H}
Como se puede ver en la figura \ref{atmel2560} el núncleo AVR consta de 32 registros de 8 bits de propósito general en los que se almacenan los datos que serán procesados en la unidad aritmético-lógica (ALU) que hay. Esta unidad soporta operaciones aritméticas y lógicas entre registros o entre un registro y una constante guardando el resultado de nuevo en los registros, en un solo ciclo. Estas operaciones son llevadas a cabo sobre operandos de 8 bits (el tamaño de cada registro) por lo que cualquier operación sobre números enteros de 32 bits (operación más habitual en programas) se requerirá más ciclos. Las operaciones aritméticas que puede realizar esta ALU son suma, resta y multiplicación de números de coma fija, sin embargo no dispone de una instrucción de división, por lo que es necesario usar una que funcione mediante software. Tampoco tiene una unidad de coma flotante (FPU) por lo que estamos limitados a números enteros o a la emulación de coma flotante a través de bibliotecas, que es lo que sucede cuando usamos variables de tipo `float' durante el desarrollo de una aplicación para Arduino.

\section{ChipKIT Max32}
El microcontrolador elegido para el trabajo es un ChipKIT Max32. Es el modelo más avanzado de la familia Chipkit, que se compone además del Uno32, uC32 y DP32.

\figura{chipKIT-Max32.jpg}{scale=0.6}{Chipkit Max32}{max32}{H}

En el centro del Max32 encontramos como microcontrolador un Microchip PIC32MX795F12L que funciona a 80MHz y con el que disponemos de 512KB de memoria flash y 128KB de memoria RAM. El Max32 tiene 83 puertos de entrada/salida además de otros dispositivos como UART, SPI e I2C. Dieciséis de los 83 puertos puede ser usados como entradas analógicas o como entradas/salidas digitales. A diferencia de su equivalente Arduino (Mega), el ChipKIT Max32 también tiene integrado en el chip 10/100 Ethernet, un controlador USB 2.0 y dos controladores CAN, aunque para hacer uso de estos dispositivos necesitaremos el ChipKIT Network Shield.\\

Este dispositivo, al igual que el resto de microcontroladores de la familia ChipKIT, funciona con un voltaje lógico de 3.3V frente a los microcontroladores Arduino como el Uno o el Mega. Es importante tener este dato en cuenta a la hora de usar shields o dispositivos pensados para placas Arduino.\\

Podemos encontrar más información además de esquemáticos y manuales en la página de información del ChipKIT Max32\cite{website:max32}.\\

Características del ChipKIT Max32:
\begin{itemize}
	\item Procesador Microchip PIC32MX795F512L
		\begin{itemize}
			\item 80MHz 32-bit MIPS
			\item 512KB FLash, 128KB RAM
			\item Controlador USB 2.0 OTG
			\item 10/100 Ethernet MAC
			\item Dos controladores CAN
		\end{itemize}
	\item Proporciona memoria adicional (con respecto al Uno32) y periféricos de comunicación avanzados
	\item Compatible con gran parte del código existente de ejemplo para Arduino, así como con materiales de referencia y otros recursos.
	\item Puede ser programado usando Microchip MPLAB, mediante un programador gracias a su interfaz JTAG.
	\item Mismo tamaño y formato que el Arduino Mega
	\item 83 pines de entrada/salida
	\item LED de usuario (Pin 13)
	\item Se conecta a un PC usando un cable USB A -> mini B
\end{itemize}

\subsubsection{Chipkit PGM}
Para la depuración de los programas con MPLAB usaremos el programador chipKIT PGM\cite{website:pgm}, con el que podremos usar el puerto JTAG del Max32. Este programador está diseñado para ser usado exclusivamente con la familia de dispositivos chipKIT, a diferencia de picKIT3\cite{website:pickit} que puede ser usado con todos los microcontroladores de Microchip.

\figura{chipKIT-PGM.png}{scale=0.3}{ChipKIT PGM}{pgm}{H}

\newpage

\subsection{ChipKIT Network Shield}

El ChipKIT Network Shield es, como su nombre indica, un shield para el ChipKIT Max32. Cabe destacar que este shield no sirve para el ChipKIT Uno32 puesto que este microcontrolador no dispone de las características de las que hace uso este shield, como Ethernet y host USB.Se coloca encima del microcontrolador tal y como lo hace cualquier shield de Arduino.

\figura{chipKIT-NetworkShield.jpg}{scale=0.5}{Chipkit Network Shield}{networkshield}{H}

Este shield nos proporciona las conexiones necesarias para poder hacer uso del controlador Ethernet y USB que lleva el Max32 integrado y a los buses CAN. Además de esto también tiene integrado un oscilador de 32.768 KHz que nos permite hacer uso del RTC que también tiene integrado el Max32 y una memoria EEPROM de 256Kb a la que podemos acceder mediante I2C. En la página web del Network Shield\cite{website:network_shield} podemos encontrar las bibliotecas que nos sirven para hacer uso de sus características, en el entorno ChipKIT.\\

Características del ChipKIT Network Shield:
\begin{itemize}
	\item Puede ser usado con el Max32
	\item SMSC LAN8720 10/100 Ethernet PHY
	\item Conector RJ45
	\item Dispositivo USB y conectores Host
	\item Dos transceptores CAN MCP2551
	\item Dos conectores de 12 pines para los buses CAN
	\item Dos conectores para cadena margarita (daisy chain) I2C
	\item EEPROM de 256Kbits con interfaz I2C
	\item Oscilador a 32.768 KHz
\end{itemize}

\newpage


\subsection{ChipKIT Basic I/O Shield}
El otro shield que usaremos será el ChipKIT Basic I/O Shield que nos servirá para interactuar físicamente con el microcontrolador ya que este shield nos proporciona diferentes métodos de entrada y salida de información.

\figura{chipKIT-BasicIOShield.jpg}{scale=0.5}{ChipKIT Basic I/O Shield}{ioshield}{H}

El ChipKIT Basic I/O shield dispone de los siguientes dispositivos:
\begin{itemize}
	\item EEPROM de 256Kbits con interfaz I2C
	\item Pantalla OLED de 128x32 píxeles
	\item Sensor de temperatura con interfaz I2C
	\item Conector para cadena margarita (daisy chain) I2C
	\item Cuatro drivers FET de colector abierto
	\item Potenciometro analógico
	\item Cuatro pulsadores y cuatro interuptores
	\item Ocho LEDs
\end{itemize}

Este shield está inicialmente pensado para ser usado en un chipKIT Uno32, pero podemos conectarlo al Max32 sin ningún problema ya que comparten el mismo esquema de pines (aunque el Max32 tiene 83 pines frente a los 42 del Uno32). También es posible conectarlo al Network Shield y este a su vez al Max32, teniendo en cuenta la posibilidad de que algunas funciones en los dos shields puedan compartir pines.\\

Igual que el Network Shield, el Basic I/O Shield también dispone de una página web\cite{website:io_shield} en la que podemos encontrar su manual, esquemáticos y las librerías necesarias para usarlas en el entorno ChipKIT.