% -*-cap2.tex-*-
% Este fichero es parte de la plantilla LaTeX para
% la realización de Proyectos Final de Carrera, protejido
% bajo los términos de la licencia GFDL.
% Para más información, la licencia completa viene incluida en el
% fichero fdl-1.3.tex

% Copyright (C) 2009 Pablo Recio Quijano 

\section{Hardware}
En esta sección se describirán los dispositivos que van a ser utilizados para la realización del proyecto.
\subsection{ChipKIT Max32}
El microcontrolador elegido para el proyecto es un ChipKIT Max32. Es el modelo más avanzado de la familia Chipkit, que se compone además del Uno32, uC32 y DP32.

\figura{chipKIT-Max32.jpg}{scale=0.6}{Chipkit Max32}{max32}{h}

En el centro del Max32 encontramos como microcontrolador un Microchip PIC32MX795F12L que funciona a 80MHz y con el que disponemos de 512KB de memoria flash y 128KB de memoria RAM. El Max32 tiene 83 puertos de entrada/salida además de otros dispositivos como UART, SPI e I2C. Dieciséis de los 83 puertos puede ser usados como entradas analógicas o como entradas/salidas digitales. A diferencia de su equivalente Arduino (Mega), el ChipKIT Max32 también tiene integrado en el chip 10/100 Ethernet, un controlador USB 2.0 y dos controladores CAN, aunque para hacer uso de estos dispositivos necesitaremos el ChipKIT Network Shield.\\
Este dispositivo, al igual que el resto de microcontroladores de la familia ChipKIT, funciona con un voltaje lógico de 3.3V frente a los microcontroladores Arduino como el Uno o el Mega. Es importante tener este dato en cuenta a la hora de usar shields o dispositivos pensados para placas Arduino.
\subsection{ChipKIT Network Shield}
\subsection{ChipKIT I/O Shield}
\section{Software}
\subsection{MPIDE}
\subsection{MPLAB X}
