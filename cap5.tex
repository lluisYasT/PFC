En este capítulo expondré las conclusiones obtenidas durante la elaboración de este trabajo. Se subdivide en dos apartados: conclusiones técnicas, donde comentaré los resultados de las mediciones y el proceso usado para obtenerlas; conclusiones personales, donde expondré mis pensamientos acerca de este trabajo junto con las cosas que he podido aprender durante la realización del mismo.

\subsection{Conclusiones técnicas}
Este trabajo tenía diferentes objetivos. El primero de ellos era analizar el proceso de compilación, enlazado y carga en la placa que lleva a cabo el IDE de Microchip (\programa{MPIDE}) de forma transparente al usuario. Una vez conocido este proceso había que desarrollar un método alternativo que me permitiera más transparencia y flexibilidad.

El segundo objetivo planteado era comparar el efecto de los distintos niveles de optimización tenían sobre el rendimiento final del programa. Para esto se proponía la creación de un programa en el que se pudiese realizar esta comparación.

Una vez hecho esto, el siguiente objetvio era poner de manifiesto las limitaciones de la biblioteca Arduino frente a funciones de más bajo nivel que el fabricante provea en una biblioteca o directamente usando los registros del procesador y de periféricos.

Por último, se proponía la creación de un programa que pusiese en práctica conocimientos adquiridos durante la carrera. El programa tendría como objetivo detectar la presencia de un tono en una entrada analógica cuya señal se obtendría con un micrófono.
