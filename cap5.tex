En este capítulo expondré las conclusiones obtenidas durante la elaboración de este trabajo. Las conclusiones se han dividido en dos conjuntos: las coclusiones técnicas y las personales. En las  conclusiones técnicas comentaré los resultados del proyecto: cumplimiento de los objetivos, las mediciones obtenidas y el proceso usado para obtenerlas. En el apartado de conclusiones personales expondré mis pensamientos acerca de este trabajo y qué creo que he aprendido durante su realización.

\subsection{Conclusiones técnicas}

% Vamos a ver... 
% Esto son conclusiones, no puedes ir diciendo con pelos y señales lo que has hecho.
% Es mejor que te centres en los resultados obtenidos haciendo enlaces a las secciones 
% donde describes el proceso llevado a cabo.
% Lo que SÍ que tienes que haces es decir si se ha alcanzado este objetivo o no.
% Obejtivos técnios:
%  1º) Analizar el proceso de compilación del MPIDE
%  2º) Averiguar el efecto de los diferentes niveles de optimización
%  3º) Demostrar las limitaciones de Arduino (biblioteca)
%  4º) Desarrollo de una aplicación en el ámbito de las telecomunicaciones.
%
% ¿Por qué no empieza con este lístado, tal cual, y luego vas diciendo si se ha alcanzado
% o no?
% 

Los objetivos planteados al inicio de este trabajo eran los siguientes:

\begin{enumerate}
    \item Analizar el proceso de compilación del \programa{MPIDE}.
    \item Averiguar el efecto de los diferentes niveles de optimización.
    \item Demostrar las limitaciones de Arduino (biblioteca).
    \item Desarrollo de una aplicación en el ámbito de las telecomunicaciones.
\end{enumerate}

%mimateo 9/9/2014: fíjate como lo hago yo. Digo que el objetivo de ha cumplido y donde se puede ver como... ahora falta la conclusión de verdad.

El primero estos objetivos se ha alcanzado plenamente. En la sección~\ref{sec:compilacion} se describe el análisis realizado sobre el proceso que lleva a cabo el programa \programa{MPIDE} para compilar, ensamblar, enlazar y cargar los programas en la placa ChipKIT\@.

Alcanzar el segundo objetivo llevó más trabajo al necesitar crear un programa específico con el que se pudiesen realizar las mediciones. Una vez hecho esto, se realizaron las mediciones y se completó el segundo de los objetivos. Todo el proceso está descrito a lo largo de las primeras secciones del capítulo~\ref{cap4}.

Una vez hecho esto se llevó a cabo el tercer objetivo: demostrar las limitaciones de la biblioteca Arduino. Las principales ventajas de la biblioteca son la sencillez de uso y la portabilidad, pero se ha demostrado que estas ventajas tienen como precio que las aplicaciones son más lentas. Esto se consiguió comparando las funciones que implementa esta biblioteca con el uso de los registros del microcontrolador directamente y se pudo comprobar la pérdida de rendimiento al usar la biblioteca con respecto a los registros.  Se describe este objetivo en la sección~\ref{sec:Limitaciones de la biblioteca Arduino.}.

Por último, se proponía la creación de un programa relacionado con el ámbito de las telecomunicaciones. Para la realización de este objetivo se optó por un programa típico de un DSP\@. Se creó un programa que permitiera al microcontrolador detectar tonos mediante el algoritmo de Goertzel, como se puede ver en la sección~\ref{sec:Ejemplo de aplicación relacionada con la carrera de telecomunicación.}.

\subsection{Conclusiones personales}

% yo empezaría las conclusiones con esta frase.
En general, la realización de este proyecto me ha parecido un trabajo muy interesante y variado.


% ya se que lo has dicho antes, pero recuerda dónde has puesto la lista de asignaturas relacionadas.
% si no has puesto esa lista antes, sería el momento de ponerlas.

En este trabajo he tenido la posibilidad de poner en práctica los conocimientos adquiridos a lo largo de toda la carrera. Me ha servido para aplicar los conceptos, no sólo de una asignatura en concreto, sino los de varias de ellas, que se han mencionado en la introducción. De esta forma he podido ver la relación entre las mismas. Esto es algo que no había podido hacer hasta este momento, ya que en el resto 
de trabajos realizados durante el transcurso de la carrera se solían centrar en la asignatura para la que se desarrollaban o una parte de ella.

% Latex, Git : esto hay que citar de donde lo has sacado y añadir entradas en la bibliografia...

No sólo he tenido que aplicar los conocimientos adquiridos en la carrera sino que también me he visto en la necesidad de aprender otros nuevos para poder llevar a cabo este trabajo. Por ejemplo, esta memoria está escrita usando Latex. Hasta ahora no había utilizado nunca este lenguaje de marcado por lo que he tenido que aprender a usarlo~\cite{website:latex-wikibooks}. Otro ejemplo de conocimiento adquirido sería el uso del software de control de versiones Git~\cite{git} que me ha permitido llevar un registro de las modificaciones hechas en la memoria. También ha sido de gran utilidad a la hora de desarrollar los programas para este trabajo ya que otorga gran flexibilidad en el momento de corregir errores o de implementar nuevas características sin “romper” la versión estable del programa. Junto con Git, he utilizado el servicio de almacenamiento y gestión de repositorios de \url{http://github.com}, que es dónde está almacenado tanto este trabajo como los programas asociados.

Este trabajo también ha sido el de mayor envergadura que he llevado a cabo. Nunca antes he realizado un trabajo de tanta magnitud e importancia. Esto ha supuesto todo un reto para mi, sobre todo a la hora de escribir la memoria que es la parte que mas me ha costado.

%Y ahora una línea final de conclusion
A nivel personal este trabajo ha resultado interesante y con el que he completado mi formación como Ingeniero de Telecomunicación.
