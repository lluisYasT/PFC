% -*-programas.tex-*-
% Este fichero es parte de la plantilla LaTeX para
% la realización de Proyectos Final de Carrera, protejido
% bajo los términos de la licencia GFDL.
% Para más información, la licencia completa viene incluida en el
% fichero fdl-1.3.tex

% Copyright (C) 2009 Pablo Recio Quijano 
Para la realización de esta memoria he utilizado diferentes programas.

\section*{Vim}
\programa{Vim} es uno de los editores de texto más utilizado en entornos Unix, enfocado en programación. La ventaja de usar este editor es que está preinstalado en cualquier sistema Unix (puede estar instalado Vim o Vi) y es altamente personalizable mediante su archivo de configuración y con plugins.

\section*{Sublime Text + Latexing}
Otro editor de textos que he utilizado para la creación de esta memoria es \programa{Sublime Text} que, a diferencia de \programa{Vim}, es de pago. Al igual que \programa{Vim} es posible ampliar su funcionalidad con plugins.

\section*{Git}
Para mantener un control de versiones he optado por utilizar \programa{Git}. Con \programa{Git} puedo, no solo guardar las versiones, sino también crear diferentes ramas o utilizar un servicio como Github.com, que me permite almacenar tanto esta memoria como los programas que he creado para este trabajo.

\section*{GNU Make}

\programa{GNU Make} permite la automatización de procesos como pueden ser, por ejemplo, compilar varios archivos que contengan código fuente y enlazarlos para crear un ejecutable. También puede ser usado para compilar archivos latex y crear un fichero en formato pdf.

Para más información \cite{pdf:make}
