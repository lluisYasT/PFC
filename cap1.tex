% -*-cap1.tex-*-
% Este fichero es parte de la plantilla LaTeX para
% la realización de Proyectos Final de Carrera, protejido
% bajo los términos de la licencia GFDL.
% Para más información, la licencia completa viene incluida en el
% fichero fdl-1.3.tex

% Copyright (C) 2009 Pablo Recio Quijano 

\section{Contexto} % (fold)
% Arduino
% Microcontroladores
% Internet of Things
% Segunda revolución del garaje
%
\label{sec:Contexto}
Un microcontrolador es un circuito integrado programable que se compone de varios bloques funcionales, ente los que encontramos una CPU, memoria y periféricos.~\cite{website:microcontrolador} Con ayuda de un número mínimo de componentes externos, el microcontrolador permite crear pequeños circuitos que realicen tareas determinadas, que pueden ir desde la creación de sensores inteligentes hasta facilitar el manejo de un avión teledirigido. Un microprocesador, por contra, no incorpora memoria programable ni periféricos, por lo que necesita un mayor número de componentes externos para poder funcionar.

%mimateo: ¡¡Cuidado!!
%mimateo: El 4004 fue el primer microporcesador pero no el primer microcontrolador... Buscando en Google dan ese honor al TMS 1000 de Texas Instruments

En 1974 Gary W. Bone, en Texas Instruments, creó el primer microcontrolador de la histora: el TMS 1000 de 4 bits, después de varias disputas legales con Intel~\footnote{http://www.datamath.org/Story/Intel.htm}. Desde entonces, los microcontroladores han ido evolucionando y en la actualidad existen multitud de modelos diferentes entre los que elegir basados en diversas arquitecturas como pueden ser AVR, ARM-M o MIPS.\@

Un microcontrolador ejecuta las instrucciones del programa que se ha grabado en su memoria. Estos programas estaban elaborados originalmente en código ensamblador, aunque en la actualidad la mayoría se programa usando el lenguaje C. Para programar un microcontrolador hay que estar familiarizado con el funcionamiento del modelo concreto que se vaya a programar. No es lo mismo programar un microcontrolador ARM que un PIC32 o un Atmel AVR.\@ A pesar de que en todos ellos se utiliza C como principal lenguaje de programación, la velocidad de ejecución, las estructuras de traducción y los dispositivos internos difieren, en muchos casos sustancialmente. Por ejemplo, para escribir en un pin de un GPIO puede ser necesario configurar diferentes registros dependiendo del microcontrolador.

%mimateo: yo lo pondría al revá: primero el problema y luego la propuesta de solución
Para minimizar estos inconvenientes, en el año 2003, se comenzó a trabajar en la plataforma Wiring~\footnote{http://wiring.org.co/},  en el instituto IVREA, en Italia. Wiring proporcionaba una capa de abstracción sobre las funciones y registros de algunos microcontroladores. Esta plataforma está ideada con los diseñadores y artistas en mente ya que las plataformas de prototipado electrónico de aquel entonces estaban enfocadas a gente con una formación técnica en este ámbito. Con Wiring se pretendía reducir la dificultad del diseño electrónico y la programación. Dos años más tarde, nació Arduino que, junto con Wiring, ofrecía una plataforma de hardware y software abierto con la que cualquier persona, con unos conocimientos básicos de programación, podía crear sus propios proyectos.
% me falta una referencia en WIRING...

% section Contexto (end)

\section{Introducción}
Como alumno de la Escuela Técnica Superior de Ingenieros de Telecomunicación he tenido la oportunidad de aprender a programar diferentes dispositivos y arquitecturas en varias asignaturas. Comencé usando el lenguaje C en ordenadores con arquitectura x86 en la asignatura Programación del primer curso.  Más adelante, en el tercer curso en la asignatura `Sistemas Electrónicos Digitales', programé en ensamblador del microprocesador Motorola 68000.

\begin{figure}[H]
\begin{center}
\begin{tikzpicture}[
		decoration={
			markings,
			mark=at position 1 with {\arrow[ultra thick]{latex}},
		},
		path/.style={
			postaction=decorate,
		},
%		every node/.style={font=\sffamily}
	]
	% Boxes
	\node[
		draw,
		xshift=-6cm,
		minimum width=2cm,
	] (hid_in) {HID};
	\node[
		draw,
		minimum width=2cm,
		minimum height=2cm,
	] (pc) {PC};
	\node[
		draw,
		xshift=6cm,
		minimum width=2cm,
	] (hid_out) {HID};
	\node[
		draw,
		xshift=-6cm,
		yshift=1cm,
		minimum width=2cm,
] (mem_in) {Memoria};
	\node[
		draw,
		xshift=6cm,
		yshift=1cm,
		minimum width=2cm,
] (mem_out) {Memoria};
	\node[
		draw,
		xshift=-6cm,
		yshift=-1cm,
		minimum width=2cm,
] (net_in) {Red};
	\node[
		draw,
		xshift=6cm,
		yshift=-1cm,
		minimum width=2cm,
] (net_out) {Red};

% Conexiones
\draw[path] (mem_in.east) -- ++(2,0) node[above,pos=0.3]{digital} |- ([yshift=-0.5cm] pc.north west);
\draw[path] (hid_in.east) -- ++(2,0) node[above,pos=0.3]{digital} |- (pc.west);
\draw[path] (net_in.east) -- ++(2,0) node[below,pos=0.3]{digital} |- ([yshift=0.5cm] pc.south west);

\draw[path] ([yshift=-0.5cm] pc.north east) -- ++(2,0)  |- node[above,pos=0.7]{digital}(mem_out.west);
\draw[path] (pc.east) -- ++(2,0) |- node[above,pos=0.7]{digital} (hid_out.west);
\draw[path] ([yshift=0.5cm] pc.south east) -- ++(2,0) |- node[below,pos=0.7]{digital} (net_out.west);

\end{tikzpicture}
\end{center}
\caption{Origen y Destino de los datos procesados en un Ordenador}
\label{io_pc}
\end{figure}
%mimateo: lo de STDIN no estoy seguro ni que de naciera en UNIX. Lo que sí tengo claro es que se usa en muchos más sistemas operativos, de ahí la modificación que he realizado
Cuando se empieza a programar, normalmente se hace en un ordenador personal, con un sistema operativo de propósito general que ofrece el concepto de máquina extendida. En este entorno, el sistema operativo se encarga de gestionar los recursos disponibles en el sistema. De esta forma, cuando se escribe un programa, no es necesario acceder a bajo nivel a los dispositivos, tan solo hay que usar las bibliotecas que nos proporciona el sistema operativo y que usarán los drivers que alguien ha creado para usar el dispositivo en el sistema operativo, típicamente ese alguien es el propio fabricante del dispositivo.

Con el M68000 tuve la oportunidad de programar al más bajo nivel, al no disponer de un sistema operativo, accediendo directamente a los dispositivos a las rutinas de manejo de interrupciones. En lugar de obtener los datos desde un fichero o la entrada estándar (`STDIN'\footnote{Normalmente está asociado al teclado en un programa de terminal, teclado controlado por ese programa o por el sistema operativo.}) lo hacía directamente desde un teclado, desde el puerto serie o incluso a través de un convertidor analógico-digital (ADC), dispositivos que era necesario configurar ya que, debido a la falta de sistema operativo, no disponía de las funciones necesarias para controlarlos.
 
\begin{figure}[H]
%mimateo: en este figura falta una salida analógica pura
\begin{center}
\begin{tikzpicture}[
		decoration={
			markings,
			mark=at position 1 with {\arrow[ultra thick]{latex}},
		},
		path/.style={
			postaction=decorate,
		},
%		every node/.style={font=\sffamily}
	]
	% Boxes
	\node[
		draw,
		xshift=-6cm,
		minimum width=2cm,
	] (hid_in) {HID};
	\node[
		draw,
		xshift=-6cm,
		yshift=2.5cm,
		minimum width=2cm,
		text width = 3.5cm,
	] (sensores) {Sensores: audio, temperatura, humedad, luz \dots};
	\node[
		draw,
		xshift=-6cm,
		yshift=-2.5cm,
		minimum width=2cm,
		text width = 3.5cm,
	] (conexiones) {Conexiones Serie/paralelo: SPI, UART, Wire, CAN \dots};
	\node[
		draw,
		xshift=-0.5cm,
		minimum width=2cm,
		minimum height=3.5cm,
	] (uc) {\(\mu\)C};
	\node[
		draw,
		xshift=5cm,
		yshift=2.5cm,
		minimum width=2cm,
		text width=3.5cm,
	] (otros) {Otros elementos digitales: LEDs, Relés \dots};
	\node[
		draw,
		xshift=5cm,
		minimum width=2cm,
	] (hid_out) {HID};
	\node [
		draw,
		xshift=-6cm,
		yshift=1cm,
		minimum width=2cm,
	] (mem_in) {Memoria};
	\node [
		draw,
		xshift=5cm,
		yshift=1cm,
		minimum width=2cm,
	] (mem_out) {Memoria};
	\node[
		draw,
		xshift=-6cm,
		yshift=-1cm,
		minimum width=2cm,
	] (net_in) {Red};
	\node[
		draw,
		xshift=5cm,
		yshift=-1cm,
		minimum width=2cm,
	] (net_out) {Red};
	\node[
		draw,
		xshift=5cm,
		yshift=-2cm,
		minimum width=2cm,
	] (pwm) {PWM};
	\node[
		draw,
		xshift=4.5cm,
		yshift=-3.5cm,
		minimum width=2cm,
		text width=3.5cm,
	] (otros_analog) {Elementos analógicos: Altavoces, Buzzers, LEDs \dots};

% Conexiones
\draw[path] (sensores.east) -- ++(2,0) node[above,pos=0.5]{analógico} |- ([yshift=-0.5cm] uc.north west);
\draw[path] (mem_in.east) -- ++(2,0) node[above,pos=0.3]{digital} |- ([yshift=-1cm] uc.north west);
\draw[path] (hid_in.east) -- ++(2,0) node[above,pos=0.3]{digital} |- (uc.west);
\draw[path] (net_in.east) -- ++(2,0) node[below,pos=0.3]{digital} |- ([yshift=1cm] uc.south west);
\draw[path] (conexiones.east) -- ++(2,0) node[below,pos=0.5]{digital} |- ([yshift=0.5cm] uc.south west);

\draw[path] ([yshift=-0.5cm] uc.north east) -- ++(1,0)  |- node[above,pos=0.7]{digital}(otros.west);
\draw[path] ([yshift=-1cm] uc.north east) -- ++(2,0)  |- node[above,pos=0.7]{digital}(mem_out.west);
\draw[path] ([yshift=-1.5cm]uc.north east) -- ++(2,0) |- node[above,pos=0.7]{digital} (hid_out.west);
\draw[path] ([yshift=1.5cm] uc.south east) -- ++(2.5,0) |- node[below,pos=0.7]{digital} (net_out.west);
\draw[path] ([yshift=1cm] uc.south east) -- ++(2.25,0) |- node[below,pos=0.7]{digital} (pwm.west);
\draw[path] (pwm.east) -- ++(1,0) node[above]{analógico} |-  (otros_analog.east);
\draw[path] ([yshift=0.5cm] uc.south east) -- ++(1.75,0) node[below,pos=0.5]{analógico} |- (otros_analog.west);

\end{tikzpicture}
\end{center}
\caption{Origen y Destino de los datos procesados en un Microcontrolador.}
\label{io_uc}
\end{figure}
Esta plataforma tenía dos inconvenientes que hacían inviable su uso fuera de un entorno de aprendizaje o profesional.  Primero, era necesario programar todo en ensamblador, no pudiendo usar las abstracciones que nos otorga un lenguaje como C, o al menos no se podía hacer de una forma sencilla.  Además, el kit que se usaba en esta asignatura (SED) es difícil de encontrar y, en el caso de hacerlo, su precio es prohibitivo, superando los 1000\$\footnote{http://www.promaxprolink.com/training/tm683.htm}.  Ese mismo año también realicé prácticas de programación con VHDL en FPGAs que también tenían los mismos problemas que el Motorola 68000: alto precio y programación de muy bajo nivel.

% sería interesante que dijeras el año exacto o la edición de las Google IO a las que te refieres
Un año más tarde conocí la plataforma Arduino, a raíz del lanzamiento del Arduino ADK durante la Google I/O del 2011.\@ La plataforma presentada se componía de una placa con un microcontrolador Atmel2560 de 8 bits y era de un tamaño mucho menor que los kits que había usado a lo largo de la carrera.  El precio era de 60~\euro{} para ese modelo, que era el más caro en ese momento. Aún así, era mucho más asequible que las plataformas de prototipado que había utilizado hasta ese momento.  

Uno de los aspectos más importantes de esta plataforma es que no es necesario conocer el microcontrolador para llevar a cabo proyectos en Arduino.  Por ejemplo, para programar el parpadeo de un LED todo lo que hay que averiguar es a qué pin (numerados en la placa) está conectado el diodo LED.\@ Se puede programar usando C o C++ y cuenta con un entorno de desarrollo que permite la utilización de múltiples funciones que hacen posible programar a un nivel más alto que al que se haría si usáramos las herramientas que proporciona Atmel para sus microcontroladores.  

El hardware y el software de Arduino son abiertos.  Es posible obtener los esquemáticos de cualquier placa Arduino desde su página web y la biblioteca principal (en la que se basa cualquier proyecto Arduino) viene incluida con el entorno de programación como código fuente.  Esto significa que existe la posibilidad de comprar una placa oficial a través de la tienda de Arduino o algún distribuidor, pero también es posible construirla uno mismo o encontrar placas personalizadas a través de Internet que lleven caracterísiticas no incluidas en las oficiales o que sean más baratas.

A esto se suma la posibilidad de utilizar el entorno Arduino y muchas de sus bibliotecas de funciones incluso con placas que no estén basadas en los mismos procesadores que las oficiales. En este tipo de placas se debe distribuir algún compilador específico para las mismas así como una implementación de las bibliotecas básicas de funcionamiento de Arduino.

Un ejemplo de este tipo de placas es la que he empleado en este trabajo: ChipKIT MAX32.  A diferencia de las placas Arduino convencionales, la gama ChipKIT de Digilent tiene como microcontrolador un PIC de 32 bits en vez de un Atmega de 8 bits.  Está placa ofrece más potencia y más prestaciones como, por ejemplo, la posibilidad de depurar nuestro programa gracias al conector JTAG que lleva incorporado. Las diferencias están comentadas más a fondo en el siguiente capítulo.

\section{Objetivos}
La utilización de una placa de la gama chipKit implica la utilización de una nueva herramienta de programación, mucho más potente que la original de Arduino. En esta plataforma disponemos de un control casi completo de la forma en la que se generarán los archivos compilados y es en este tipo de control en los que hemos centrado este PFC.\@

Teniendo en cuenta esta orientación al estudio a bajo nivel de los programas gnerados con el entorno Arduino, podemos dividir los objetivos de la siguiente forma:

\subsubsection{Analizar el proceso de compilación de \programa{MPIDE}}
Lo primero que realizaré será un análisis del proceso que se sigue en el entorno de desarrollo de ChipKIT desde que escribimos el programa hasta que este esté funcionando en la placa microcontroladora.  Esto incluye todas las órdenes o programas que intervengan en este proceso y sus respectivas opciones, además de los ficheros involucrados.

Para lograr flexibilidad en este proceso, propondré una alternativa al IDE, que hará uso de Makefiles y me permitirá realizar todo el proceso anterior con una sola orden. También podré generar resultados intermedios como, por ejemplo, el código compilado, pero no ensamblado, y podré modificar las opciones de cada una de las órdenes.

\subsubsection{Comparar el efecto de los distintos niveles de optimización}
El compilador ofrece diferentes niveles de optimización y según el nivel se llevan a cabo unas u otras optimizaciones. Uno de los objetivos será identificar estas optimizaciones y ver su efecto en el código. Usando una aplicación desarrollada para el microcontrolador pondré de manifiesto las diferencias entre los distintos niveles de optimización, qué ventajas y qué desventajas tienen cada uno de ellos.

\subsubsection{Poner de manifiesto las limitaciones de la biblioteca Arduino}
La biblioteca Arduino nos ofrece una capa de abstracción sobre las funciones específicas del microcontrolador pero que tiene limitaciones. Comprobaré la diferencia entre usar esta biblioteca o acudir a las bibliotecas del fabricante; en este caso, Microchip.

\subsubsection{Desarrollar una aplicación enfocada en el ámbito de las telecomunicaciones}
Desarrollaré una aplicación en la que se pondrán en práctica conceptos relacionados con las telecomunicaciones, aprendidos durante la carrera. La función de la aplicación será detectar tonos mediante el uso del algoritmo de Goertzel.


\section{Estructura de la memoria}
Este trabajo se organiza en diferentes capítulos en los que trataré temas específicos.

En el capítulo~\ref{cap2} hablaré sobre los dispositivos hardware empleados en el trabajo. Esto incluirá una descripción de la placa microcontroladora elegida junto con los accesorios que la acompañan. La descripción de la placa incluirá también un apartado dedicado a su chip microcontrolador.

En el capítulo~\ref{cap3} describo el apartado software de la plataforma Chipkit. Realizaré una descripción de los programas que se pueden usar para programar una placa Chipkit: \programa{MPIDE} y \programa{MPLAB}. En este capítulo también trataré el primero de los objetivos, ya que analizaré el proceso de compilación de \programa{MPIDE} y propondré una alternativa al mismo.

Usando los elementos tratados en los dos capítulos anteriores, en el capítulo~\ref{cap4} se llevan a cabo el resto de objetivos planteados. Comenzaré tratando los niveles de optimización y sus distintos efectos. Pondré a prueba los distintos niveles con la ayuda de una aplicación que corra en el Chipkit Max32. Una vez hecho esto comentaré las posibles limitaciones de la plataforma Arduino y como es posible solventar alguna de ellas. Para finalizar el capítulo describiré una aplicación orientada a la carrera. Como ya he comentado en los objetivos, se tratará de la implementación de un detector de tonos usando el algoritmo de Goertzel, una aplicación típica de DSPs pero que se puede implementar también en el Max32.

Por último, en el capítulo~\ref{cap5} comentaré mis conclusiones, tanto técnicas como personales, sobre lo que ha supuesto este trabajo.

Al final de la memoria también se pueden encontrar varios apéndices que desarrollan más detalladamente ciertos temas como, por ejemplo, la descripción detallada del archivo Makefile que utilizo como alternativa al IDE de Chipkit o el funcionamiento del programa `Sorted' que uso en el capítulo~\ref{cap4} para medir las diferencias entre niveles de optimización.
