% -*-cap1.tex-*-
% Este fichero es parte de la plantilla LaTeX para
% la realización de Proyectos Final de Carrera, protejido
% bajo los términos de la licencia GFDL.
% Para más información, la licencia completa viene incluida en el
% fichero fdl-1.3.tex

% Copyright (C) 2009 Pablo Recio Quijano 

\section{Introducción}
%% TODO

\section{Objetivos}
Podemos dividir los objetivos de la siguiente forma:

\subsection{Proceso de compilación}
Lo primero que realizaré será un análisis del proceso que se sigue en el entorno de desarrollo de ChipKIT desde que escribimos el programa hasta que este esté funcionando en la placa microcontroladora.  Esto incluye todos los comandos o programas que intervengan en este proceso y sus respectivas opciones, además de los ficheros involucrados.\\
Para lograr flexibilidad en este proceso, propondré una alternativa al IDE que hará uso de Makefiles y me permitirá realizar todo el proceso anterior con un solo comando. También podré generar resultados intermedios, como por ejemplo el código compilado, pero no ensamblado, y podré modificar las opciones de cada uno de los comandos.

\subsection{Niveles de optimización}
El compilador ofrece diferentes niveles de optimización y según el nivel se llevan a cabo unas u otras optimizaciones. Uno de los objetivos será identificar estas optimizaciones y ver su efecto en el código. Usando una aplicación desarrollada para el microcontrolador pondré de manifiesto las diferencias entre los distintos niveles de optimización, que ventajas y que desventajas tienen cada uno de ellos.

\subsection{Limitaciones de la biblioteca Arduino}
La biblioteca Arduino nos ofrece simplicidad a la hora de llevar acabo operaciones sencillas, es una capa de abstracción sobre las funciones específicas del microcontrolador. Comprobaré la diferencia entre usar esta biblioteca o acudir a las bibliotecas del fabricante, en este caso Microchip.

\subsection{Desarrollo de una aplicación enfocada en el ámbito de las telecomunicaciones}
Desarrollaré una aplicación en la que se pondrán en práctica conceptos relacionados con las telecomunicaciones, aprendidos durante la carrera. La función de la aplicación será detectar tonos mediante el uso del algoritmo de Goertzel.

\section{Mapa de la memoria}